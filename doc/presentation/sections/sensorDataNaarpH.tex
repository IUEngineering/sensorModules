\section{Sensor data naar pH omzetten}

\subsection*{Uitlezen ISFET}

    \begin{frame}
        \frametitle{Principe schakeling}
    
        \begin{figure}
            \centering
            \def\svgwidth{0.6\textwidth}
            \input{ISFETCircuitBest.pdf_tex}
        \end{figure}
    
    \end{frame}
    \begin{frame}
        \frametitle{Ruis analyse}
    
        \begin{figure}
            \centering
            \def\svgwidth{0.6\textwidth}
            \input{ISFETCircuitBestNoise.pdf_tex}
        \end{figure}
        \begin{equation}\label{eq:measureNoiseOut}
            S_{u_{{n,out}}} = \left(S_{u_{{n,ref}}} + S_{u_{{n,n}}} + S_{i_{{n,in}}}\left(Z_{fet} // R\right)^2\right) \cdot H^2(\ph)
        \end{equation}
    
    \end{frame}
    \begin{frame}
        \frametitle{Energie}

        $U_{dd}=3v3$

        \noindent
        $I_{ds}=50\mu$A
    
        \begin{equation}\label{eq:measurePower}
            P_{statisch} = P_{n,quiescent} + U_{dd}I_{ds}
        \end{equation}

        \pause

        \begin{equation}
            P_{statisch} = P_{n,quiescent} + 165\mu\mathrm{W}
        \end{equation}
    
    \end{frame}

    \subsection*{ADC}
    \begin{frame}
        \frametitle{Eisen}
    
        \begin{table}[ht]
    \centering
    \begin{tabular}{l|c|l}
        Symbol      & Waarde & Eenheid\\\hline
        $SNR_{in}$  & 37        & dB\\
        NF          & 3         & dB\\
        $u_{in}$    & 2.5       & mV\\
    \end{tabular}
    \caption{De eisen voor het omzetten van het analoge signaal naar een digitaal signaal.}
    \label{tab:systemSpecADC}
\end{table}
    
    \end{frame}
    \begin{frame}
        \frametitle{Minimum aantal bits}
        \centering

        De toelaatbare fout ten gevolge van de eindige resolutie van de ADC
        \begin{equation}\label{eq:calcSpecifiedRmsError}
            \overline{e_{eff}^2}=\left(10^{\frac{NF}{10}}-1\right)\left(\frac{S_{rms}}{10^{\left(SNR+NF\right)/20}}\right)^2
        \end{equation}
        \pause

        Berekenen minimum benodigde ADC resolutie
        \begin{equation}\label{eq:calcNeededQ}
            Q=\sqrt{12\cdot\overline{e_{eff}^2}}
        \end{equation}
        \pause

        Berekenen minimum aantal bits van de ADC op basis van de minimaal benodigde ADC resolutie 
        \begin{equation}\label{eq:calcMinNumberADCbits}
            n=\left\lceil\log_2\left(\frac{1}{Q}+1\right)\right\rceil=14
        \end{equation}
    
    \end{frame}

    \begin{frame}
        \frametitle{Sample frequentie}
        \centering
        
        Toelaatbare fout
        \begin{equation}\label{eq:ADCmaxSampleError}
            E=10^{\frac{-NF}{10}}
        \end{equation}
        \pause

        Minimale sample frequentie berekenen
        \begin{equation}\label{eq:ADCminFs}
            f_{s,min}=\frac{\pi f_h}{E}=45
        \end{equation}
    
    \end{frame}

    \subsection*{AA filter}
    \begin{frame}
        \frametitle{Eisen anti aliasing filter}
        
        \centering

        Dempen 22.5Hz


        $P_{max}=200\mu$W


        $NF=3\si{\decibel}$

        %TODO! Voeg hier een afbeelding van een RC filter
        [Voeg hier een afbeelding van een RC filter]
    
    \end{frame}
    \begin{frame}
        \frametitle{Ruis \& vermogensanalyse RC filter}
    
        \begin{equation}\label{eq:dividerNoise}
            u_{n,out}^2 = \frac{kT}{C}
        \end{equation}

        \begin{equation} \label{eq:filterPower}
            P = \frac{1}{\sqrt{2}}\omega_cCU_{in,max}^2
        \end{equation}

        \pause

        \begin{equation} \label{eq:filterCapMin}
            C_{min} = \frac{kT}{u_{n,in}^2}
        \end{equation}

        \begin{equation}
            R = \frac{1}{2\pi fC}
        \end{equation}
    
    \end{frame}
    \begin{frame}
        \frametitle{AA filter}
    
        \begin{table}[ht]
            \centering
            \begin{tabular}{l|l|l}
                Symbool & Waarde & Eenheid \\
                \hline
                $C$         & 82    & $\si{\nano\farad}$\\
                $R$         & 180   & $\si{\kilo\ohm}$  \\
                $f_c$       & 10.8  & $\si{\hertz}$     \\
                $P$         & 408   & $\si{\nano\watt}$ \\
                $u_{n,out}$ & 225   & $\si{\nano\volt}$ \\
                NF          & 0.23  & $\si{\decibel}$   \\
            \end{tabular}
            \caption{De gekozen waardes van het filter, en de resulterende vermogens- en ruiseigenschappen.}
            \label{tab:filterValues}
        \end{table}
    
    \end{frame}
    \begin{frame}
        \frametitle{Effecten van de load}
    
        Spanningsdeler met de ADC, hierdoor wordt het ingangssignaal kleiner.
    
    \end{frame}

    \subsection*{Berekenen pH}
    \begin{frame}
        \frametitle{Gemeten waarde omrekenen naar pH}
        
        %TODO jochem doe werk/voeg dingen toe
        % $\mathrm{CAL_{\mathrm{pH,H}}}=7$

        % $\mathrm{CAL_{\mathrm{pH,L}}}=4$

        % \begin{equation}
        %     \mathrm{pH}=\frac{\mathrm{S}-\mathrm{CAL_{\mathrm{ADC,L}}}}{\mathrm{CAL_{\mathrm{ADC,H}}}-\mathrm{CAL_{\mathrm{ADC,L}}}}\left(\mathrm{CAL_{\mathrm{pH,H}}}-\mathrm{CAL_{\mathrm{pH,L}}}\right)+\mathrm{CAL_{\mathrm{pH,L}}}
        % \end{equation}
    
    \end{frame}