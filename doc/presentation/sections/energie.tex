\section{Energie}
\begin{frame}
    \frametitle{Energie budget}
    \begin{table}[ht]
        \centering
        \begin{tabular}{l|l}
            Func. blok          & Vermogen [mW] \\
            \hline                              
            Reken $U_{GS}\rightarrow$pH & 0.6   \\
            ADC                 & 1             \\
            AA-filter           & 0.2           \\
            Meet $U_{GS}$       & 0.2           \\
            Zenden              & 5             \\
            Oplader             & 0.5           \\
            Beveiliging         & 0.5           \\
            Spanningsregeling   & 1             \\ 
            \hline
            \hline
            Totaal              & 9
            
        \end{tabular}
        \label{tab:energieBudgetEstimatie}
    \end{table}
    
\end{frame}


    \subsection{Energy Harvesting}
    \begin{frame}
        \frametitle{Energy Harvesting}
        \begin{itemize}
            \item $>$0 mW harvesting
            \item Kan indoor gebruikt worden
        \end{itemize}

    \end{frame}


    \begin{frame}
        \frametitle{Piëzo-elektrisch element}
    
        Zet mechanische trillingen om in elektrische energie.\\
        Gekozen piëzo element:
        \begin{itemize}
            \item Mide Technology PPA-1021 
            \item Maximum 4.5 mW
        \end{itemize}
        \begin{figure}[h]
            \raggedleft
            \includegraphics[scale=0.2]{img/peizo.png}
        \end{figure}

    \end{frame}

    \begin{frame}
        \frametitle{Resultaten Energy Harvesting}
        Pieken van 1.2 mW.
        \begin{figure}[h]
            \raggedright
            \includegraphics[scale=0.40]{img/resultaten_piezo.pdf}
        \end{figure}
    \end{frame}

    \subsection{Batterij}
    \begin{frame}
        \frametitle{Batterij}
        Gekozen voor Lithium-ion technologie
        \begin{itemize}
            \item Multicomp LIR2450
            \item 
            \begin{tabular}{|l|c c c|}
                \hline
                 & min & nom & max \\ \hline
                Spanning    & 2.7 V   & 3.6 V   & 4.2 V\\ \hline
                Capaciteit  & 110 mAh & 120 mAh &      \\ \hline
            \end{tabular}            
        \end{itemize}
        \begin{figure}[h]
            \raggedleft
            \includegraphics[scale=0.3]{img/batterij.png}
        \end{figure}
    \end{frame}
        

    \begin{frame}
        \frametitle{BMS}
        Eisen voor BMS
        \begin{itemize}
            \item Lage stroom verbruik
            \item Beveiligt voor:
                \begin{itemize}
                    \item Onderspanning
                    \item Overspanning
                \end{itemize}
            \item Kan de stroom aan van het systeem
            
        \end{itemize}
        
        
    
    \end{frame}

    \begin{frame}
        \frametitle{LTC4071}
        \begin{itemize}
            \item 550 nA tijdens werking
            \item Max spanning 4.2 V
            \item $<$0.1 nA na `Low battery disconnect'
            \item `Low battery disconnect' instelbaar
                \begin{itemize}
                    \item 2.7 V Lithium Ion (Li-ion)
                \end{itemize}
            \item Bij overspanning, 50 mA shunt naar GND
            \item Stroom uitgang
                \begin{itemize}
                    \item $\pm 60$ mA continu 
                    \item 400 mA kleine pulse van $<$10ms
                \end{itemize}
        \end{itemize}

    \end{frame}

    \subsection{Spanning omzetting}

    \begin{frame}
        \frametitle{PMIC}
        \begin{figure}[h]
            \centering
            \includegraphics[scale=0.6]{img/energie_systeem.pdf}
        \end{figure}
    \end{frame}
    
    \subsection{Verbruik}
    \begin{frame}
        \frametitle{Energieverbruik energie system}
        \begin{tabular}{|l|c|}
        \hline
            & Vermogen \\ \hline
            Buck-boost & 0.13 mW* \\ \hline
            Buck-boost + LDO  & 0.38 mW*\\ \hline
        \end{tabular} \\       
        *LTC4071 verbruik zit bij de metingen.\\
        Metingen zijn gedaan zonder load.
        \vspace{1cm}

            \centering
            Uit specificaties max verbruik spanningsregeling 1 mW.\\ 
            $0.38$ mW $< 1$ mW\\
            Dus spanningsregeling deel voldoet aan max verbruik specificaties. 

    \end{frame}
    \begin{frame}
        \frametitle{Energieverbruik gehele systeem}
        \begin{figure}[h]
            \centering
            \includegraphics[scale=0.35]{img/vermogensMeting.pdf}
        \end{figure}
        \centering
        Max verbruik gehele systeem is 9 mW volgens de specs.
        Gebruik gehele systeem is 6.84 mW\\
        6.84 mW $<$ 9 mW\\
        Dus systeem voldoet aan de opgestelde specificaties.

    \end{frame}

    % \begin{frame}
    %     \frametitle{Demo}
    
        
    
    % \end{frame}