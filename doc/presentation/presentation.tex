\documentclass[compress]{beamer}
\usetheme{Warsaw}

\usepackage[dutch]{babel}

\usepackage[utf8]{inputenc}
\usepackage{amsmath}
\usepackage{amssymb}
\usepackage{txfonts}
\usepackage{graphicx, import}
\usepackage{csquotes}
\usepackage[backend=biber, style=ieee]{biblatex}
\usepackage{pgfplots}
\usepackage{siunitx}
\usepackage{caption}
\usepackage{subcaption}
% \RequirePackage[left=2cm,right=2cm,top=2.2cm,bottom=4cm]{geometry}
% \RequirePackage[pdftex,pdfpagelabels,bookmarks,hyperindex,hyperfigures,hidelinks]{hyperref}
% \RequirePackage{listings}  % for code
\usepackage{xcolor} % needs to be after listings

\newcommand\ph{\mathrm{pH}}

\institute{HvA}
\date{\today}

\begin{document}

\title{pH meten met een ISFET}
\subtitle{Sensor modules}
\author[Tycho Jöbsis \and Jochem Leijenhorst \and Illya Ustenko]{
    {
        % \section*{\hspace*{l}}
        \begin{tabular}{ll}
            Tycho Jöbsis        & (500845792)\tabularnewline
            Jochem Leijenhorst  & (500855372)\tabularnewline
            Illya Ustenko       & (500845492)        
        \end{tabular}
    }
}

    \begin{frame}
        \titlepage
    \end{frame}
    
    \begin{frame}
        \frametitle{Inhoudsopgave}\tableofcontents
    \end{frame} 

    \begin{frame}
        \frametitle{Opdracht}

        Elk ontwikkelteam gaat een eigen systeemontwerp bedenken en verder uitwerken: blokschema, circuitniveau, realisatie, en tenslotte metingen ter validatie.
    
        % Elk ontwikkelteam bestaat uit 3 en soms 4 studenten en gaat zijn eigen systeemontwerp bedenken en verder uitwerken: blokschema, circuitniveau, realisatie, en tenslotte metingen ter validatie.

        Per ontwikkelteam wordt tenminste 1 sensormodule gemaakt. De ontwikkeling en de validatie van de sensormodule en moeten tijdens de eindpresentatie overzichtelijk gestructureerd binnen de tijd worden gepresenteerd.

    
    \end{frame}

    %\section{pH meten}
    %\subsection*{pH meettechnieken}
    \begin{frame}
        \frametitle{pH meettechnieken}
    
        \begin{itemize}
            \item pH-meetstrip
            \item ISFET
            \item Glas-probe
        \end{itemize}
        
    \end{frame}

    %\subsection*{ISFET informatie dragend signaal}
    \begin{frame}
        \frametitle{ISFET informatie dragend signaal}

        De threshold spanning is afhankelijk van de pH.

        Met constante $U_{DS}$ en $I_{DS}$ is $U_{GS}$ lineair afhankelijk van de pH.
    
    
    \end{frame}

    \section{Systeem}
    \begin{frame}
        \frametitle{<title>}
    
        
    
    \end{frame}

    %\section{Sensor data naar pH omzetten}

%\subsection*{Uitlezen ISFET}

    \begin{frame}
        \frametitle{Principe schakeling}
    
        \begin{figure}
            \centering
            \def\svgwidth{0.6\textwidth}
            \input{ISFETCircuitBest.pdf_tex}
        \end{figure}
    
    \end{frame}
    

    

    %\subsection*{Berekenen pH}
    

    \section{Energie}
    \begin{frame}
        \frametitle{<title>}
    
        
    
    \end{frame}

    % \section{Vragen}
    \begin{frame}
        \frametitle{Vragen?}
        
        \centering
        Zijn er nog vragen?
    
    \end{frame}

\end{document}