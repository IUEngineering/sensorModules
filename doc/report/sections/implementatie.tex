\section{Implementatie}
In \cref{sec:ontwerp} is voor alle onderdeel van het \si{\pH} meetsysteem besproken wat de eisen zijn.
Nu zal uitgewerkt worden hoe deze losse systeemonderdelen geïmplementeerd worden.
Als eerste zal ingegaan worden op het ontwerp van de ISFET uitleesschakeling, daarna zullen de benodigde ADC specificaties worden bepaalt waarmee vervolgens een anti aliasing filter ontworpen kan worden. Vervolgens zal er kort op de benodigde temperatuursensor worden ingegaan. Naast de signaal verwerking moet het systeem ook draadloos kunnen communiceren en zal er ingegaan worden op hoe de draadloze communicatie is geïmplementeerd. Uit de hiervoor benoemde paragrafen komen een aantal specificaties voor de microcontroller die in dit systeem worden genoemd. De laatste paragraaf zal ingaan op hoe het systeem wordt gevoed en hoe de energie harvesting is geïmplementeerd.
% Hierbij wordt de zelfde volgorde aan gehouden van \cref{sec:ontwerp}.

%  van dat onderdeel. Dit hoofdstuk gaat in op de implementatie van elk van deze systeemonderdelen.

\subsection{De ISFET uitlezen}
Om een ISFET uit te kunnen lezen moet er een ISFET worden gekozen. Voor dit project moet \si{\pH} tot op \qty{0.05}{\pH} worden gemeten\footnote{Zie de specificaties in \cref{sec:systemSpecifications}.}. Om dit te kunnen doen is er een ISFET met een hoge chemische gevoeligheid nodig. Om hieraan te voldoen is er een ISFET nodig die is gebaseerd op $\mathrm{Ta_2O_5}$. De MSFET 3330-2 ISFET van MICROSENS is gebaseerd op $\mathrm{Ta_2O_5}$ en heeft dus een hoge chemische gevoeligheid \cite{bergveld2003thirtyYearsISFET,bergveld1985impactOfMosfetBasedSensors,isfet}.

De MSFET 3330-2 heeft een gevoeligheid van \qty{-55}{\milli\volt\pH^{-1}} \cite{isfet}. Een verandering van \qty{0.05}{\pH} geeft een spanningsverandering van \qty{2.75}{\milli\volt}. Wanneer de \SNR voor dit kleinste signaal minimale \qty{40}{\decibel} moet zijn, mag de ruisvloer niet hoger zijn dan \qty{27.5}{\micro\volt}. Volgens \cref{eq:measureNoiseFull} is de ruisvloer van de uitleesschakeling opgebouwd uit verschillende ruisbronnen. \Cref{eq:measureNoiseFull2} is een herhaling van \cref{eq:measureNoiseFull}.
\begin{equation}\label{eq:measureNoiseFull2}
    S_{u_{{n,out}}} = \left(S_{u_{{n,ref}}} + S_{u_{{n,n}}} + S_{i_{{n,in}}}\left(Z_{fet} // R\right)^2\right) \cdot \left(\frac{U_{o,max}}{U_{ref}}\right)^2
    \tagaddtext{[\si{\volt\squared\per\hertz}]}
\end{equation}

In de datasheet van de MSFET 3330-2 is aangeraden om de ISFET uit te lezen met een drainstroom van \qty{100}{\micro\ampere} en een drain source spanning van \qty{500}{\milli\volt} \cite{isfet}. Het is dan ook mogelijk om de drain source impedantie te berekenen. $Z_{fet}$ komt dan uit op \qty{5}{\kilo\ohm}.
\begin{figure}[!htbp]
    \centering
    \def\svgwidth{0.4\textwidth}
    \input{img/ISFETCircuitBest.pdf_tex}
    \caption{De schakeling die gebruikt zal worden om de ISFET mee uit te lezen.}
    \label{fig:measureResistorImplementatie}
\end{figure}
De weerstand $R$ in \cref{fig:measureResistorImplementatie} wordt gebruikt om de drainstroom in te stellen. Met een voedingsspanning van \qty{3.3}{\volt} moet $R$ \qty{28}{\kilo\ohm} gekozen worden om een drainstroom van \qty{100}{\micro\ampere} te krijgen. Door \cref{eq:measureNoiseFull2} te gebruiken kan berekend worden dat de spanningsreferentie en de nullor implementatie elk niet meer dan \qty{1.3}{\pico\volt^2\hertz^{-1}} aan ruis mogen genereren.

In \cref{sec:energyBudgets} is gespecificeerd dat de ISFET uitleesschakeling niet meer dan \qty{600}{\micro\watt} mag verbruiken. Omdat de drainstroom is ingesteld op \qty{100}{\micro\ampere} en de voedingsspanning \qty{3.3}{\volt} is zal de uitleesschakeling minimaal \qty{330}{\micro\watt} verbruiken. Hierdoor mogen de spanningsreferentie en de nullor implementatie elk maximaal \qty{135}{\micro\watt} verbruiken.

In de aankomende twee subparagrafen zal ingegaan worden op de implementatie van de spanningsreferentie en de implementatie van de nullor. Hierbij zal rekening worden gehouden met de ruis en vermogens eisen die in deze paragraaf aanbod zijn gekomen.

\subsubsection{Spanningsreferentie}
Zoals besproken in \cref{sec:referenceVoltage} kunnen de weerstandswaardes van de spanningsreferentie erg hoog gekozen worden. Met een $R_1$ van \qty{5.6}{\mega\ohm} gebruikt de spanningsdeler \qty{1.65}{\micro\watt}.

Volgens \cref{eq:dividerNoise} heeft de condensatorwaarde wel effect op de ruis. Met een condensator van \qty{1}{\micro\farad} produceert de spanningsreferentie \qty{64.4}{\nano\volt} aan ruis. Dit zorgt voor een signaal-ruis verhouding van \qty{138}{\decibel}, wat meer dan genoeg is.

Deze gekozen waardes en de resulterende eigenschappen zijn te vinden in \cref{tab:divider}.

\begin{table}[!htbp]
    \centering
    \begin{tabular}{l|l|l}
        Symbool & Waarde & Eenheid \\
        \hline
        $R_1$       & 5.6  & $\si{\mega\ohm}$   \\
        $R_2$       & 1.0  & $\si{\mega\ohm}$   \\
        $C$         & 1.0  & $\si{\micro\farad}$\\
        $P$         & 1.65 & $\si{\micro\watt}$ \\
        $u_{n,out}$ & 64.4 & $\si{\nano\volt}$  \\
        SNR         & 138  & $\si{\decibel}$
    \end{tabular}
    \caption{De gekozen waardes van de spanningsdeler, met het resulterende vermogensverbruik en de ruiseigenschappen.}
    \label{tab:divider}
\end{table}

\begin{figure}[!htbp]
    \centering
    \def\svgwidth{7cm}
    \subsection{Spanningsreferentie}\label{sec:referenceVoltage}

De ISFET uitleesschakeling heeft een spanningsreferentie nodig om te werken. Deze spanningsreferentie kan op meerdere manieren gegenereerd worden.
% TODO: Vertel misschien over andere methoden.
Uiteindelijk is er een spanningsdeler gekozen om de spanningsreferentie mee te implementeren. De schakeling van deze spanningsdeler is te zien in \autoref{fig:divider}.
De condensator wordt gebruikt om ruis te verminderen op hogere frequenties, en dient ook als filter voor hoogfrequente fouten in de voedingsspanning.

\begin{figure}[ht]
    \centering
    \def\svgwidth{0.5\textwidth}
    \subsection{Spanningsreferentie}\label{sec:referenceVoltage}

De ISFET uitleesschakeling heeft een spanningsreferentie nodig om te werken. Deze spanningsreferentie kan op meerdere manieren gegenereerd worden.
% TODO: Vertel misschien over andere methoden.
Uiteindelijk is er een spanningsdeler gekozen om de spanningsreferentie mee te implementeren. De schakeling van deze spanningsdeler is te zien in \autoref{fig:divider}.
De condensator wordt gebruikt om ruis te verminderen op hogere frequenties, en dient ook als filter voor hoogfrequente fouten in de voedingsspanning.

\begin{figure}[ht]
    \centering
    \def\svgwidth{0.5\textwidth}
    \input{img/divider.pdf_tex}
    \caption{De schakeling van de spanningsdeler die dient als spanningsreferentie.}
    \label{fig:divider}
\end{figure}

\noindent
De overdracht van deze spanningsdeler is te vinden in \autoref{eq:dividerTransfer}.
\begin{equation}\label{eq:dividerTransfer}
    H(s) = \frac{U_{ref}(s)}{U_{dd}(s)} = \frac{R_2}{R_1 + R_2 + R_2Cs}
\end{equation}

\noindent
Het vermogen dat de spanningsdeler dissipeert, kan met \autoref{eq:dividerPower} berekend worden.
\begin{equation}\label{eq:dividerPower}
    P(s) = U_{dd}(s)^2\frac{1+R_2Cs}{R_1 + R_2 + R_1R_2Cs}
\end{equation}
Met een constante DC ingangsspanning kan dit vereenvoudigd worden naar \autoref{eq:dividerPowerSimple}.
\begin{equation}\label{eq:dividerPowerSimple}
    P = \frac{U_{dd}^2}{R_1 + R_2}
\end{equation}

\noindent
Om de ruis van deze schakeling te berekenen moet een aantal stappen genomen worden. Aangezien de ingangsbron $U_{dd}$ een spanningsbron is, kan deze als kortsluiting genomen worden. Op deze manier kunnen de twee weerstanden parallel genomen worden, en verandert de schakeling in een simpel RC filter. In \autoref{fig:dividerNoise} is deze omgebouwde schakeling te zien.

\begin{figure}[ht]
    \centering
    \def\svgwidth{0.35\textwidth}
    \input{img/dividerNoise.pdf_tex}
    \caption{De omgebouwde schakeling om ruis mee te berekenen.}
    \label{fig:dividerNoise}
\end{figure}

\noindent
Voor de spectrale spanningsruisdichtheid aan de uitgang $U_{ref}$ kan \autoref{eq:dividerNoiseLaplace} worden opgesteld.
\begin{equation}\label{eq:dividerNoiseLaplace}
    S_{n,u_{ref}} = 4kTR_e\left(\frac{1}{1 + R_eCs}\right)^2
\end{equation}
Wanneer de absolute waarde van de ruis wordt genomen, kan deze over de bandbreedte geïntegreerd worden. Dit resulteert in \autoref{eq:dividerNoiseInt}.
\begin{equation}\label{eq:dividerNoiseInt}
    u_{n,ref}^2 = \int_{\omega_l}^{\omega_h} 4kTR_e\left(\frac{1}{\sqrt{1 + (R_eC\omega)^2}}\right)^2 d\omega
\end{equation}
Het integraal van deze formule komt uit op \autoref{eq:dividerNoiseIntegrated}.
\begin{equation}\label{eq:dividerNoiseIntegrated}
    u_{n,ref}^2 = \frac{4kT}{C}\left[\arctan(R_eC\omega_h) - \arctan(R_eC\omega_l)\right]
\end{equation}

Zolang de spanning stabiel is hoeft de referentie geen exact gedefinieerde spanning te hebben. Dit is omdat de ADC die de waarde van de sensor gaat uitlezen deze referentiespanning ook als referentie zal gebruiken. De waarde moet echter ergens rond de 1.1V zitten, om de stroombron te laten werken.
In \autoref{sec:currentSource} is hier meer over te lezen.
Aangezien de ingangsspanning 1.8V is, zal de DC overdracht $1.1 / 1.8 = \frac{11}{18}$ zijn. Hieruit komt de weerstandsverhouding in \autoref{eq:dividerResistors}.
\begin{equation}\label{eq:dividerResistors}
    \frac{R_1}{R_2} = \frac{7}{11}
\end{equation}
Een hogere $R_1$ zorgt voor een lager vermogensverbruik, maar ook een hogere ruis. Dit is te zien in \autoref{fig:dividerPlots}.

\begin{figure}
    \centering
    \begin{subfigure}[b]{0.45\textwidth}
        \centering
        \input{plots/dividerNoise}
        \caption{Spanningsruis}
        \label{fig:dividerNoisePlot}
    \end{subfigure}
    \hfill
    \begin{subfigure}[b]{0.45\textwidth}
        \centering
        \input{plots/dividerPower}
        \caption{Vermogensverbruik}
        \label{fig:dividerPower}
    \end{subfigure}
    \caption{De ruis en het vermogensverbruik van de spanningsdeler, ten opzichte van de gekozen weerstandswaarde $R_1$.}
    \label{fig:dividerPlots}
\end{figure}
Op $R_1 = 1\si{\mega\ohm}$ en $C = 10\si{\micro\farad}$ is de spanningsruis $u_{n,ref} = 51\si{\nano\volt}$ en het vermogensverbruik $P = 1.3\si{\micro\watt}$. De andere weerstandswaarde is dan $R_2 \approx 1.6 \si{\mega\ohm}$. Deze waardes vallen binnen de specificaties.

[WELKE SPECS??]
    \caption{De schakeling van de spanningsdeler die dient als spanningsreferentie.}
    \label{fig:divider}
\end{figure}

\noindent
De overdracht van deze spanningsdeler is te vinden in \autoref{eq:dividerTransfer}.
\begin{equation}\label{eq:dividerTransfer}
    H(s) = \frac{U_{ref}(s)}{U_{dd}(s)} = \frac{R_2}{R_1 + R_2 + R_2Cs}
\end{equation}

\noindent
Het vermogen dat de spanningsdeler dissipeert, kan met \autoref{eq:dividerPower} berekend worden.
\begin{equation}\label{eq:dividerPower}
    P(s) = U_{dd}(s)^2\frac{1+R_2Cs}{R_1 + R_2 + R_1R_2Cs}
\end{equation}
Met een constante DC ingangsspanning kan dit vereenvoudigd worden naar \autoref{eq:dividerPowerSimple}.
\begin{equation}\label{eq:dividerPowerSimple}
    P = \frac{U_{dd}^2}{R_1 + R_2}
\end{equation}

\noindent
Om de ruis van deze schakeling te berekenen moet een aantal stappen genomen worden. Aangezien de ingangsbron $U_{dd}$ een spanningsbron is, kan deze als kortsluiting genomen worden. Op deze manier kunnen de twee weerstanden parallel genomen worden, en verandert de schakeling in een simpel RC filter. In \autoref{fig:dividerNoise} is deze omgebouwde schakeling te zien.

\begin{figure}[ht]
    \centering
    \def\svgwidth{0.35\textwidth}
    \begin{tikzpicture}
    \pgfplotsset{width=\textwidth}
    \newcommand\BOLZ{1.380649e-23}
    \newcommand\TEMP{300}
    \newcommand\OMEGAC{15*2*pi}
    \newcommand\RESRAT{(7/11)}
    \newcommand\REQU{(1/(1/x + \RESRAT/x))}
    \newcommand\CAP{0.000001}

    \pgfplotsset{set layers}
    \begin{axis}[
        xmode=log,
        ymode=log,
        xlabel={$R_1 [\si{\ohm}]$},
        ylabel={$u_{n,out} [\si{\volt}]$},
        xmin=1e3, xmax=1e7,
        grid=major
    ]

    \addplot [
        red,
        domain=1e3:1e7,
        samples=201
    ]
    {sqrt((4 * \BOLZ * \TEMP / \CAP) * rad(atan(\REQU * \CAP * \OMEGAC)))};
    \end{axis}
\end{tikzpicture}
    \caption{De omgebouwde schakeling om ruis mee te berekenen.}
    \label{fig:dividerNoise}
\end{figure}

\noindent
Voor de spectrale spanningsruisdichtheid aan de uitgang $U_{ref}$ kan \autoref{eq:dividerNoiseLaplace} worden opgesteld.
\begin{equation}\label{eq:dividerNoiseLaplace}
    S_{n,u_{ref}} = 4kTR_e\left(\frac{1}{1 + R_eCs}\right)^2
\end{equation}
Wanneer de absolute waarde van de ruis wordt genomen, kan deze over de bandbreedte geïntegreerd worden. Dit resulteert in \autoref{eq:dividerNoiseInt}.
\begin{equation}\label{eq:dividerNoiseInt}
    u_{n,ref}^2 = \int_{\omega_l}^{\omega_h} 4kTR_e\left(\frac{1}{\sqrt{1 + (R_eC\omega)^2}}\right)^2 d\omega
\end{equation}
Het integraal van deze formule komt uit op \autoref{eq:dividerNoiseIntegrated}.
\begin{equation}\label{eq:dividerNoiseIntegrated}
    u_{n,ref}^2 = \frac{4kT}{C}\left[\arctan(R_eC\omega_h) - \arctan(R_eC\omega_l)\right]
\end{equation}

Zolang de spanning stabiel is hoeft de referentie geen exact gedefinieerde spanning te hebben. Dit is omdat de ADC die de waarde van de sensor gaat uitlezen deze referentiespanning ook als referentie zal gebruiken. De waarde moet echter ergens rond de 1.1V zitten, om de stroombron te laten werken.
In \autoref{sec:currentSource} is hier meer over te lezen.
Aangezien de ingangsspanning 1.8V is, zal de DC overdracht $1.1 / 1.8 = \frac{11}{18}$ zijn. Hieruit komt de weerstandsverhouding in \autoref{eq:dividerResistors}.
\begin{equation}\label{eq:dividerResistors}
    \frac{R_1}{R_2} = \frac{7}{11}
\end{equation}
Een hogere $R_1$ zorgt voor een lager vermogensverbruik, maar ook een hogere ruis. Dit is te zien in \autoref{fig:dividerPlots}.

\begin{figure}
    \centering
    \begin{subfigure}[b]{0.45\textwidth}
        \centering
        \begin{tikzpicture}
    \pgfplotsset{width=\textwidth}
    \newcommand\BOLZ{1.380649e-23}
    \newcommand\TEMP{300}
    \newcommand\OMEGAC{15*2*pi}
    \newcommand\RESRAT{(7/11)}
    \newcommand\REQU{(1/(1/x + \RESRAT/x))}
    \newcommand\CAP{0.000001}

    \pgfplotsset{set layers}
    \begin{axis}[
        xmode=log,
        ymode=log,
        xlabel={$R_1 [\si{\ohm}]$},
        ylabel={$u_{n,out} [\si{\volt}]$},
        xmin=1e3, xmax=1e7,
        grid=major
    ]

    \addplot [
        red,
        domain=1e3:1e7,
        samples=201
    ]
    {sqrt((4 * \BOLZ * \TEMP / \CAP) * rad(atan(\REQU * \CAP * \OMEGAC)))};
    \end{axis}
\end{tikzpicture}
        \caption{Spanningsruis}
        \label{fig:dividerNoisePlot}
    \end{subfigure}
    \hfill
    \begin{subfigure}[b]{0.45\textwidth}
        \centering
        \begin{tikzpicture}
    \pgfplotsset{width=\textwidth}
    \newcommand\OMEGAC{10*2*pi}
    \newcommand\RESRAT{(7/11)}

    \begin{axis}[
        xmode=log,
        ymode=log,
        xlabel={$R_1 [\si{\ohm}]$},
        ylabel={$P [\si{\watt}]$},
        xmin=1e3, xmax=2e6,
        grid=major
    ]

    \addplot [
        blue,
        domain=1e3:2e6,
        samples=201
    ]
    {1.8 / (x + x/\RESRAT)};
    \end{axis}
\end{tikzpicture}
        \caption{Vermogensverbruik}
        \label{fig:dividerPower}
    \end{subfigure}
    \caption{De ruis en het vermogensverbruik van de spanningsdeler, ten opzichte van de gekozen weerstandswaarde $R_1$.}
    \label{fig:dividerPlots}
\end{figure}
Op $R_1 = 1\si{\mega\ohm}$ en $C = 10\si{\micro\farad}$ is de spanningsruis $u_{n,ref} = 51\si{\nano\volt}$ en het vermogensverbruik $P = 1.3\si{\micro\watt}$. De andere weerstandswaarde is dan $R_2 \approx 1.6 \si{\mega\ohm}$. Deze waardes vallen binnen de specificaties.

[WELKE SPECS??]
    \caption{De spanningsreferentie schakeling, ook te vinden in \cref{fig:divider}.}
    \label{fig:dividerForContext}
\end{figure}

% \paragraph{Simulatie}
Om te verifiëren dat de spanningsreferentie goed werkt, is er een aantal simulaties uitgevoerd.
In \cref{fig:referenceSimFreq} is het resultaat van een AC simulatie te zien. Hier is $H(f)$ de overdracht van de voeding naar de uitgang van de spanningsdeler. De overdracht begint op \qty{-16.4}{\decibel}, aangezien dat de overdracht van de schakeling is op \qty{0}{\hertz}.

\begin{figure}[!htbp]
    \centering
    \pgfplotsset{width=0.7\textwidth}
    \begin{tikzpicture}
    \tikzset{
        small dot/.style={fill=black,circle,scale=0.3},
    }

    \begin{axis}[
        xmode=log,
        xlabel={$f$ [\unit{\hertz}]},
        ylabel={$H(f)$ [\unit{\decibel}]},
        grid=major
    ]
        \addplot [
            mark=none,
            line width=0.5mm
        ] table[x=freq,y=out] {sim/referenceSimFreq.dat};
        % \addplot [
        %     red,
        %     mark=*
        % ] coordinates {(0.18714337, -19.391)};
        \node [small dot,pin=0:{kantelpunt}] at (0.18714337, -19.391) {};
    \end{axis}
\end{tikzpicture}
    \caption{Het resultaat van een AC simulatie van de spanningsreferentie.}
    \label{fig:referenceSimFreq}
\end{figure}

In \cref{fig:referenceSimTrans} is het resultaat van een transient simulatie te zien. Deze simulatie laat zien dat het 5 seconden duurt voordat de spanningsreferentie de gewenste spanning van \qty{0.15}{\volt} bereikt.
\begin{figure}[!htbp]
    \centering
    \pgfplotsset{width=0.7\textwidth}
    \begin{tikzpicture}

    \begin{axis}[
        xlabel={$t$ [\unit{\second}]},
        ylabel={$H(f)$ [\unit{\decibel}]},
        grid=major
    ]
    \addplot [
        mark=none,
        line width=0.5mm
    ] table[x=time,y=out] {sim/referenceSimTrans.dat};
    \end{axis}
\end{tikzpicture}


    \caption{Het resultaat van een transient simulatie van de spanningsreferentie.}
    \label{fig:referenceSimTrans}
    % \label{fig:referenceSimCis}
\end{figure}

In \cref{fig:referenceSimNoise} is het resultaat van een ruissimulatie te zien. Hier mag tycho iets over vertellen veel plezier tycho!%TODO: dat
\begin{figure}[!htbp]
    \centering
    \pgfplotsset{width=0.7\textwidth}
    \begin{tikzpicture}

    \begin{axis}[
        xmode=log,
        xlabel={$f$ [\unit{\hertz}]},
        ylabel={$\sqrt{S_{u,n}} \,\,\,\, \left[\unit{\nano\volt}/\sqrt{\unit{\hertz}}\right]$},
        grid=major,
        height=6cm
    ]
    \addplot [
        mark=none,
        line width=0.5mm,
        y filter/.code={\pgfmathparse{#1*1e9}\pgfmathresult}
    ] table[x=freq,y=noise] {sim/referenceSimNoise.dat};
    \end{axis}
\end{tikzpicture}


    \caption{Het resultaat van een ruissimulatie van de spanningsreferentie.}
    \label{fig:referenceSimNoise}
\end{figure}


\subsubsection{Nullor implementatie}
Voor de nullor die gebruikt wordt om de ISFET uit te lezen in \cref{sec:ISFETLees} moet een implementatie gekozen worden. De uitleesschakeling mag volgens de specificaties maximaal \qty{200}{\micro\watt}  gebruiken. De constante stroom die door de weerstand en ISFET in \cref{fig:measureResistor} heen loopt, zorgt voor een constant vermogensverbruik van \qty{165}{\micro\watt}. Hierdoor mag de nullor implementatie maximaal \qty{35}{\micro\watt} gebruiken. Het maximale dynamische vermogen dat deze nullor implementatie aan de uitgang zal moeten kunnen leveren, is gelijk aan het maximale vermogen dat het filter kan dissiperen. Er blijft dan afgerond nog \qty{34}{\micro\watt} aan statisch vermogen over. Dit resulteert in een maximale quiescent stroom van \qty{10.3}{\micro\ampere}.

De uitleesschakeling moet een minimale SNR hebben van 40 dB. De maximale ruisspanning en stroom die de nullor mag genereren aan de ingang zijn te berekenen met \cref{eq:nullorImplementNoise}. Deze vergelijking is afgeleid uit \Cref{eq:measureNoiseFull}.
\begin{equation} \label{eq:nullorImplementNoise}
    S_{u_{{n,n}}} + S_{i_{{n,in}}}\left(Z_{fet} // R\right)^2 = \frac{S_{u_{{n,out}}}}{H^2(\ph)} - S_{u_{{n,ref}}}
\end{equation}

De LTC2064 opamp heeft (buiten shutdown) een quiescent stroom van \qty{2.5}{\micro\ampere}, wat op \qty{3.3}{\volt} resulteert in een vermogen van \qty{8.25}{\micro\watt}. Daarbij heeft deze een spectrale ruisdichtheid van \qty{12}{\femto\ampere\hertz^{-0.5}} en $\qty{220}{\nano\volt\hertz^{-0.5}}$\cite{LTC2064}. Dit zit volgens \cref{eq:nullorImplementNoise} ver onder het maximum. Daarnaast is zowel de ingangsafwijking als de 1/f ruis van deze opamp erg laag. Hierdoor is deze opamp gekozen voor het ontwerp.


\begin{figure}[!htbp]
    \centering
    \pgfplotsset{width=0.7\textwidth}
    \begin{tikzpicture}
    \tikzset{
        small dot/.style={fill=black,circle,scale=0.4},
    }

    \begin{axis}[
        ylabel={$U_{out}$ [\unit{\volt}]},
        xlabel={$U_{t}$ [\unit{\milli\volt}] - \qty{1.8}{\volt}},
        grid=major,
        height=6cm,
    ]
        \addplot [
            mark=none,
            line width=0.5mm,
        ] table[x=thresh,y=out] {sim/readoutSimTrans.dat};
        % \node [small dot,pin=0:{Voeding wordt geactiveerd}] at (1,0) {};
    \end{axis}
\end{tikzpicture}


    \caption{Het resultaat van meerdere transient simulaties op de uitleesschakeling. De uitgangsspanning van de schakeling is geplot op basis van de ingestelde drempelspanning van de FET.}
    \label{fig:readoutSimTrans}
\end{figure}
\subsection{ADC} \label{sec:selectingADCandReqs}
De specificaties van de ADC die nodig is voor dit project kunnen berekend worden op basis van de specificaties voor het ADC blok. Deze specificaties staan in \cref{tab:systemSpecADC} samengevat.
\begin{table}[!htb]
    \centering
    \begin{tabular}{l|c|l}
        Symbol      & Waarde & Eenheid\\\hline
        $SNR_{in}$  & 37        & dB\\
        NF          & 3         & dB\\
        $u_{in}$    & 2.5       & mV\\
    \end{tabular}
    \caption{De eisen voor het omzetten van het analoge signaal naar een digitaal signaal.}
    \label{tab:systemSpecADC}
\end{table}

Door gebruik te maken van de formules uit \cref{sec:ADC:numBits,sec:ADC:sampleFreq} kunnen de specificaties voor de benodigde ADC berekend worden. De resultaten hiervan zijn in  \cref{tab:specADC} geplaatst.

Bij het berekenen van deze specificaties is er van uit gegaan dat het totale noise figure 1 op 1 is verdeeld tussen de resolutie en de bemonsteringsfrequentie.
\begin{table}[!htb]
    \centering
    \begin{tabular}{l|c|l}
        Symbol      & Waarde    & Eenheid\\\hline
        n           & 14        & bits\\
        $f_{s,min}$ & 45        & Hz\\
        $f_{s,max}$ & 515       & kHz\\
    \end{tabular}
    \caption{De eisen voor het omzetten van het analoge signaal naar een digitaal signaal.}
    \label{tab:specADC}
\end{table}

% Het blijkt het geval te zijn dat de ingebouwde ADC die in de \mcu  zit, voldoet aan de specificaties die in \cref{tab:specADC,tab:systemSpecADC} staan \cite{nrf52810}. Dit zorgt er voor dat het niet nodig is om een externe ADC te gebruiken.
\subsection{Filter}
Het laagdoorlaatfilter dat voor de ADC zit, heeft als functie om aliasing te voorkomen. Aliasing treedt op vanaf de helft van de sample frequentie van de ADC. In \cref{sec:selectingADCandReqs} is uitgerekend dat de minimale sample frequentie 45 Hz is. Hieruit volgt dat er vanaf 22.5 Hz aliasing optreedt. In \cref{sec:energyBudgets} is aangegeven dat de te verwachten ingangssignaal ruis verhouding 40 dB is. Om de aliasing tot een acceptabel niveau te onderdrukken moeten signalen vanaf 22.5 Hz met minimaal 40 dB worden gedempt. Als laatste mag de implementatie van het filter niet meer dan 200 $\si{\micro\watt}$ verbruiken. In \cref{tab:specsAAfilter} staan alle specificaties voor het anti aliasing filter samengevat.
\begin{table}[ht]
    \centering
    \begin{tabular}{c|c|c}
        Specificatie & Waarde & Eenheid \\\hline
        $f_h$       & 10   & $[\si{\hertz}]$ \\
        $D_f$       & 3    & $[\mathrm{dB}]$ \\
        $f_d$       & 22.5 & $[\si{\hertz}]$ \\
        $D_D$       & 40   & $[\mathrm{dB}]$ \\
        $NF$        & 3    & $[\mathrm{dB}]$ \\
        $u_{in,min}$& 1.94 & $[\si{\milli\volt}]$  \\
        $\overline{u_{n,in}}$ & 378 & $[\si{\pico\volt^2}]$\\
        $P$         & 200  & $\si{\micro\watt}$ \\
    \end{tabular}
    \caption{De specificaties voor het anti aliasing laagdoorlaatfilter.}
    \label{tab:specsAAfilter}
\end{table}

\subsubsection{Filter orde berekenen}
Met de specificaties in \cref{tab:specsAAfilter} is met \cref{eq:minOrderOfAAfilter} uit te rekenen dat er minimaal een zesde orde antialiasing filter nodig is. In \cref{tab:AA3dBspecs} staan voor filter ordes zes tot en met tien de acceptabele verschuiving van de kantelfrequentie.
\begin{table}[!htbp]
    \centering
    \begin{tabular}{c|c|c|c}
        orde & $f_c\,[\si{\hertz}]$ & $\epsilon_{f_c}\,[\si{\hertz}]$ & $\epsilon_{f_c}\,[\%]$ \\\hline
        6    & 10.53 & 0.53 & 5.03  \\
        7    & 11.12 & 1.12 & 10.07 \\
        8    & 11.61 & 1.60 & 13.82 \\
        9    & 12.01 & 2.01 & 16.71 \\
        10   & 12.35 & 2.35 & 19.00 \\
    \end{tabular}
    \caption{De specificaties voor de kantelfrequentie voor zesde tot en met tiende orde filters.}
    \label{tab:AA3dBspecs}
\end{table}
\Cref{tab:AA3dBspecs} laat zien dat het mogelijk een goede keuze kan zijn om een hogere orde dan de minimale zesde orde te kiezen. Een rede hiervoor kan zijn dat er rekening gehouden wordt met de onnauwkeurigheid van componenten. Het is te zien dat het ophogen van de filter orde invers proportioneel is aan de toename van de toegestane afwijking van de kantelfrequentie\footnote{Zie \cref{sec:DetermineAAorder}.}.

\subsubsection{Filter implementeren} \label{sec:filterFout}
Tijdens het ontwerpen van het pH meetsysteem is er foutief aangenomen dat een eerste orde filter zou voldoen. Dit is de rede dat in de eerste versie van dit prototype een eerste orde laagdoorlaatfilter is gebruikt. Er is wel rekening gehouden met de ruis en vermogenseisen.

Het eerste orde laagdoorlaatfilter bestaat uit een weerstand en een condensator die beiden van een waarde voorzien moeten worden. De minimale condensator waarde kan berekend worden op basis van de ruis van het voorgaande systeem met \cref{eq:filterCapMin}.
% Met de ruis van het voorgaande systeem kan de minimale condensatorwaarde berekend worden door middel van \cref{eq:filterCapMin}.
% Deze komt uit op ongeveer \qty{60}{\pico\farad}.
Hieruit volgt dat de condensator minimaal \qty{11}{\pico\farad} moet zijn. De benodigde weerstandswaarde die uit deze condensatorwaarde volgt is \qty{1.3}{\giga\ohm}. Een weerstandswaarde van \qtylist{1.3}{\giga\ohm} is niet praktisch, omdat een kleine parallele capaciteit de effectieve weerstandswaarde snel doet dalen. Door nu met de vermogenseis een maximale capaciteit uit te rekenen kan er gekeken worden of er een lagere weerstandswaarde gekozen kan worden. \Cref{eq:filterPower} kan herschreven worden om op basis van vermogen een maximale capaciteit te berekenen, zie \cref{eq:maxCinRCforMaxP}. Met \cref{eq:maxCinRCforMaxP} blijkt dat de maximale capaciteit \qty{609}{\nano\farad} is.
\begin{equation}\label{eq:maxCinRCforMaxP}
    C_{max}=\frac{P_{max}}{u_i^2\omega_c}
\end{equation}
% Hiermee moet de weerstandswaarde echter \qty{270}{\mega\ohm} zijn, wat niet praktisch is. Met een condensator van \qty{10}{\nano\farad} kunnen de waardes in \cref{tab:filterValues} berekend worden. Deze waardes vallen binnen de specificaties.

Voor de implementatie van het RC laagdoorlaatfilter is gekozen voor een capaciteit van \qty{100}{\nano\farad}. In \cref{tab:filterValues} staan de daaruit volgende eigenschappen van het RC filter.

\begin{table}[!htbp]
    \centering
    \begin{tabular}{l|l|l}
        Symbool & Waarde & Eenheid \\
        \hline
        $C$         & 100    & $\si{\nano\farad}$\\
        $R$         & 147   & $\si{\kilo\ohm}$  \\
        $f_c$       & 10.83  & $\si{\hertz}$     \\
        $P$         & 32.84   & $\si{\micro\watt}$ \\
        $\sqrt{\overline{u_{n,out}}}$ & 19.44   & $\si{\micro\volt}$ \\
        $\overline{u_{n,out}}$ & 378 & $\si{\pico\volt^2}$\\
        NF          & 0  & $\si{\decibel}$   \\
    \end{tabular}
    \caption{De gekozen waardes van het filter, en de resulterende vermogens- en ruiseigenschappen.}
    \label{tab:filterValues}
\end{table}

Nu de waardes van het filter bekend zijn moet gecontroleerd worden of de ADC nog steeds aan de specificaties voldoet. Het is nodig om dit te controleren omdat het filter passief is geïmplementeerd. Door de passieve implementatie ontstaat er een spanningsdeler. Deze spanningsdeler wordt gevormd door de ingangsimpedantie van de ADC en de weerstandswaarde van het anti-aliasing filter. Uit \cref{eq:calcMinNumberADCbits} blijkt echter dat de eisen voor de ADC niet veranderen en dat de interne ADC van de \mcu nog steeds gebruikt kan worden.


\subsubsection{Simulatie}
Het filter is op meerdere manieren gesimuleerd. \Cref{fig:filterSimFreq} toont een AC simulatie van het filter. Hierin is te zien dat het kantelpunt inderdaad op ongeveer \qty{10.8}{\hertz} ligt

\begin{figure}[!htbp]
    \centering
    \pgfplotsset{width=0.7\textwidth}
    \begin{tikzpicture}
    \tikzset{
        small dot/.style={fill=black,circle,scale=0.3},
    }
    \begin{axis}[
        xmode=log,
        xlabel={$f$ [\unit{\hertz}]},
        ylabel={$\left|H\left(f\right)\right| \,\,\,\, \left[\unit{\decibel}\right]$},
        grid=major,
        height=6cm
    ]
    \addplot [
        mark=none,
        line width=0.5mm,
    ] table[x=freq,y=out] {sim/filterSimFreq.dat};
    \node [small dot,pin={[pin edge={line width=0.3mm,black}]0:kantelpunt}] at (10.80152, -3) {};
    \end{axis}
\end{tikzpicture}




    \caption{Het resultaat van een AC simulatie van het filter.}
    \label{fig:filterSimFreq}
\end{figure}

\begin{figure}[!htbp]
    \centering
    \pgfplotsset{width=0.7\textwidth}
    \begin{tikzpicture}
    \begin{axis}[
        xmode=log,
        xlabel={$f$ [\unit{\hertz}]},
        ylabel={$\sqrt{S_{u,n}} \,\,\,\, \left[\unit{\nano\volt}/\sqrt{\unit{\hertz}}\right]$},
        grid=major,
        height=6cm
    ]
    \addplot [
        mark=none,
        line width=0.5mm,
    ] table[x=freq,y expr=\thisrow{noise}] {sim/filterSimNoise.dat};
    \end{axis}
\end{tikzpicture}


    \caption{Het resultaat van een ruis simulatie van het filter.}
    \label{fig:filterSimNoise}
\end{figure}
\subsection{Temperatuursensor}

Er is nog niks ontworpen voor het meten van de temperatuur van de te meten oplossing. De \mcu heeft echter wel een ingebouwde temperatuursensor.
Deze kan gebruikt worden om te compenseren voor veranderingen in temperatuur, zolang de temperatuur van de \mcu niet te veel afwijkt van de temperatuur van de te meten oplossing.



\subsection{Het RF systeem}
In \cref{sec:ontwerp:RF} is ingegaan op wat het minimum zendvermogen is dat nodig is om een BER van $1\times 10^{-5}$ te behalen wanneer de \si{\pH} sensormodule \qty{10}{\meter} van het basis station is verwijderd.
\begin{table}[!htb]
    \centering
    \begin{tabular}{l|c|c}
        Specificatie    & Waarde    & Eenheid \\\hline
        $P_{zend,min}$  & 7.1       & \si{\micro\watt}  \\
        $B$             & 2         & \si{\mega\hertz}  \\
        Modulatie       & GSK/BLE   &                   \\
    \end{tabular}
    \caption{De specificaties waaraan de RF zend implementatie moet voldoen.}
    \label{tab:specRfsending}
\end{table}
% In \cref{sec:ontwerp:RF} is ingegaan op het minimum zendvermogen dat nodig is. Dit minimum vermogen is \qty{7.1}{\micro\watt}. Het is belangrijk om op te merken dat dit enkel het minimum vermogen is dat in het RF signaal zit. Het genereren van dit RF signaal kan mogelijk meer energie kosten. Bij het kiezen van de microcontroller is de \mcu gekozen, onder andere omdat er een \qty{2.4}{\giga\hertz} transceiver in zit. Uit de datasheet van de \mcu is te halen dat deze transceiver \qty{4.6}{\milli\ampere} aan stroom trekt, indien er draadloos wordt gezonden met \qty{0}{\deci\belmilliwatt} en een datasnelheid van \qty{1}{\mega\hertz} \cite{nrf52810}. Door \cref{eq:calcPperPacket,eq:calcRfAvaragePower} te gebruiken is een gemiddeld RF vermogen van \qty{45}{\micro\watt} te berekenen. \qty{45}{\micro\watt} zit onder het energie budget dat beschikbaar is voor het draadloos zenden.

Er zijn meerdere manieren om een RF zender te implementeren.
\begin{itemize}
    \item Zelf met discrete componenten een RF zender implementeren,
    \item Een losse RF zend IC gebruiken,
    \item Een MCU gebruiken met een geïntegreerde RF transceiver.
\end{itemize}
Omdat de complexiteit van een RF zender op basis van discrete componenten te groot is zal dit niet gedaan worden. De overblijvende twee opties zijn een losse RF zend IC en een MCU met geïntegreerde RF transceiver. In eerste instantie zal er gekeken worden of er een MCU gekozen kan worden die een RF transceiver heeft die aan de eisen uit \cref{tab:specRfsending} voldoet.

Bij de implementatie van het RF zenden, is het belangrijk om de RF uitgangsimpedantie van de zender te matchen met de antenne impedantie. Dit is belangrijk om een zo klein mogelijke hoeveelheid aan energie te verspillen aan reflecties \cite{FundamentalsofAppliedElectromagnetics}. In de meeste datasheets van zenders staat beschreven hoe de RF uitgang gematcht kan worden aan \qty{50}{\ohm}.
\subsection{Microcontroller}
De microcontroller moet de digitale signaalverwerking uitvoeren maar moet ook beschikken over een ADC en rf transceiver. In \cref{tab:specMCU} staan alle specificaties waaraan de microcontroller moet voldoen onder elkaar.
\begin{table}[!htbp]
    \centering
    \begin{tabular}{l|c|c}
        Specificatie    & Waarde    & Eenheid \\\hline
        $P_{zend,min}$  & 7.1       & \si{\micro\watt}  \\
        $B$             & 2         & \si{\mega\hertz}  \\
        Modulatie       & GSK/BLE   &                   \\\hline
        $f_s$           & 45        & \si{\hertz}       \\
        $n$             & 14        & bits              \\
        ADC ingangen    & 2         &                   \\\hline
        $P_{max}$       & 6         & \si{\milli\watt}
    \end{tabular}
    \caption{De specificaties waaraan de microcontroller van het \si{\pH} meetsysteem moet voldoen.}
    \label{tab:specMCU}
\end{table}
Het maximale vermogen dat de microcontroller mag verbruiken is samengesteld uit de maximale vermogens van de ADC en de rf zender. In \cref{sec:energyBudgets} is te zien hoe de \qty{6}{\milli\watt} over de ADC en de rf zender is verdeeld.

% Het digitale gedeelte van de implementatie kan opgedeeld worden in 3 onderdelen: de ADC, de digitale signaalverwerking (\cref{fig:digitaleBewerkingsFunctie}) en het draadloos versturen van data. Het is mogelijk om elk van deze onderdelen met aparte componenten te implementeren. Er zijn echter ook componenten beschikbaar die al over elk van deze functionaliteiten beschikken.

Een microcontroller die voldoet aan de eisen van \cref{tab:specMCU} is de \mcu. De \mcu beschikt over 8 14 bit ADC kanalen\footnote{De ADC kanalen zijn alleen 14 bit met oversampling \cite{nrf52810}.} die op maximaal \qty{200}{\kilo\hertz} kunnen samplen. Daarnaast heeft de \mcu ook een ingebouwde 2.4GHz Bluetooth transceiver \cite{nrf52810}.

Wanneer de ADC van de \mcu op \qty{16}{\hertz} sampled trekt de ADC ongeveer \qty{1.1}{\milli\ampere}. Deze samplefrequentie ligt aanzienlijk hoger dan de \qty{45}{\hertz} waarop dit systeem moet gaan samplen. Hierdoor zal het vermogen dat verbruikt wordt wanneer de ADC actief is ongeveer rond de \qty{3.1}{\micro\ampere} liggen.

% Een voorbeeld van een dergelijk component is de nRF52810. Deze microcontroller beschikt over meerdere 14 bit ADC kanalen\footnote{De ADC kanalen zijn alleen 14 bit met oversampling.} en een ingebouwde 2.4GHz Bluetooth transceiver.
De \mcu trekt \qty{4.6}{\milli\ampere} aan stroom, indien er draadloos wordt gezonden met \qty{0}{\deci\belmilliwatt} en een datasnelheid van \qty{1}{\mega\hertz} \cite{nrf52810}. Door \cref{eq:calcPperPacket,eq:calcRfAvaragePower} te gebruiken is een gemiddeld rf vermogen van \qty{45}{\micro\watt} te berekenen.

Ook heeft de microcontroller een slaapstand die onderbroken kan worden door een ingebouwde RTC, wat nuttig is voor het periodiek samplen en versturen van pH waardes. In deze slaapstand wordt er zo'n \qty{1.5}{\micro\ampere} gebruikt. Met een voedingsspanning van \qty{3.3}{\volt} komt dit uit op een vermogensverbruik van \qty{4.95}{\micro\watt}. Daarbij heeft de microcontroller ook de mogelijkheid om onderdelen van het geheugen uit te zetten, wat tot meer energiebesparing kan leiden \cite{nrf52810}.

Aangezien de \mcu niet constant hoeft te zenden zal er niet constant met de rf transceiver worden gezonden. Aangezien de rf transceiver het meeste vermogen trekt lijkt het daarmee er op dat de \mcu gebruikt kan worden als microcontroller voor het \si{\pH} meetsysteem.
\subsection{Voeding}
De voeding voor het \si{\pH} meetsysteem moet een aantal dingen kunnen doen. Deze verschillende eisen voor de voeding worden hieronder kort benoemd.
\begin{itemize}
    \item De analoge ISFET uitleesschakeling voeden op een vaste spanning (rond \qty{3.3}{\volt})
    \item De digitale schakeling voeden \qty{3.3}{\volt}
    \item De batterij opladen via een usb-c kabel
    \item De batterij opladen via een piëzo element
    \item De batterij beschermen
    \item Minder dan \qty{2}{\milli\watt} verbruiken
\end{itemize}
In deze paragraaf zal in gegaan worden op de implementatie van de voeding.

\subsubsection{Batterij en bescherming}

Voor de gekozen LiPO batterij technologie is er bescherming nodig. De celspanning mag niet boven de \qty{4.2}{\volt} of onder de \qty{2.7}{\volt} komen. Dit kan op meerdere manieren gedaan worden. In de implementatie van de sensor module is er gekozen voor een LTC4071 batterij beschermings IC. Wanneer de spanning van de batterij boven de \qty{4.2}{\volt} komt, gebruikt de LTC4071 een \qty{50}{\milli\ampere} shunt om de ingang stroom naar hitte om te zetten. Wanneer de batterijspanning onder de \qty{2.7}{\volt} komt, zet de IC de uitgang uit, om te voorkomen dat de batterijspanning verder daalt.

\subsubsection{Voeding}
Voor de voeding is er gekozen voor een LTC3330 van Analog Devices. Dit is een zogenaamde `power management integrated circuit' (PMIC). Deze PMIC voldoet aan de specs van \cref{tab:systemSpecs}, en heeft een aantal nuttige eigenschappen:
\begin{itemize}
    \item Ingebouwde ideale diodes voor AC-DC omzetting
    \item Een buck-boost converter
    \item Een low dropout regulator (LDO)
    \item Mogelijkheid om de LDO uit te zetten
    \item Lage 750 nA quiescent current
\end{itemize}

De gekozen PMIC is een IC die is ontworpen voor energy harvesting en low power modules. De uitgang van de energy harvesting wordt als eerste gelijkgericht door een ideale diode gelijkrichter. Dit zorgt voor minimaal energie verlies. Hierna bepaalt de LTC3330 of de rest van het systeem de stroom nodig heeft of dat de energie opgeslagen moet worden in de accu. De PMIC heeft 2 spanning omzet methodes ingebouwd. Een buck-boost converter en een LDO die aan en uit kan. De LDO wordt gevoed door de buck-boost.
\subsection{Printplaten}
De schakelingen voor de voeding en voor het uitlezen van de ISFET zijn opgedeeld in twee verschillende printplaten. Dit is gedaan zodat beide schakelingen apart van elkaar getest kunnen worden. De uitlees PCB is te zien in \cref{fig:sensorPCB}. Deze PCB bevat de ISFET uitleesschakeling en de \mcu, die de gemeten pH waarde naar het basisstation opstuurt. De voeding printplaat is te zien in \cref{fig:powerPCB}. Deze PCB regelt de energy harvesting, samen met het veilig opladen en het ontladen van de batterij. De twee printplaten zijn met elkaar verbonden door middel van pin headers. Beide schakelingen zijn te zien \cref{fig:PCBs}.

Beide PCB's hebben op elk belangrijk signaal een testpunt. Op deze manier kan er eenvoudig gemeten worden.


\begin{figure}[!htbp]
    \centering
    \begin{subfigure}[b]{0.48\textwidth}
        \centering
        \includegraphics[width=\textwidth]{sensorBord}
        \caption{De ISFET uitlezende PCB.}
        \label{fig:sensorPCB}
    \end{subfigure}
    \hfill
    \begin{subfigure}[b]{0.60\textwidth}
        \centering
        \includegraphics[width=\textwidth]{powerandharvest}
        \caption{De voeding en harvesting PCB.}
        \label{fig:powerPCB}
    \end{subfigure}
    \caption{De gemaakte PCB's.}
    \label{fig:PCBs}
\end{figure}




