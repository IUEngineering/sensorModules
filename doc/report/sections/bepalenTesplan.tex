\section{Testen}
Om de sensormodule te testen zijn een aantal materialen nodig. Deze materialen zijn gespecificeerd in \cref{tab:testMaterialen}.
\begin{table}[ht]
    \centering
    \begin{tabular}{l|l|l}
        Apparaat         & Serienummer & Beschrijving \\
        \hline
        MSREF1           & 23/RS03     & Referentie elektrode       \\
        MSFET 3330-2     & 23/205      & ISFET pH sensor            \\
        TBS 1072B-EDU    & C030387     & Oscilloscoop               \\
        ISFETLezer       & 1           & ISFET uitlees schakeling   \\
        Harvester        & 1           & BMS en energy harvester    \\    
        \hline
    \end{tabular}
    \caption{Materialen die zijn gebruikt voor de tests}
    \label{tab:testMaterialen}
\end{table}

Om het ISFET uitleessysteem te testen moet eerst alles aangesloten worden. De ISFET uitlees PCB wordt verbonden met de energie PCB, zodat de uitleesschakeling een voeding heeft.
Vervolgens wordt er een probe van de oscilloscoop verbonden met de uitgang van de opamp. De ISFET wordt samen met de referentieelektrode in een bufferoplossing gestopt. Hiermee kan gemeten worden of de uitgang van de schakeling stabiel is.

\begin{figure}[ht]
    \centering
    \def\svgwidth{0.5\textwidth}
    \input{img/ISFETCircuitBestTest.pdf_tex}
    \caption{De locatie van de scope probes in de schakeling om te testen of de uitgang stabiel is.}
    \label{fig:test ISFET circuit best}
\end{figure}


\subsection{Resultaten}

De eerste test resulteerde in een oscillerend signaal. De uitgang van de opamp gaf een signaal dat elke [tijd]ms pulseerde.
De reden hiervoor is hoogst waarschijnlijk dat de ISFET niet snel genoeg reageert op de spanningsverandering tussen de gate (de referentie elektrode) en de source. Wanneer de opamp uitgang laag is, waardoor de gate-source spanning op de ISFET 0V wordt, komt de drain-source spanning van de ISFET namelijk erg langzaam omhoog. Wanneer deze spanning boven de referentiespanning komt, gaat de uitgang van de opamp omhoog, waarna de drain-source spanning van de ISFET pas later erg snel daalt. Dit zorgt voor een oscillerende werking.
Er zijn hier een aantal mogelijke oplossingen voor.

Eén mogelijke oplossing is een integrator plaatsen tussen de uitgang van de opamp en de gate van de ISFET. Deze integrator is dan zo ingesteld dat de maximale helling van de uitgang minder stijl is dan de helling van de drain-source spanning van de ISFET bij een gate-source spanning van 0V. Zo kan de ISFET de opamp bijhouden, waardoor de negatieve feedback gaat werken.

Een andere mogelijke oplossing is een RC-filter te plaatsen tussen de drain van de ISFET en de niet-inverterende ingang van de opamp. Dit zorgt ervoor dat de opamp trager gaat werken, waardoor de ISFET de opamp bij kan houden.

Uit datasheets van de ISFET blijkt dat de gate-source spanning negatief moet zijn om de weerstand tussen de drain en de source naar oneindig te laten gaan. Een andere mogelijke oplossing is dus om ervoor te zorgen dat de uitgang van de opamp zo'n negatieve spanning kan bereiken.



\subsection{Conclusie 1}
