
pH is een maat die gebruikt wordt voor het meten van de zuurgraad van een waterige oplossing \cite{ck12-chemistry}. De pH-waarde van een stof kan berekend worden door middel van \cref{eq:ph}, waar $[H^+]$ de molaire concentratie van $H^+$ ionen in \unit{\mol/\liter} is \cite{ck12-chemistry}.
\begin{equation}\label{eq:ph}
    \ph = -\log\left[H^+\right]
\end{equation}
Een oplossing met een pH-waarde van 7 wordt \textit{neutraal} genoemd, aangezien dat de pH-waarde van water is. Een oplossing is \textit{zuur} wanneer de pH-waarde lager is dan 7 en \textit{basisch} wanneer de pH-waarde hoger is dan 7 \cite{ck12-chemistry}.

De pH-waarde van een oplossing kan op verschillende manieren gemeten worden. Een van deze manieren is door gebruik te maken van een ISFET.