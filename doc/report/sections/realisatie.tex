\section{Realisatie}


\subsection{Nullor implementatie}


\subsection{Spanningsreferentie}
Zoals besproken in \cref{sec:referenceVoltage} kunnen de weerstandswaardes van de spanningsreferentie erg hoog gekozen worden. Met een $R_1$ van $5.6\si{\mega\ohm}$ gebruikt de spanningsdeler $1.65\si{\micro\watt}$.

Volgens \cref{eq:dividerNoise} heeft de condensatorwaarde wel effect op de ruis. Met een condensator van $1\si{\micro\farad}$ produceert de spanningsreferentie $64.4\si{\nano\volt}$ aan ruis. Dit zorgt voor een signaal-ruis verhouding van $138\si{\decibel}$, wat meer dan genoeg is.

Deze gekozen waardes en de resulterende eigenschappen zijn te vinden in \cref{tab:divider}.

\begin{table}[ht]
\centering
\begin{tabular}{l|l|l}
    Symbool & Waarde & Eenheid \\
    \hline
    $R_1$       & 5.6  & $\si{\mega\ohm}$   \\
    $R_2$       & 1.0  & $\si{\mega\ohm}$   \\
    $C$         & 1.0  & $\si{\micro\farad}$\\
    $P$         & 1.65 & $\si{\micro\watt}$ \\
    $u_{n,out}$ & 64.4 & $\si{\nano\volt}$  \\
    SNR         & 138  & $\si{\decibel}$
\end{tabular}
\caption{De gekozen waardes van de spanningsdeler, met het resulterende vermogensverbruik en de ruiseigenschappen.}
\label{tab:divider}
\end{table}


\subsection{ADC}
De ADC die in de \mcu zit ingebouwd voldoet aan de specificaties in \cref{tab:specADC,tab:systemSpecADC}\cite{nrf52810}. Het is dus niet nodig om een externe ADC te gebruiken. 


\subsection{Filter}
Het laagdoorlaatfilter dat voor de ADC zit, heeft een weerstand en een condensator die van een waarde voorzien moeten worden.
Met de ruis van het voorgaande systeem kan de minimale condensatorwaarde berekend worden door middel van \cref{eq:filterCapMin}. Deze komt uit op ongeveer $60\si{\pico\farad}$. Hiermee moet de weerstandswaarde echter $270\si{\mega\ohm}$ zijn, wat niet praktisch is. Met een condensator van $10\si{\nano\farad}$ kunnen de waardes in \autoref{tab:filterValues} berekend worden. Deze waardes vallen binnen de specificaties.

\begin{table}[ht]
    \centering
    \begin{tabular}{l|l|l}
        Symbool & Waarde & Eenheid \\
        \hline
        $C$         & 82    & $\si{\nano\farad}$\\
        $R$         & 180   & $\si{\kilo\ohm}$  \\
        $f_c$       & 10.8  & $\si{\hertz}$     \\
        $P$         & 408   & $\si{\nano\watt}$ \\
        $u_{n,out}$ & 225   & $\si{\nano\volt}$ \\
        NF          & 0.23  & $\si{\decibel}$   \\
    \end{tabular}
    \caption{De gekozen waardes van het filter, en de resulterende vermogens- en ruiseigenschappen.}
    \label{tab:filterValues}
\end{table}


\subsection{Rf}

\subsection{Batterij en bescherming}

Voor de gekozen LiPO batterij technologie is er bescherming nodig. De batterij moet beschermt worden zodat de spanning niet boven de 4.2 V en niet onder de 2.7 V komt. Dit kan op meerdere manieren gedaan worden. In de implementatie van de sensor module is er gekozen voor een simple LTC4071 batterij bescherming IC. Als de spanning van de batterij boven de 4.2 V komt, gebruikt de LTC4071 een 50mA shunt om de ingang stroom naar hitte om te zetten. Wanneer de batterijspanning onder de 2.7 V komt, zet de IC de uitgang uit, om te voorkomen dat de batterijspanning lager wordt.

\subsection{Voeding}
Voor de voeding is er gekozen voor een LTC3330 van Analog Devices. Dit is een zogenaamde PMIC (Power Management Integrated Circuit). De LTC3330 PMIC heeft een aantal nuttige eigenschappen en voldoet aan de specs van \cref{tab:systemSpecs}:
\begin{itemize}
    \item Ingebouwde ideale diodes voor piezo AC-DC omzetting
    \item Een buck-boost converter
    \item Een low dropout regulator (LDO)
    \item Mogelijkheid om de LDO uit te zetten
    \item Lage 750 nA quiescent current
\end{itemize}

De gekozen PMIC is een IC die is ontworpen voor energy harvesting en low power modules. De stroom die uit de energy harvesting komt wordt als eerste gelijkgericht door een ideale diode gelijkrichter. Dit zorgt voor minimaal energie verlies. Daarna bepaalt de LTC3330 of de rest van het systeem de stroom nodig heeft of dat de energie opgeslagen moet worden in de accu. De PMIC heeft 2 spanning omzet methodes ingebouwd. Een buck-boost converter en een LDO die aan en uit kan. De LDO wordt gevoed door de buck-boost. 

\begin{figure}
    \centering

    \label{}
\end{figure}

% software 
% hardware
