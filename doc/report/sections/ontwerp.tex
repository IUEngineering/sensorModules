\section{Ontwerp}\label{sec:ontwerp}
Het systeem bestaat uit 2 hoofdonderdelen: de sensormodule en een basisstation. De sensormodule meet de pH waarde van een oplossing, en verzend deze naar het basisstation. Het basisstation ontvangt de informatie en slaat deze informatie op.
In \cref{fig:functional} is dit te zien in een systeemdiagram.

\begin{figure}[ht]
    \centering
    \includegraphics{toplevelDiagram}
    \caption[short]{Een diagram van het volledige systeem.}
    \label{fig:functional}
\end{figure}

Het sensormodule blok kan wederom opgedeeld worden in aparte blokken. Dit is te zien in \cref{fig:moduleDiagram}.

\begin{figure}[ht]
    \centering
    \includegraphics[width=0.75\textwidth]{moduleDiagram}
    \caption{Een systeemdiagram van de sensormodule.} 
    \label{fig:moduleDiagram}
\end{figure}

\subsection{Functionele decompositie}
De sensormodule kan opgedeeld worden in 2 aparte systemen: de voeding, die verder wordt besproken in \cref{sec:voeding}, en de signaalverwerking.

De signaalverwerking bestaat zelf ook weer uit 2 onderdelen: het analoge gedeelte en het digitale gedeelte. In \cref{fig:analogeBewerkingsFunctie} is een decompositie te zien van de signaalbewerkingsfuncties die toegepast worden in het analoge gedeelte. In de komende paragrafen wordt elk van deze blokken apart besproken.

\begin{figure}[ht]
    \centering
    \includegraphics[width=0.95\textwidth]{analogeBewerkingsFunctie}
    \caption{Het analoge gedeelte van de signaalbewerking.} 
    \label{fig:analogeBewerkingsFunctie}
\end{figure}


Het digitale gedeelte bestaat uit een aantal berekeningen. Deze berekeningen zijn nodig om de gemeten spanning om te rekenen naar een pH waarde. Hiervoor zijn een aantal kalibratiewaardes nodig, zoals besproken in \cref{sec:werkingISFET}. Het digitale gedeelte heeft als ingang een temperatuursafhankelijke spanning $U_T$ en een pH-afhankelijke spanning $U_{pH}$. Dit is te zien in \cref{fig:digitaleBewerkingsFunctie}. Beide van deze spanningen zijn de ruwe ADC waardes die gemeten worden, en hebben dus de hoogst mogelijke resolutie, namelijk de resolutie van de ADC. Voor beide van deze waardes zal de eenheid `bit' gebruikt worden.

Om op de uiteindelijk pH waarde te komen, wordt van de pH-afhankelijke spanning de pH-afhankelijke kalibratiespanning afgetrokken. Vervolgens wordt hier $\frac{pH_{kal}}{C_{pH}}$ bij opgeteld. Op deze manier kan er zo lang mogelijk met integers gewerkt worden die dezelfde resolutie hebben als de ingangs-ADC waarde. Het resultaat wordt vervolgens vermenigvuldigd met $C_{pH}$. Dit is de gevoeligheid van de sensor, in pH/bit.

Om de temperatuursafwijking te berekenen, wordt eerst van de temperatuursafhankelijke spanning $U_T$ de temperatuursafhankelijke kalibratiespanning $U_{T,kal}$ afgetrokken. Vervolgens wordt dit vermenigvuldigt met constante $C_T$. $C_T$ is de temperatuursafhankelijkheid van de pH-sensor, in pH/bit. Deze waarde kan afgeleid worden met de temperatuursafhankelijkheid van de pH-sensor die in de datasheet gegeven wordt in mV/K \cite{isfet}.

\begin{figure}[ht]
    \centering
    \includegraphics[width=0.95\textwidth]{digitaleBewerkingsFunctie}
    \caption{Het digitale gedeelte van de signaalbewerking.} 
    \label{fig:digitaleBewerkingsFunctie}
\end{figure}


\section{De drempelspanning van de ISFET uitlezen}

% TODO: Bronnen
Om de pH waarde van de ISFET uit te lezen moet de drempelspanning gemeten worden. Deze is namelijk linear afhankelijk van de pH waarde.
Om de drempelspanning te meten kan een regelsysteem gebruikt worden. Door de gate-source spanning te variëren kan de spanning over en de stroom door de drain en de source van de ISFET gelijk gehouden worden.

Er zijn meerdere mogelijke implementaties van een dergelijk regelsysteem. In \autoref{fig:measureCircuits} staan er drie.
Elk van deze schakelingen gebruikt een nullor om de drain-source spanning van de ISFET gelijk te houden. Ook gebruikt elk van deze schakelingen een referentiespanning. De implementatie van deze referentiespanning wordt verder besproken in \autoref{sec:referenceVoltage}.

De drain-source spanning $U_{ds}$ en drain-source stroom $I_{ds}$ zijn van te voren gedefinieerd. Deze zijn ook te vinden in de datasheet van de ISFET\cite{isfet}. Uit deze twee waardes kunnen de referentiespanningen en weerstandswaardes van de schakelingen gevonden worden.
Voor de schakeling in \autoref{fig:measureCurrent} is de spanningsreferentie te vinden door middel van \autoref{eq:URefSource}.
\begin{equation}\label{eq:URefSource}
    U_{ref,s} = U_{dd} - U_{ds}
\end{equation}
Voor \autoref{fig:measureResistor} is de referentiespanning gelijk aan de drain-source spanning.
\begin{equation}\label{eq:URefDrain}
    U_{ref,d} = U_{ds}
\end{equation}
Voor de waarde van de weerstand in \autoref{fig:measureResistor} kan \autoref{eq:measureResistorVal} gebruikt worden.
\begin{equation}\label{eq:measureResistorVal}
    R = \frac{U_{dd} - U_{ds}}{I_{ds}}
\end{equation}


\begin{figure}[ht]
    \centering
    \begin{subfigure}[b]{0.45\textwidth}
        \centering
        \def\svgwidth{\textwidth}
        \input{img/ISFETCircuitBest.pdf_tex}
        \caption{Met een weerstand aan de drain.}
        \label{fig:measureResistor}
    \end{subfigure}
    \hfill
    \begin{subfigure}[b]{0.45\textwidth}
        \centering
        \def\svgwidth{\textwidth}
        \input{img/ISFETCircuit.pdf_tex}
        \caption{Met een stroombron.}
        \label{fig:measureCurrent}
    \end{subfigure}
    \caption{De uitleesschakelingen voor de ISFET.}
    \label{fig:measureCircuits}
\end{figure}

Beide schakelingen heeft voor- en nadelen.
Bij de schakeling in \autoref{fig:measureResistor} zit de source van de ISFET direct verbonden met de aarde. Dit heeft als voordeel dat de uitgang van de nullor gelijk is aan de gate-source spanning. Hierdoor hoeft de nullor lagere spanningen te genereren om de gate-source spanning van de mosfet naar de goede waarde te krijgen. De spanning die de nullor moet genereren in het geval van een drain weerstand is te vinden door middel van \autoref{eq:nullorVoltageDrain}. In het geval van een stroombron aan de source is dat \autoref{eq:nullorVoltageSource}.

\begin{equation}\label{eq:nullorVoltageDrain}
    U_{nullor,d} = U_{gs}
    \tagaddtext{[\si{\volt}]}
\end{equation}
\begin{equation}\label{eq:nullorVoltageSource}
    U_{nullor,s} = U_{gs} + U_{ref,s}
    \tagaddtext{[\si{\volt}]}
\end{equation}

De schakeling met een stroombron aan de source heeft de mogelijkheid om betere ruiseigenschappen te hebben. De stroombron kan ook een hogere impedantie hebben dan de weerstand, wat goed is volgens mij. 

\newcommand\ph{\mathrm{pH}}

De meetschakeling heeft een aantal ruisbronnen. De nullor heeft een ingangsstroom- en spanninsruisbron. Daarnaast genereert de weerstand ook thermische ruis. Deze ruisbronnen zijn te zien in \autoref{fig:measureNoise}.
\begin{figure}[ht]
    \centering
    \def\svgwidth{0.6\textwidth}
    \input{img/ISFETCircuitBestNoise.pdf_tex}
    \caption{De ruisbronnen van de meetschakeling.}
    \label{fig:measureNoise}
\end{figure}

De overdracht van deze schakeling is gelijk aan de uitgangsspanning gedeeld door de ingangsspanning van de nullor. Door de werking van de schakeling blijft de ingansspanning altijd gelijk en is de uitgangsspanning lineair afhankelijk van de pH waarde. Hierdoor is de overdracht $H(\ph)$ een functie van de gemeten pH waarde.
Omdat $U_{ds}$ en $I_{ds}$ van de ISFET niet veranderen, kan de impedantie ervan gezien worden als weerstand, met een waarde van $\frac{U_{ds}}{I_{ds}}$. Hierdoor wordt er een nieuwe ruisbron $i_{n,ds}$ toegevoegd. Met deze weerstand kunnen de bronnen $i_{n,ref}$, $i_{n,R}$ en $i_{n,ds}$ worden getransformeerd naar een spanningsbron $u{n,in}$ aan de ingang van de nullor.
Vervolgens kan deze, samen met de spanningsruisbronnen $u_{n,ref}$ en $u_{n,n}$ naar de uitgang getransformeerd worden. Dit komt uit op een spanningsruisbron aan de uitgang, zoals te zien in \autoref{fig:measureNoiseMoved}. De spectrale spanningsruisdichtheid hiervan is te berekenen door middel van \autoref{eq:measureNoiseOut}.

\begin{equation}\label{eq:measureNoiseOut}
    Su_{n,out} = \left(Su_{n,ref} + Su_{n,n} + Si_{n,in}\left(Z_{fet} // R\right)^2\right) \cdot H^2(\ph)
    \tagaddtext{[\si{\volt\squared\per\hertz}]}
\end{equation}
\begin{equation}
    Si_{n,in} = Si_{n,n} + Si_{n,R} + Si_{n,ds}
    \tagaddtext{[\si{\ampere\squared\per\hertz}]}
    \label{eq:measureNoiseCurrentIn}
\end{equation}


De waardes van deze ruisbronnen zijn te vinden in \autoref{tab:measureNoiseValues}.

\begin{table}[ht]
    \centering
    \begin{tabular}{c|l}
        Ruisbron & Waarde \\
        \hline 
        $Su_{n,ref}$ & Zie \autoref{sec:referenceVoltage} \\
        $Su_{n,n}$   & Implementatie nullor \\
        $Si_{n,n}$   & Implementatie nullor \\
        $Si_{n,R}$   & $\frac{4kT}{R}$ \\
        $Si_{n,ds}$  & $4kT\frac{I_{ds}}{U_{ds}}$ \\
    \end{tabular}
    \caption{Waar de waardes van de ruisbronnen vandaan gehaald kunnen worden.}
    \label{tab:measureNoiseValues}
\end{table}

\begin{figure}[ht]
    \centering
    \def\svgwidth{0.6\textwidth}
    \input{img/ISFETCircuitBestNoiseMoved.pdf_tex}
    \caption{De meetschakeling met verschoven ruisbronnen.}
    \label{fig:measureNoiseMoved}
\end{figure}
\subsection{Spanningsreferentie}\label{sec:referenceVoltage}

De ISFET uitleesschakeling heeft een spanningsreferentie nodig om te werken. Deze spanningsreferentie kan op meerdere manieren gegenereerd worden.
% TODO: Vertel misschien over andere methoden.
Uiteindelijk is er een spanningsdeler gekozen om de spanningsreferentie mee te implementeren. De schakeling van deze spanningsdeler is te zien in \autoref{fig:divider}.
De condensator wordt gebruikt om ruis te verminderen op hogere frequenties, en dient ook als filter voor hoogfrequente fouten in de voedingsspanning.

\begin{figure}[ht]
    \centering
    \def\svgwidth{0.5\textwidth}
    \subsection{Spanningsreferentie}\label{sec:referenceVoltage}

De ISFET uitleesschakeling heeft een spanningsreferentie nodig om te werken. Deze spanningsreferentie kan op meerdere manieren gegenereerd worden.
% TODO: Vertel misschien over andere methoden.
Uiteindelijk is er een spanningsdeler gekozen om de spanningsreferentie mee te implementeren. De schakeling van deze spanningsdeler is te zien in \autoref{fig:divider}.
De condensator wordt gebruikt om ruis te verminderen op hogere frequenties, en dient ook als filter voor hoogfrequente fouten in de voedingsspanning.

\begin{figure}[ht]
    \centering
    \def\svgwidth{0.5\textwidth}
    \subsection{Spanningsreferentie}\label{sec:referenceVoltage}

De ISFET uitleesschakeling heeft een spanningsreferentie nodig om te werken. Deze spanningsreferentie kan op meerdere manieren gegenereerd worden.
% TODO: Vertel misschien over andere methoden.
Uiteindelijk is er een spanningsdeler gekozen om de spanningsreferentie mee te implementeren. De schakeling van deze spanningsdeler is te zien in \autoref{fig:divider}.
De condensator wordt gebruikt om ruis te verminderen op hogere frequenties, en dient ook als filter voor hoogfrequente fouten in de voedingsspanning.

\begin{figure}[ht]
    \centering
    \def\svgwidth{0.5\textwidth}
    \input{img/divider.pdf_tex}
    \caption{De schakeling van de spanningsdeler die dient als spanningsreferentie.}
    \label{fig:divider}
\end{figure}

\noindent
De overdracht van deze spanningsdeler is te vinden in \autoref{eq:dividerTransfer}.
\begin{equation}\label{eq:dividerTransfer}
    H(s) = \frac{U_{ref}(s)}{U_{dd}(s)} = \frac{R_2}{R_1 + R_2 + R_2Cs}
\end{equation}

\noindent
Het vermogen dat de spanningsdeler dissipeert, kan met \autoref{eq:dividerPower} berekend worden.
\begin{equation}\label{eq:dividerPower}
    P(s) = U_{dd}(s)^2\frac{1+R_2Cs}{R_1 + R_2 + R_1R_2Cs}
\end{equation}
Met een constante DC ingangsspanning kan dit vereenvoudigd worden naar \autoref{eq:dividerPowerSimple}.
\begin{equation}\label{eq:dividerPowerSimple}
    P = \frac{U_{dd}^2}{R_1 + R_2}
\end{equation}

\noindent
Om de ruis van deze schakeling te berekenen moet een aantal stappen genomen worden. Aangezien de ingangsbron $U_{dd}$ een spanningsbron is, kan deze als kortsluiting genomen worden. Op deze manier kunnen de twee weerstanden parallel genomen worden, en verandert de schakeling in een simpel RC filter. In \autoref{fig:dividerNoise} is deze omgebouwde schakeling te zien.

\begin{figure}[ht]
    \centering
    \def\svgwidth{0.35\textwidth}
    \input{img/dividerNoise.pdf_tex}
    \caption{De omgebouwde schakeling om ruis mee te berekenen.}
    \label{fig:dividerNoise}
\end{figure}

\noindent
Voor de spectrale spanningsruisdichtheid aan de uitgang $U_{ref}$ kan \autoref{eq:dividerNoiseLaplace} worden opgesteld.
\begin{equation}\label{eq:dividerNoiseLaplace}
    S_{n,u_{ref}} = 4kTR_e\left(\frac{1}{1 + R_eCs}\right)^2
\end{equation}
Wanneer de absolute waarde van de ruis wordt genomen, kan deze over de bandbreedte geïntegreerd worden. Dit resulteert in \autoref{eq:dividerNoiseInt}.
\begin{equation}\label{eq:dividerNoiseInt}
    u_{n,ref}^2 = \int_{\omega_l}^{\omega_h} 4kTR_e\left(\frac{1}{\sqrt{1 + (R_eC\omega)^2}}\right)^2 d\omega
\end{equation}
Het integraal van deze formule komt uit op \autoref{eq:dividerNoiseIntegrated}.
\begin{equation}\label{eq:dividerNoiseIntegrated}
    u_{n,ref}^2 = \frac{4kT}{C}\left[\arctan(R_eC\omega_h) - \arctan(R_eC\omega_l)\right]
\end{equation}

Zolang de spanning stabiel is hoeft de referentie geen exact gedefinieerde spanning te hebben. Dit is omdat de ADC die de waarde van de sensor gaat uitlezen deze referentiespanning ook als referentie zal gebruiken. De waarde moet echter ergens rond de 1.1V zitten, om de stroombron te laten werken.
In \autoref{sec:currentSource} is hier meer over te lezen.
Aangezien de ingangsspanning 1.8V is, zal de DC overdracht $1.1 / 1.8 = \frac{11}{18}$ zijn. Hieruit komt de weerstandsverhouding in \autoref{eq:dividerResistors}.
\begin{equation}\label{eq:dividerResistors}
    \frac{R_1}{R_2} = \frac{7}{11}
\end{equation}
Een hogere $R_1$ zorgt voor een lager vermogensverbruik, maar ook een hogere ruis. Dit is te zien in \autoref{fig:dividerPlots}.

\begin{figure}
    \centering
    \begin{subfigure}[b]{0.45\textwidth}
        \centering
        \input{plots/dividerNoise}
        \caption{Spanningsruis}
        \label{fig:dividerNoisePlot}
    \end{subfigure}
    \hfill
    \begin{subfigure}[b]{0.45\textwidth}
        \centering
        \input{plots/dividerPower}
        \caption{Vermogensverbruik}
        \label{fig:dividerPower}
    \end{subfigure}
    \caption{De ruis en het vermogensverbruik van de spanningsdeler, ten opzichte van de gekozen weerstandswaarde $R_1$.}
    \label{fig:dividerPlots}
\end{figure}
Op $R_1 = 1\si{\mega\ohm}$ en $C = 10\si{\micro\farad}$ is de spanningsruis $u_{n,ref} = 51\si{\nano\volt}$ en het vermogensverbruik $P = 1.3\si{\micro\watt}$. De andere weerstandswaarde is dan $R_2 \approx 1.6 \si{\mega\ohm}$. Deze waardes vallen binnen de specificaties.

[WELKE SPECS??]
    \caption{De schakeling van de spanningsdeler die dient als spanningsreferentie.}
    \label{fig:divider}
\end{figure}

\noindent
De overdracht van deze spanningsdeler is te vinden in \autoref{eq:dividerTransfer}.
\begin{equation}\label{eq:dividerTransfer}
    H(s) = \frac{U_{ref}(s)}{U_{dd}(s)} = \frac{R_2}{R_1 + R_2 + R_2Cs}
\end{equation}

\noindent
Het vermogen dat de spanningsdeler dissipeert, kan met \autoref{eq:dividerPower} berekend worden.
\begin{equation}\label{eq:dividerPower}
    P(s) = U_{dd}(s)^2\frac{1+R_2Cs}{R_1 + R_2 + R_1R_2Cs}
\end{equation}
Met een constante DC ingangsspanning kan dit vereenvoudigd worden naar \autoref{eq:dividerPowerSimple}.
\begin{equation}\label{eq:dividerPowerSimple}
    P = \frac{U_{dd}^2}{R_1 + R_2}
\end{equation}

\noindent
Om de ruis van deze schakeling te berekenen moet een aantal stappen genomen worden. Aangezien de ingangsbron $U_{dd}$ een spanningsbron is, kan deze als kortsluiting genomen worden. Op deze manier kunnen de twee weerstanden parallel genomen worden, en verandert de schakeling in een simpel RC filter. In \autoref{fig:dividerNoise} is deze omgebouwde schakeling te zien.

\begin{figure}[ht]
    \centering
    \def\svgwidth{0.35\textwidth}
    \begin{tikzpicture}
    \pgfplotsset{width=\textwidth}
    \newcommand\BOLZ{1.380649e-23}
    \newcommand\TEMP{300}
    \newcommand\OMEGAC{15*2*pi}
    \newcommand\RESRAT{(7/11)}
    \newcommand\REQU{(1/(1/x + \RESRAT/x))}
    \newcommand\CAP{0.000001}

    \pgfplotsset{set layers}
    \begin{axis}[
        xmode=log,
        ymode=log,
        xlabel={$R_1 [\si{\ohm}]$},
        ylabel={$u_{n,out} [\si{\volt}]$},
        xmin=1e3, xmax=1e7,
        grid=major
    ]

    \addplot [
        red,
        domain=1e3:1e7,
        samples=201
    ]
    {sqrt((4 * \BOLZ * \TEMP / \CAP) * rad(atan(\REQU * \CAP * \OMEGAC)))};
    \end{axis}
\end{tikzpicture}
    \caption{De omgebouwde schakeling om ruis mee te berekenen.}
    \label{fig:dividerNoise}
\end{figure}

\noindent
Voor de spectrale spanningsruisdichtheid aan de uitgang $U_{ref}$ kan \autoref{eq:dividerNoiseLaplace} worden opgesteld.
\begin{equation}\label{eq:dividerNoiseLaplace}
    S_{n,u_{ref}} = 4kTR_e\left(\frac{1}{1 + R_eCs}\right)^2
\end{equation}
Wanneer de absolute waarde van de ruis wordt genomen, kan deze over de bandbreedte geïntegreerd worden. Dit resulteert in \autoref{eq:dividerNoiseInt}.
\begin{equation}\label{eq:dividerNoiseInt}
    u_{n,ref}^2 = \int_{\omega_l}^{\omega_h} 4kTR_e\left(\frac{1}{\sqrt{1 + (R_eC\omega)^2}}\right)^2 d\omega
\end{equation}
Het integraal van deze formule komt uit op \autoref{eq:dividerNoiseIntegrated}.
\begin{equation}\label{eq:dividerNoiseIntegrated}
    u_{n,ref}^2 = \frac{4kT}{C}\left[\arctan(R_eC\omega_h) - \arctan(R_eC\omega_l)\right]
\end{equation}

Zolang de spanning stabiel is hoeft de referentie geen exact gedefinieerde spanning te hebben. Dit is omdat de ADC die de waarde van de sensor gaat uitlezen deze referentiespanning ook als referentie zal gebruiken. De waarde moet echter ergens rond de 1.1V zitten, om de stroombron te laten werken.
In \autoref{sec:currentSource} is hier meer over te lezen.
Aangezien de ingangsspanning 1.8V is, zal de DC overdracht $1.1 / 1.8 = \frac{11}{18}$ zijn. Hieruit komt de weerstandsverhouding in \autoref{eq:dividerResistors}.
\begin{equation}\label{eq:dividerResistors}
    \frac{R_1}{R_2} = \frac{7}{11}
\end{equation}
Een hogere $R_1$ zorgt voor een lager vermogensverbruik, maar ook een hogere ruis. Dit is te zien in \autoref{fig:dividerPlots}.

\begin{figure}
    \centering
    \begin{subfigure}[b]{0.45\textwidth}
        \centering
        \begin{tikzpicture}
    \pgfplotsset{width=\textwidth}
    \newcommand\BOLZ{1.380649e-23}
    \newcommand\TEMP{300}
    \newcommand\OMEGAC{15*2*pi}
    \newcommand\RESRAT{(7/11)}
    \newcommand\REQU{(1/(1/x + \RESRAT/x))}
    \newcommand\CAP{0.000001}

    \pgfplotsset{set layers}
    \begin{axis}[
        xmode=log,
        ymode=log,
        xlabel={$R_1 [\si{\ohm}]$},
        ylabel={$u_{n,out} [\si{\volt}]$},
        xmin=1e3, xmax=1e7,
        grid=major
    ]

    \addplot [
        red,
        domain=1e3:1e7,
        samples=201
    ]
    {sqrt((4 * \BOLZ * \TEMP / \CAP) * rad(atan(\REQU * \CAP * \OMEGAC)))};
    \end{axis}
\end{tikzpicture}
        \caption{Spanningsruis}
        \label{fig:dividerNoisePlot}
    \end{subfigure}
    \hfill
    \begin{subfigure}[b]{0.45\textwidth}
        \centering
        \begin{tikzpicture}
    \pgfplotsset{width=\textwidth}
    \newcommand\OMEGAC{10*2*pi}
    \newcommand\RESRAT{(7/11)}

    \begin{axis}[
        xmode=log,
        ymode=log,
        xlabel={$R_1 [\si{\ohm}]$},
        ylabel={$P [\si{\watt}]$},
        xmin=1e3, xmax=2e6,
        grid=major
    ]

    \addplot [
        blue,
        domain=1e3:2e6,
        samples=201
    ]
    {1.8 / (x + x/\RESRAT)};
    \end{axis}
\end{tikzpicture}
        \caption{Vermogensverbruik}
        \label{fig:dividerPower}
    \end{subfigure}
    \caption{De ruis en het vermogensverbruik van de spanningsdeler, ten opzichte van de gekozen weerstandswaarde $R_1$.}
    \label{fig:dividerPlots}
\end{figure}
Op $R_1 = 1\si{\mega\ohm}$ en $C = 10\si{\micro\farad}$ is de spanningsruis $u_{n,ref} = 51\si{\nano\volt}$ en het vermogensverbruik $P = 1.3\si{\micro\watt}$. De andere weerstandswaarde is dan $R_2 \approx 1.6 \si{\mega\ohm}$. Deze waardes vallen binnen de specificaties.

[WELKE SPECS??]
    \caption{De schakeling van de spanningsdeler die dient als spanningsreferentie.}
    \label{fig:divider}
\end{figure}

\noindent
De overdracht van deze spanningsdeler is te vinden in \autoref{eq:dividerTransfer}.
\begin{equation}\label{eq:dividerTransfer}
    H(s) = \frac{U_{ref}(s)}{U_{dd}(s)} = \frac{R_2}{R_1 + R_2 + R_2Cs}
\end{equation}

\noindent
Het vermogen dat de spanningsdeler dissipeert, kan met \autoref{eq:dividerPower} berekend worden.
\begin{equation}\label{eq:dividerPower}
    P(s) = U_{dd}(s)^2\frac{1+R_2Cs}{R_1 + R_2 + R_1R_2Cs}
\end{equation}
Met een constante DC ingangsspanning kan dit vereenvoudigd worden naar \autoref{eq:dividerPowerSimple}.
\begin{equation}\label{eq:dividerPowerSimple}
    P = \frac{U_{dd}^2}{R_1 + R_2}
\end{equation}

\noindent
Om de ruis van deze schakeling te berekenen moet een aantal stappen genomen worden. Aangezien de ingangsbron $U_{dd}$ een spanningsbron is, kan deze als kortsluiting genomen worden. Op deze manier kunnen de twee weerstanden parallel genomen worden, en verandert de schakeling in een simpel RC filter. In \autoref{fig:dividerNoise} is deze omgebouwde schakeling te zien.

\begin{figure}[ht]
    \centering
    \def\svgwidth{0.35\textwidth}
    \begin{tikzpicture}
    \pgfplotsset{width=\textwidth}
    \newcommand\BOLZ{1.380649e-23}
    \newcommand\TEMP{300}
    \newcommand\OMEGAC{15*2*pi}
    \newcommand\RESRAT{(7/11)}
    \newcommand\REQU{(1/(1/x + \RESRAT/x))}
    \newcommand\CAP{0.000001}

    \pgfplotsset{set layers}
    \begin{axis}[
        xmode=log,
        ymode=log,
        xlabel={$R_1 [\si{\ohm}]$},
        ylabel={$u_{n,out} [\si{\volt}]$},
        xmin=1e3, xmax=1e7,
        grid=major
    ]

    \addplot [
        red,
        domain=1e3:1e7,
        samples=201
    ]
    {sqrt((4 * \BOLZ * \TEMP / \CAP) * rad(atan(\REQU * \CAP * \OMEGAC)))};
    \end{axis}
\end{tikzpicture}
    \caption{De omgebouwde schakeling om ruis mee te berekenen.}
    \label{fig:dividerNoise}
\end{figure}

\noindent
Voor de spectrale spanningsruisdichtheid aan de uitgang $U_{ref}$ kan \autoref{eq:dividerNoiseLaplace} worden opgesteld.
\begin{equation}\label{eq:dividerNoiseLaplace}
    S_{n,u_{ref}} = 4kTR_e\left(\frac{1}{1 + R_eCs}\right)^2
\end{equation}
Wanneer de absolute waarde van de ruis wordt genomen, kan deze over de bandbreedte geïntegreerd worden. Dit resulteert in \autoref{eq:dividerNoiseInt}.
\begin{equation}\label{eq:dividerNoiseInt}
    u_{n,ref}^2 = \int_{\omega_l}^{\omega_h} 4kTR_e\left(\frac{1}{\sqrt{1 + (R_eC\omega)^2}}\right)^2 d\omega
\end{equation}
Het integraal van deze formule komt uit op \autoref{eq:dividerNoiseIntegrated}.
\begin{equation}\label{eq:dividerNoiseIntegrated}
    u_{n,ref}^2 = \frac{4kT}{C}\left[\arctan(R_eC\omega_h) - \arctan(R_eC\omega_l)\right]
\end{equation}

Zolang de spanning stabiel is hoeft de referentie geen exact gedefinieerde spanning te hebben. Dit is omdat de ADC die de waarde van de sensor gaat uitlezen deze referentiespanning ook als referentie zal gebruiken. De waarde moet echter ergens rond de 1.1V zitten, om de stroombron te laten werken.
In \autoref{sec:currentSource} is hier meer over te lezen.
Aangezien de ingangsspanning 1.8V is, zal de DC overdracht $1.1 / 1.8 = \frac{11}{18}$ zijn. Hieruit komt de weerstandsverhouding in \autoref{eq:dividerResistors}.
\begin{equation}\label{eq:dividerResistors}
    \frac{R_1}{R_2} = \frac{7}{11}
\end{equation}
Een hogere $R_1$ zorgt voor een lager vermogensverbruik, maar ook een hogere ruis. Dit is te zien in \autoref{fig:dividerPlots}.

\begin{figure}
    \centering
    \begin{subfigure}[b]{0.45\textwidth}
        \centering
        \begin{tikzpicture}
    \pgfplotsset{width=\textwidth}
    \newcommand\BOLZ{1.380649e-23}
    \newcommand\TEMP{300}
    \newcommand\OMEGAC{15*2*pi}
    \newcommand\RESRAT{(7/11)}
    \newcommand\REQU{(1/(1/x + \RESRAT/x))}
    \newcommand\CAP{0.000001}

    \pgfplotsset{set layers}
    \begin{axis}[
        xmode=log,
        ymode=log,
        xlabel={$R_1 [\si{\ohm}]$},
        ylabel={$u_{n,out} [\si{\volt}]$},
        xmin=1e3, xmax=1e7,
        grid=major
    ]

    \addplot [
        red,
        domain=1e3:1e7,
        samples=201
    ]
    {sqrt((4 * \BOLZ * \TEMP / \CAP) * rad(atan(\REQU * \CAP * \OMEGAC)))};
    \end{axis}
\end{tikzpicture}
        \caption{Spanningsruis}
        \label{fig:dividerNoisePlot}
    \end{subfigure}
    \hfill
    \begin{subfigure}[b]{0.45\textwidth}
        \centering
        \begin{tikzpicture}
    \pgfplotsset{width=\textwidth}
    \newcommand\OMEGAC{10*2*pi}
    \newcommand\RESRAT{(7/11)}

    \begin{axis}[
        xmode=log,
        ymode=log,
        xlabel={$R_1 [\si{\ohm}]$},
        ylabel={$P [\si{\watt}]$},
        xmin=1e3, xmax=2e6,
        grid=major
    ]

    \addplot [
        blue,
        domain=1e3:2e6,
        samples=201
    ]
    {1.8 / (x + x/\RESRAT)};
    \end{axis}
\end{tikzpicture}
        \caption{Vermogensverbruik}
        \label{fig:dividerPower}
    \end{subfigure}
    \caption{De ruis en het vermogensverbruik van de spanningsdeler, ten opzichte van de gekozen weerstandswaarde $R_1$.}
    \label{fig:dividerPlots}
\end{figure}
Op $R_1 = 1\si{\mega\ohm}$ en $C = 10\si{\micro\farad}$ is de spanningsruis $u_{n,ref} = 51\si{\nano\volt}$ en het vermogensverbruik $P = 1.3\si{\micro\watt}$. De andere weerstandswaarde is dan $R_2 \approx 1.6 \si{\mega\ohm}$. Deze waardes vallen binnen de specificaties.

[WELKE SPECS??]
\begin{frame}
    \frametitle{ADC}
    
    \begin{figure}
        \centering
        \includegraphics[width=\textwidth]{adcBlock}
    \end{figure}

\end{frame}

\begin{frame}
    \frametitle{Eisen}

    \begin{table}[ht]
        \centering
        \begin{tabular}{l|c|l}
            Symbol      & Waarde & Eenheid\\\hline
            $SNR_{in}$  & 37        & dB\\
            NF          & 3         & dB\\
            $u_{in}$    & 2.5       & mV\\
        \end{tabular}
        \caption{De eisen voor het omzetten van het analoge signaal naar een digitaal signaal.}
        \label{tab:systemSpecADC}
    \end{table}
    
\end{frame}
\begin{frame}
    \frametitle{Minimum aantal bits}
    \centering

    De toelaatbare fout ten gevolge van de eindige resolutie van de ADC
    \begin{equation}\label{eq:calcSpecifiedRmsError}
        \overline{e_{eff}^2}=\left(10^{\frac{NF}{10}}-1\right)\left(\frac{S_{rms}}{10^{\left(SNR+NF\right)/20}}\right)^2
    \end{equation}
    \pause

    Berekenen minimum benodigde ADC resolutie
    \begin{equation}\label{eq:calcNeededQ}
        Q=\sqrt{12\cdot\overline{e_{eff}^2}}
    \end{equation}
    \pause

    Berekenen minimum aantal bits van de ADC op basis van de minimaal benodigde ADC resolutie 
    \begin{equation}\label{eq:calcMinNumberADCbits}
        n=\left\lceil\log_2\left(\frac{1}{Q}+1\right)\right\rceil=14
    \end{equation}

\end{frame}

\begin{frame}
    \frametitle{Sample frequentie}
    \centering
    
    Toelaatbare fout
    \begin{equation}\label{eq:ADCmaxSampleError}
        E=10^{\frac{-NF}{10}}
    \end{equation}
    \pause

    Minimale sample frequentie
    \begin{equation}\label{eq:ADCminFs}
        f_{s,min}=\frac{\pi f_h}{E}= \qty{45}{\hertz}
    \end{equation}

\end{frame}

\subsection{Filter}
Tussen de ADC en de uitleesschakeling van de sensor zit een filter. Dit filter zorgt ervoor dat alle frequenties buiten de bandbreedte weggefilterd worden. Er is gekozen om hiervoor een eerste orde laagdoorlaatfilter te gebruiken.
De schakeling van dit filter is te zien in \cref{fig:filterCircuit}. De kantelfrequentie van het filter ligt aan de waardes van $C$ en $R$, volgens \cref{eq:cutoffFreq}.
\begin{figure}[!htbp]
    \centering
    \def\svgwidth{0.3\textwidth}
    \subsection{Filter}
Tussen de ADC en de uitleesschakeling van de sensor zit een filter. Dit filter zorgt ervoor dat alle frequenties buiten de bandbreedte weggefilterd worden. Er is gekozen om hiervoor een eerste orde laagdoorlaatfilter te gebruiken.
De schakeling van dit filter is te zien in \cref{fig:filterCircuit}. De kantelfrequentie van het filter ligt aan de waardes van $C$ en $R$, volgens \cref{eq:cutoffFreq}.
\begin{figure}[!htbp]
    \centering
    \def\svgwidth{0.3\textwidth}
    \subsection{Filter}
Tussen de ADC en de uitleesschakeling van de sensor zit een filter. Dit filter zorgt ervoor dat alle frequenties buiten de bandbreedte weggefilterd worden. Er is gekozen om hiervoor een eerste orde laagdoorlaatfilter te gebruiken.
De schakeling van dit filter is te zien in \cref{fig:filterCircuit}. De kantelfrequentie van het filter ligt aan de waardes van $C$ en $R$, volgens \cref{eq:cutoffFreq}.
\begin{figure}[!htbp]
    \centering
    \def\svgwidth{0.3\textwidth}
    \input{img/filter.pdf_tex}
    \caption{Het eerste-orde filter.}
    \label{fig:filterCircuit}
\end{figure}
\begin{equation} \label{eq:cutoffFreq}
    2\pi f_c = \omega_c = \frac{1}{RC}
    \tagaddtext{[\si{\radian\per\second}]}
\end{equation}

\subsubsection{Ruis}
De spectrale ruisdichtheid aan de ingang van het filter is te berekenen met \cref{eq:filterNoiseDensity}.
De spectrale ruisdichtheid aan de uitgang van het filter is hetzelfde als die van de spanningsdeler in \cref{sec:referenceVoltage}. Deze is te berekenen met \cref{eq:dividerNoise}.


% TODO: BEPAAL OVERDRACHT

% \begin{equation} \label{eq:filterTransfer}
%     H(s) = \frac{1}{1+sRC}
% \end{equation}

\begin{equation} \label{eq:filterNoiseDensity}
    S_{u_{in}} = 4kTR
    \tagaddtext{[\si{\volt\squared\per\hertz}]}
\end{equation}

% De signaal-ruis verhouding aan de uitgang van dit filter is te berekenen met \cref{eq:filterSNR}
% \begin{equation}\label{eq:filterSNR}
%     \mathrm{SNR} = 20\log\left(U_{out,min}\sqrt{\frac{C}{kT}}\right)
%     \tagaddtext{[\si{\decibel}]}
% \end{equation}

\subsubsection{Vermogen}
Het vermogensverbruik van het filter is te berekenen met \cref{eq:filterPowerLaplace}.
\begin{equation} \label{eq:filterPowerLaplace}
    P = \frac{U_{in,max}^2}{\left|R + \frac{1}{sC}\right|}
    \tagaddtext{[\si{\watt}]}
\end{equation}
Omdat volgens \cref{eq:cutoffFreq} $R$ te definiëren is in $\omega_c$ en $C$, volgt hieruit \cref{eq:filterPower}.
\begin{equation} \label{eq:filterPower}
    P = \frac{1}{2}\omega_cCU_{in,max}^2
    \tagaddtext{[\si{\watt}]}
\end{equation}
In deze formule is te zien dat het vermogensverbruik lineair evenredig is met de condensatorwaarde. Om het vermogensverbruik te minimaliseren moet dus een zo klein mogelijke condensatorwaarde gekozen worden. Aangezien de noise-figure van dit filter maximaal 3dB mag zijn, mag dit filter maximaal evenveel spanningsruis genereren als het systeem ervoor. Hieruit volgt \cref{eq:filterCapMin}, waarmee de minimale condensatorwaarde te berekenen is. Hierbij is $u_{n,in}$ de ruisspanning aan de ingang van het filter.
\begin{equation} \label{eq:filterCapMin}
    C_{min} = \frac{kT}{u_{n,in}^2}
    \tagaddtext{[\si{\farad}]}
\end{equation}
    \caption{Het eerste-orde filter.}
    \label{fig:filterCircuit}
\end{figure}
\begin{equation} \label{eq:cutoffFreq}
    2\pi f_c = \omega_c = \frac{1}{RC}
    \tagaddtext{[\si{\radian\per\second}]}
\end{equation}

\subsubsection{Ruis}
De spectrale ruisdichtheid aan de ingang van het filter is te berekenen met \cref{eq:filterNoiseDensity}.
De spectrale ruisdichtheid aan de uitgang van het filter is hetzelfde als die van de spanningsdeler in \cref{sec:referenceVoltage}. Deze is te berekenen met \cref{eq:dividerNoise}.


% TODO: BEPAAL OVERDRACHT

% \begin{equation} \label{eq:filterTransfer}
%     H(s) = \frac{1}{1+sRC}
% \end{equation}

\begin{equation} \label{eq:filterNoiseDensity}
    S_{u_{in}} = 4kTR
    \tagaddtext{[\si{\volt\squared\per\hertz}]}
\end{equation}

% De signaal-ruis verhouding aan de uitgang van dit filter is te berekenen met \cref{eq:filterSNR}
% \begin{equation}\label{eq:filterSNR}
%     \mathrm{SNR} = 20\log\left(U_{out,min}\sqrt{\frac{C}{kT}}\right)
%     \tagaddtext{[\si{\decibel}]}
% \end{equation}

\subsubsection{Vermogen}
Het vermogensverbruik van het filter is te berekenen met \cref{eq:filterPowerLaplace}.
\begin{equation} \label{eq:filterPowerLaplace}
    P = \frac{U_{in,max}^2}{\left|R + \frac{1}{sC}\right|}
    \tagaddtext{[\si{\watt}]}
\end{equation}
Omdat volgens \cref{eq:cutoffFreq} $R$ te definiëren is in $\omega_c$ en $C$, volgt hieruit \cref{eq:filterPower}.
\begin{equation} \label{eq:filterPower}
    P = \frac{1}{2}\omega_cCU_{in,max}^2
    \tagaddtext{[\si{\watt}]}
\end{equation}
In deze formule is te zien dat het vermogensverbruik lineair evenredig is met de condensatorwaarde. Om het vermogensverbruik te minimaliseren moet dus een zo klein mogelijke condensatorwaarde gekozen worden. Aangezien de noise-figure van dit filter maximaal 3dB mag zijn, mag dit filter maximaal evenveel spanningsruis genereren als het systeem ervoor. Hieruit volgt \cref{eq:filterCapMin}, waarmee de minimale condensatorwaarde te berekenen is. Hierbij is $u_{n,in}$ de ruisspanning aan de ingang van het filter.
\begin{equation} \label{eq:filterCapMin}
    C_{min} = \frac{kT}{u_{n,in}^2}
    \tagaddtext{[\si{\farad}]}
\end{equation}
    \caption{Het eerste-orde filter.}
    \label{fig:filterCircuit}
\end{figure}
\begin{equation} \label{eq:cutoffFreq}
    2\pi f_c = \omega_c = \frac{1}{RC}
    \tagaddtext{[\si{\radian\per\second}]}
\end{equation}

\subsubsection{Ruis}
De spectrale ruisdichtheid aan de ingang van het filter is te berekenen met \cref{eq:filterNoiseDensity}.
De spectrale ruisdichtheid aan de uitgang van het filter is hetzelfde als die van de spanningsdeler in \cref{sec:referenceVoltage}. Deze is te berekenen met \cref{eq:dividerNoise}.


% TODO: BEPAAL OVERDRACHT

% \begin{equation} \label{eq:filterTransfer}
%     H(s) = \frac{1}{1+sRC}
% \end{equation}

\begin{equation} \label{eq:filterNoiseDensity}
    S_{u_{in}} = 4kTR
    \tagaddtext{[\si{\volt\squared\per\hertz}]}
\end{equation}

% De signaal-ruis verhouding aan de uitgang van dit filter is te berekenen met \cref{eq:filterSNR}
% \begin{equation}\label{eq:filterSNR}
%     \mathrm{SNR} = 20\log\left(U_{out,min}\sqrt{\frac{C}{kT}}\right)
%     \tagaddtext{[\si{\decibel}]}
% \end{equation}

\subsubsection{Vermogen}
Het vermogensverbruik van het filter is te berekenen met \cref{eq:filterPowerLaplace}.
\begin{equation} \label{eq:filterPowerLaplace}
    P = \frac{U_{in,max}^2}{\left|R + \frac{1}{sC}\right|}
    \tagaddtext{[\si{\watt}]}
\end{equation}
Omdat volgens \cref{eq:cutoffFreq} $R$ te definiëren is in $\omega_c$ en $C$, volgt hieruit \cref{eq:filterPower}.
\begin{equation} \label{eq:filterPower}
    P = \frac{1}{2}\omega_cCU_{in,max}^2
    \tagaddtext{[\si{\watt}]}
\end{equation}
In deze formule is te zien dat het vermogensverbruik lineair evenredig is met de condensatorwaarde. Om het vermogensverbruik te minimaliseren moet dus een zo klein mogelijke condensatorwaarde gekozen worden. Aangezien de noise-figure van dit filter maximaal 3dB mag zijn, mag dit filter maximaal evenveel spanningsruis genereren als het systeem ervoor. Hieruit volgt \cref{eq:filterCapMin}, waarmee de minimale condensatorwaarde te berekenen is. Hierbij is $u_{n,in}$ de ruisspanning aan de ingang van het filter.
\begin{equation} \label{eq:filterCapMin}
    C_{min} = \frac{kT}{u_{n,in}^2}
    \tagaddtext{[\si{\farad}]}
\end{equation}
\begin{frame}
    \frametitle{Ontvangstgevoeligheid}

    \begin{table}
        \centering
        \begin{tabular}{l|c}
            Eigenschap & Waarde \\\hline
            BER & $1\times10^{-5}$ \\
            $\Delta N$ & -105 dBm \\
            Noise Figure & 12.6 dB \\
            Modulatie & GFSK \\
        \end{tabular}
        \caption{Eigenschappen van de ontvanger op het basisstation.}
    \end{table}

    \pause 

    $S_{or}=-57$ dBm bij B=\qty{1}{\mega\hertz}

    $S_{or}=-54$ dBm bij B=\qty{2}{\mega\hertz}

\end{frame}

\begin{frame}
    \frametitle{Minimum zendvermogen}

    \begin{table}
        \centering
        \begin{tabular}{l|c}
            Eigenschap & Waarde \\\hline
            Afstand & \qty{10}{\meter} \\
            Hoogte & \qty{1}{\meter} \\
        \end{tabular}
        \caption{Antenne plaatsing.}
    \end{table}

    Path loss = 53.2 dB

    \pause

    $\Rightarrow$

    $P_{z}=-4$dBm bij B=\qty{1}{\mega\hertz}

    $P_{z}=-1$dBm bij B=\qty{2}{\mega\hertz}
\end{frame}

\begin{frame}
    \frametitle{Energie en gemiddeld vermogen}

    Energie kosten per verzonden pakket:
    \begin{equation*}
        E_{z,p}=\frac{l}{DR}P_z
    \end{equation*}

    \pause

    $E_{z,p}=$\qty{117.8}{\nano\joule} bij B=\qty{1}{\mega\hertz}

    $E_{z,p}=$\qty{117.6}{\nano\joule} bij B=\qty{2}{\mega\hertz}

    \pause

    \vspace{1cm}
    $\overline{P_z}=E_{z,p}f_s$ $\Rightarrow$ \qty{7.1}{\micro\watt} in geval van een bandbreedte van \qty{2}{\mega\hertz}

\end{frame}
\subsection{Batterij} \label{sec:battery}
In onderzoek \cite{BatteryComparison} is te zien dat Lithium-Polymeer batterijen (LiPo) een hoge energiedichtheid hebben in vergelijking met andere soorten batterijen. Er zijn anderen die hogere energie densiteit hebben, maar daarvan zijn de kosten hoog, of zijn er andere nadelige effecten\cite{BatteryComparison}. Dit heeft ervoor gezorgd dat voor de sensor module ontwerp een LiPo gekozen is als batterij. Spanning van een cel LiPo (1s) is maximaal 4.2 V en minimaal veilige spanning is 2.7 V\cite{BatteryComparison}.

\subsection{Voeding} \label{sec:voeding} 

%!! TODO: energy harvesting, spanning, batterij laden, beveiliging, stroom. 

De voedingsspanning is gekozen vanuit de maximale spanning die nodig is voor de ISFET sensor\cite{isfet}. Hieruit volgt een maximale systeemspanning van 3.3 V. 


Zoals te lezen in \cref{sec:battery} is er gekozen voor LiPo batterij technologie. De batterij heeft een beveiliging voor beide op- en ontladen nodig. De celspanning moet omgezet worden naar systeemspanning van 3.3 V. Dit wordt op 2 manieren gedaan, met een DC-DC buck-boost converter en een low dropout regelaar(LDO). De buck-boost is efficiënter dan de LDO. Een voordeel van de LDO is dat de spanningsrimpel veel lager is dan bij een buck-boost. Daarom wordt de LDO gebruikt voor het voeden van de analoge uitleesschakeling. De buck-boost gaat naar het digitale deel. Als een microcontroller goed ontkoppeld is dan maakt het spanningsrimpel niet uit voor de werking van de microcontroller. Daardoor is de hogere rimpel spanning van de buck-boost niet een probleem voor de microcontroller. De voeding is schematisch te zien in \cref{fig:voedingSchematisch}.

Voor energy harvesting is er een piezo element gekozen. Een piezo element kan gezien worden als een AC bron. Deze AC bron moet omgezet worden naar DC die door het systeem gebruikt kan worden om de batterij mee op te laden. De AC bron wordt met een gelijkrichter naar DC omgezet. Deze DC spanning is niet hetzelfde als de systeemspanning dus die moet omgezet worden naar een spanning die de batterij in gaat, zodat de batterij kan opladen.

\begin{figure}[ht]
    \centering 
    \includegraphics{voedingSchematisch.pdf}
    \caption{Voeding schematisch}
    \label{fig:voedingSchematisch}
\end{figure}





\subsection{Energie budget}

\input{sections/energieBudget}






% \begin{table}[ht]
%     \centering
%     \begin{tabular}{|l|r|}
%         \hline     
%         Ingangen    & Oplossing met een pH waarde tussen de 2 en 10 \\
%                     & Een temperatuur tussen de fliep en floep $^\circ$C \\
%         \hline
%         Uitgangen   & Een BLE RF signaal \\
%         \hline
%         Functie     & Meet de pH waarde van een oplossing en stuurt deze op naar een basisstation.  \\
%         \hline
%     \end{tabular}
%     \caption{Specificaties van }
%     \label{tab:in en uitgangen}
% \end{table}


% TODO: Moet gereformateerd zodat het past in dit formaat.
%% Haskell is mijn favoriete taal

Het signaalverwerkingsblok maakt een nuttig signaal van de te meten grootheid. Een uitbreiding van dit blok is te zien in \cref{fig:signaalverwerking}.
Van links naar rechts zijn de functies van de blokken als volgt:
\begin{enumerate}
    \item De grootheid, de pH waarde, wordt gemeten. Dit wordt gedaan door de gate-source spanning $U_{GS}$ van de ISFET te meten.
    \item Het signaal wordt gefilterd om de bandbreedte te limiteren.
    \item Er wordt voor de kruisgevoeligheid van de pH sensor gecompenseerd d.m.v. een temperatuursensor.
    \item De waarde van $U_{GS}$ wordt omgerekend naar de pH waarde.
    \item Deze waarde wordt draadloos opgestuurd naar de ontvanger.
\end{enumerate}
Het `enable' blok wordt gebruikt om het signaalverwerkingsonderdeel van het systeem te activeren en te deactiveren. Op deze manier hoeft het systeem alleen maar aan te staan wanneer het nodig is, en wordt er minder energie verbruikt.

% \begin{figure}[!htbp]
%     \centering
%     \includegraphics[width=\textwidth]{meetGedeelte.pdf}
%     \caption{Het signaalverwerkende onderdeel van het systeem, onderverdeeld naar het analoge en digitale domein.}
%     \label{fig:signaalverwerking}
% \end{figure}



\begin{figure}[!htbp]
    \centering
    \includegraphics[width=0.95\textwidth]{analogeBewerkingsFunctie}
    \caption{Het analoge gedeelte van de signaalbewerking.}
    \label{fig:analogeBewerkingsFunctie}
\end{figure}


\begin{figure}[!htbp]
    \centering
    \includegraphics[width=0.95\textwidth]{digitaleBewerkingsFunctie}
    \caption{Het digitale gedeelte van de signaalbewerking.}
    \label{fig:digitaleBewerkingsFunctie}
\end{figure}



% TODO: TAAL
\subsection{Spanningsregeling}
Voor spanningsregeling zijn er meerdere componenten nodig, zoals te zien in \cref{fig:spanningsregeling}. De energy harvesting produceert een spanning, die omgezet moet worden naar iets de rest van het systeem iets mee kan. Deze omgevormde spanning kan dan zowel gebruikt worden voor het opladen van de batterij, als het voeden van de rest van het systeem.
Om de batterij op te laden is een batterijregelingssysteem (BMS) nodig. De BMS kan via een beveiliging de batterij opladen. De beveiliging limiteert de batterijspanning en -stroom. De tweede beveiliging zit tussen de batterij en de rest van het systeem. Deze beveiliging zorgt ervoor dat er niet te veel stroom uit de batterij getrokken wordt, waardoor deze kapot kan gaan. De spanningsregelaar zet de spanning die uit de batterij komt om naar een spanning die gebruikt kan worden door de rest van het systeem.

\begin{figure}[!htbp]
    \centering
    \includegraphics[width=0.7\textwidth]{spanningsRegeling3.drawio.pdf}
    \caption{Het Spanningsregeling van het systeem.}
    \label{fig:spanningsregeling}
\end{figure}

\subsection{RF}
In \cref{fig:functional} is het TX blok het blok dat de data draadloos verstuurt. Dit blok zal direct ingekocht worden; er zijn meer dan genoeg kant-en-klare oplossingen beschikbaar om de functie van dit blok te vervullen.


\subsection{Microcontroller}
Een gedeelte van de signaalverwerking zal gebeuren in het digitale domein. Hiervoor is een microcontroller de voor de hand liggende oplossing. Bij het kiezen van een microcontroller moet er een aantal eigenschappen overwogen worden. Een paar van deze eigenschappen zijn:
\begin{itemize}
    \item gebruikte vermogen,
    \item mogelijke slaapstanden,
    \item beschikbare peripherals,
    \item kloksnelheid,
    \item geheugen,
    \item programmeergeheugen
    \item en de prijs.
\end{itemize}

