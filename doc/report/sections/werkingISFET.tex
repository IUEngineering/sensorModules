Een ISFET (Ion-Sensitive Field-Effect Transistor) is een FET die gevoelig is voor ionen. Hierdoor is het mogelijk om er pH-waardes mee te meten\cite{modeling}. Een ISFET is in principe een MOSFET zonder gate. De ISFET gedraagt zich in de eerste instantie ook als transistor. Er wordt een referentie-elektrode aan de te meten oplossing toegevoegd om de pH-waarde te kunnen meten. Deze referentie-elektrode kan gebruikt worden als de gate van de MOSFET\cite{van1987isfet}. De drempelspanning van de transistor is echter een functie van de pH-waarde van de gemeten oplossing. Een hogere pH-waarde geeft een lagere drempelspanning\cite{isfet}.