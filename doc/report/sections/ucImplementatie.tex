\subsection{Microcontroller}
Het digitale gedeelte van de implementatie kan opgedeeld worden in 3 onderdelen: de ADC, de digitale signaalverwerking (\cref{fig:digitaleBewerkingsFunctie}) en het draadloos versturen van data. Het is mogelijk om elk van deze onderdelen met aparte componenten te implementeren. Er zijn echter ook componenten beschikbaar die al over elk van deze functionaliteiten beschikken.

Een voorbeeld van een dergelijk component is de nRF52810. Deze microcontroller beschikt over meerdere 14 bit ADC kanalen\footnote{De ADC kanalen zijn alleen 14 bit met oversampling.} en een ingebouwde 2.4GHz Bluetooth transceiver.
Ook heeft de microcontroller een slaapstand die onderbroken kan worden door een ingebouwde RTC, wat nuttig is voor het periodiek samplen en versturen van pH waardes. In deze slaapstand wordt er zo'n \qty{1.5}{\micro\ampere} gebruikt. Met een voedingsspanning van \qty{3.3}{\volt} komt dit uit op een vermogensverbruik van \qty{4.95}{\micro\watt}. Daarbij heeft de microcontroller ook de mogelijkheid om onderdelen van het geheugen uit te zetten, wat tot meer energiebesparing kan leiden \cite{nrf52810}.