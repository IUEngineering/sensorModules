\section{De ISFET als pH sensor}
ISFET staat voor Ion Sensitive Field Effect Transistor. Een ISFET is een FET die gevoelig is voor ionen. Door gebruik te maken van een chemisch filter is de ISFET gevoelig voor H$^+$ ionen. 

\begin{equation}
    pH=-\log_{10}\left(\left[H^+\right]\right)
\end{equation}
$H^+$ is in mol per liter \cite{buck2002measurement}.

Voor FET in saturatie
\begin{equation}
    I_{DS}=\frac{1}{2}\beta\left(U_{GS}-U_{th\left(pH\right)}\right)^2
\end{equation}
Voor FET unsaturated
\begin{equation}
    I_{DS}=\beta U_{DS}\left(U_{GS}-U_{th\left(pH\right)}-\frac{U_{DS}}{2}\right)
\end{equation}
ISFET gedrag afhankelijk van de threshold spanning \cite{martinoia2005modeling}
<Laat afbeelding zien van >
\begin{equation}
    R_{DS}\left(U_{GS},U_{T}\right)=\frac{1}{\mu_n C_{ox}}\frac{L}{W}\frac{1}{U_{GS}-U_T}
\end{equation}
\begin{equation}
    lkj
\end{equation}