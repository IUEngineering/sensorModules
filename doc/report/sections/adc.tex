\subsection{ADC}

% min bits
Een ADC zet analoge signaal om in digitale signalen. Hierbij heeft een ADC een zekere resolutie. Deze resolutie is afhankelijk van het aantal bits dat de ADC heeft. Een andere oorzaak van fouten die bij een ADC kunnen optreden is de sample frequentie. Als deze niet hoog genoeg is zal dit ook een fout creëren.

\subsubsection{Number of bits} \label{sec:ADC:numBits}
De resolutie van een ADC kan worden uitgerekend met \cref{eq:adcRes}, waarbij n het aantal bits van de ADC is.
\begin{equation}\label{eq:adcRes}
    Q=\frac{1}{2^n-1}
\end{equation}

Met \cref{eq:meanSquareErrorADC} is de fout die ontstaat door de eindige resolutie van de ADC te berekenen \cite{MJHcalcADC}. In het geval er uit specificaties een maximale $\overline{e_{eff}^2}$ kan worden gehaald kan met \cref{eq:calcNeededQ} de minimale resolutie worden berekend.
\begin{equation}\label{eq:meanSquareErrorADC} 
    \overline{e_{eff}^2}=\frac{Q^2}{12}
\end{equation}
\begin{equation}\label{eq:calcNeededQ}
    Q=\sqrt{12\cdot\overline{e_{eff}^2}}
\end{equation}

% $\overline{e_{eff}^2}$ mag niet groter dan de helft van het ingangsruis vermogen zijn (noise figure van 1.5dB).
Voor dit ontwerp is er een noise figure gegeven en is de SNR voor het kleinste signaal bekend. Ook is het kleinste ingangssignaal bekend. Door gebruik te maken van \cref{eq:calcSpecifiedRmsError}, is het mogelijk om uit te rekenen hoe groot de fout ten gevolge van de eindige resolutie van de ADC mag zijn.
\begin{equation}\label{eq:calcSpecifiedRmsError}
    \overline{e_{eff}^2}=\left(10^{\frac{NF}{10}}-1\right)\left(\frac{S_{rms}}{10^{SNR+NF/20}}\right)^2
\end{equation}

Door gebruik te maken van \cref{eq:adcRes,eq:meanSquareErrorADC,eq:calcSpecifiedRmsError}, kan het minimum aantal bits van de benodigde ADC berekend worden met \cref{eq:calcMinNumberADCbits}.
\begin{equation}\label{eq:calcMinNumberADCbits}
    n=\left\lceil\log_2\left(\frac{1}{Q}+1\right)\right\rceil
\end{equation}

\subsubsection{Sample frequentie}\label{sec:ADC:sampleFreq}
% min sample rate
De maximale sample rate voor een gegeven ADC is afhankelijk van het aantal bits van de ADC en de hoogste te meten signaal frequentie. De hoogste sample rate zal dan ook geen fouten meer introduceren \cite{MJHcalcADC}. De formule om de hoogste sample rate mee te berekenen is gegeven door \cref{eq:ADCmaxFs}.
\begin{equation}\label{eq:ADCmaxFs}
    f_{s,max}\left(n\right)=2^n\pi f_h
\end{equation}
In veel gevallen is de hoogste sample rate niet van interesse omdat die zo hoog ligt dat het een puur theoretisch getal is. Om een minimale sample frequentie te berekenen moet er eerst een toe te stane fout bepaald worden. Deze fout kan door middel van een noise figure gespecificeerd worden. Met een bekende noise figure kan door gebruik te maken van \cref{eq:ADCmaxSampleError} is er een factor uit te rekenen die gebruikt kan worden in \cref{eq:ADCminFs} om de minimale sample frequentie uit te rekenen.
\begin{equation}\label{eq:ADCmaxSampleError}
    E=10^{\frac{-NF}{10}}
\end{equation}
\begin{equation}\label{eq:ADCminFs}
    f_{s,min}\left(E\right)=\frac{\pi f_h}{E}
\end{equation}