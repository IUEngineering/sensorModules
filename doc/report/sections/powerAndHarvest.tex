\subsection{Energy Harvesting}
In veel toepassingen is het nuttig dat een sensor module weinig onderhoud nodig heeft. Daarbij is het batterijleven van de sensormodule een van de belangrijkste aspecten. Energy harvesting is een manier om de levensuur van apparaten aanzienlijk te kunnen vergroten. Hierbij wordt energie uit de omgeving omgezet naar bruikbare elektrische energie. Er zijn meerdere methodes om energie te verkrijgen uit de omgeving. Er ziin een aantal mogelijke energiebronnen \cite{energyHarvesting}:
\begin{itemize}
    \item Kinetische energie
    \begin{itemize}
        \item Wind
        \item Water
        \item Vibratie
    \end{itemize}
    \item Elektromagnetische energie
    \begin{itemize}
        \item Zonne energie
        \item Radio-frequentie
    \end{itemize}
    \item Thermische energie
    \item Atomische energie
    \begin{itemize}
        \item Radioactief verval
    \end{itemize}
\end{itemize} 

Deze energiebronnen wekken energie op die door de sensor module accu kan worden opgeslagen. Om er een methode te kiezen is het belangrijk om te kijken naar de omgeving waar het module gebruikt gaat worden. Volgens de opdrachtgevers (groep onderzoekers) kan de sensor module binnen worden ingezet. Hierdoor valt een deel van de mogelijke energiebronnen al af: wind en zonne energie. Daarbij is wordt het sensor module gebruikt in een industriële omgeving. Binnen industriële omgevingen worden veel pompen en andere apparaten gebruikt die vibratie veroorzaken. Hierdoor zou je vibratie goed kunnen gebruiken om energie uit te halen. Voor het energie 

% \subsubsection{Voor en nadelen van energiebronnen}

% \begin{table}[!htbp]
%     \centering
%     \begin{tabular}{ | l | c | c | c | }
%         \hline
%         Energiebron & Voordelen & Nadelen \\
%         \hline
%         Wind &   \\
%         Water   \\
%         Vibratie   \\
%         Elektromagnetisme     \\
%         Zonne     \\
%         Radio-frequentie   \\
%         Thermische     \\
%         Atomische     \\
%         Radioactief verval   \\
%         \hline
%     \end{tabular}
%     \caption{Voor en nadelen energiebronnen}
%     \label{tab:fuckingKanker}
% \end{table}