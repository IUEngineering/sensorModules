\subsection{Energy Harvesting}
In veel toepassingen is het nuttig dat een sensormodule weinig onderhoud nodig heeft. Daarbij is het batterijleven van de sensormodule één van de belangrijkste aspecten. Energy harvesting is een manier om de levensuur van apparaten aanzienlijk te kunnen vergroten. Hierbij wordt energie uit de omgeving omgezet naar bruikbare elektrische energie. Er zijn meerdere methodes om energie te verkrijgen uit de omgeving. Voorbeelden van mogelijke energiebronnen zijn \cite{energyHarvesting}:
\begin{itemize}
    \item Kinetische energie
    \begin{itemize}
        \item Wind
        \item Water
        \item Vibratie
    \end{itemize}
    \item Elektromagnetische energie
    \begin{itemize}
        \item Zonne-energie
        \item Radio-frequentie
    \end{itemize}
    \item Thermische energie
    \item Atomische energie
    \begin{itemize}
        \item Radioactief verval
    \end{itemize}
\end{itemize}

Deze energiebronnen wekken energie op die in de sensormodule accu kan worden opgeslagen. Bij het kiezen van één van deze energiebronnen is het belangrijk om te kijken naar de omgeving waar de module gebruikt gaat worden. Volgens de opdrachtgevers zal de sensormodule binnen worden gebruikt. Hierdoor valt een deel van de mogelijke energiebronnen af: wind en zonne-energie. Daarnaast wordt de sensor module in een industriële omgeving gebruikt. Binnen industriële omgevingen worden veel pompen en andere apparaten gebruikt die vibratie veroorzaken. Hierdoor is het aannemelijk dat vibratie een goede energiebron is.

% \subsubsection{Voor en nadelen van energiebronnen}

% \begin{table}[!htb]
%     \centering
%     \begin{tabular}{ | l | c | c | c | }
%         \hline
%         Energiebron & Voordelen & Nadelen \\
%         \hline
%         Wind &   \\
%         Water   \\
%         Vibratie   \\
%         Elektromagnetisme     \\
%         Zonne     \\
%         Radio-frequentie   \\
%         Thermische     \\
%         Atomische     \\
%         Radioactief verval   \\
%         \hline
%     \end{tabular}
%     \caption{Voor en nadelen energiebronnen}
%     \label{tab:fuckingKanker}
% \end{table}

