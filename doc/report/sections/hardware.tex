\subsection{Printplaten}
De schakelingen voor de voeding en voor het uitlezen van de ISFET zijn opgedeeld in twee verschillende printplaten. Dit is gedaan zodat beide schakelingen apart van elkaar getest kunnen worden. De uitlees PCB is te zien in \cref{fig:sensorPCB}. Deze PCB bevat de ISFET uitleesschakeling en de \mcu, die de gemeten pH waarde naar het basisstation opstuurt. De voeding printplaat is te zien in \cref{fig:powerPCB}. Deze PCB regelt de energy harvesting, samen met het veilig opladen en het ontladen van de batterij. De twee printplaten zijn met elkaar verbonden door middel van pin headers. Beide schakelingen zijn te zien \cref{fig:PCBs}.

Beide PCB's hebben op elk belangrijk signaal een testpunt. Op deze manier kan er eenvoudig gemeten worden.


\begin{figure}[!htbp]
    \centering
    \begin{subfigure}[b]{0.48\textwidth}
        \centering
        \includegraphics[width=\textwidth]{sensorBord}
        \caption{De ISFET uitlezende PCB.}
        \label{fig:sensorPCB}
    \end{subfigure}
    \hfill
    \begin{subfigure}[b]{0.60\textwidth}
        \centering
        \includegraphics[width=\textwidth]{powerandharvest}
        \caption{De voeding en harvesting PCB.}
        \label{fig:powerPCB}
    \end{subfigure}
    \caption{De gemaakte PCB's.}
    \label{fig:PCBs}
\end{figure}