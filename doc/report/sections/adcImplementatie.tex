\subsection{ADC} \label{sec:selectingADCandReqs}
De specificaties van de ADC die nodig is voor dit project kunnen berekend worden op basis van de specificaties voor het ADC blok. Deze specificaties staan in \cref{tab:systemSpecADC} samengevat.
\begin{table}[!htbp]
    \centering
    \begin{tabular}{l|c|l}
        Symbol      & Waarde & Eenheid\\\hline
        $SNR_{in}$  & 37        & dB\\
        NF          & 3         & dB\\
        $u_{in}$    & 2.5       & mV\\
    \end{tabular}
    \caption{De eisen voor het omzetten van het analoge signaal naar een digitaal signaal.}
    \label{tab:systemSpecADC}
\end{table}

Door gebruik te maken van de formules uit \cref{sec:ADC:numBits,sec:ADC:sampleFreq} kunnen de specificaties voor de benodigde ADC berekend worden. De resultaten hiervan zijn in  \cref{tab:specADC} geplaatst.

Bij het berekenen van deze specificaties is er van uit gegaan dat het totale noise figure 1 op 1 is verdeeld tussen de resolutie en de bemonsteringsfrequentie.
\begin{table}[!htbp]
    \centering
    \begin{tabular}{l|c|l}
        Symbol      & Waarde    & Eenheid\\\hline
        n           & 14        & bits\\
        $f_{s,min}$ & 45        & Hz\\
        $f_{s,max}$ & 515       & kHz\\
    \end{tabular}
    \caption{De eisen voor het omzetten van het analoge signaal naar een digitaal signaal.}
    \label{tab:specADC}
\end{table}

% Het blijkt het geval te zijn dat de ingebouwde ADC die in de \mcu  zit, voldoet aan de specificaties die in \cref{tab:specADC,tab:systemSpecADC} staan \cite{nrf52810}. Dit zorgt er voor dat het niet nodig is om een externe ADC te gebruiken.