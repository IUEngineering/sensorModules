\section{De drempelspanning van de ISFET uitlezen}

% TODO: Bronnen
Om de pH waarde van de ISFET uit te lezen moet de drempelspanning gemeten worden. Deze is namelijk linear afhankelijk van de pH waarde.
Om de drempelspanning te meten kan een regelsysteem gebruikt worden. Door de gate-source spanning te variëren kan de spanning over en de stroom door de drain en de source gelijk gehouden worden.

Er zijn meerdere mogelijke implementaties van een dergelijk regelsysteem. In \autoref{fig:measureCircuits} staan er drie.
Elk van deze schakelingen gebruikt een nullor om de drain-source spanning van de ISFET gelijk te houden. Ook gebruikt elk van deze schakelingen een referentiespanning. De implementatie van deze referentiespanning wordt verder besproken in \autoref{sec:referenceVoltage}.

De drain-source spanning $U_{ds}$ en drain-source stroom $I_{ds}$ zijn van te voren gedefinieerd. Deze zijn ook te vinden in de datasheet van de ISFET\cite{isfet}. Uit deze twee waardes kunnen de referentiespanningen en weerstandswaardes van de schakelingen gevonden worden.
Voor de schakeling in \autoref{fig:measureCurrent} is de spanningsreferentie te vinden door middel van \autoref{eq:URefSource}.
\begin{equation}\label{eq:URefSource}
    U_{ref,s} = U_{dd} - U_{ds}
\end{equation}
Voor \autoref{fig:measureResistor} is de referentiespanning gelijk aan de drain-source spanning.
\begin{equation}\label{eq:URefDrain}
    U_{ref,d} = U_{ds}
\end{equation}
Voor de waarde van de weerstand in \autoref{fig:measureResistor} kan \autoref{eq:measureResistorVal} gebruikt worden.
\begin{equation}\label{eq:measureResistorVal}
    R = \frac{U_{dd} - U_{ds}}{I_{ds}}
\end{equation}


\begin{figure}[ht]
    \centering
    \begin{subfigure}[b]{0.45\textwidth}
        \centering
        \def\svgwidth{\textwidth}
        \input{img/ISFETCircuitBest.pdf_tex}
        \caption{Met een weerstand aan de drain.}
        \label{fig:measureResistor}
    \end{subfigure}
    \hfill
    \begin{subfigure}[b]{0.45\textwidth}
        \centering
        \def\svgwidth{\textwidth}
        \input{img/ISFETCircuit.pdf_tex}
        \caption{Met een stroombron.}
        \label{fig:measureCurrent}
    \end{subfigure}
    \caption{De uitleesschakelingen voor de ISFET.}
    \label{fig:measureCircuits}
\end{figure}

Beide schakelingen heeft voor- en nadelen.
Bij de schakeling in \autoref{fig:measureResistor} zit de source van de ISFET direct verbonden met de aarde. Dit heeft als voordeel dat de uitgang van de nullor gelijk is aan de gate-source spanning. Hierdoor hoeft de nullor lagere spanningen te genereren om de gate-source spanning van de mosfet naar de goede waarde te krijgen. De spanning die de nullor moet genereren in het geval van een drain weerstand is te vinden door middel van \autoref{eq:nullorVoltageDrain}. In het geval van een stroombron aan de source is dat \autoref{eq:nullorVoltageSource}.

\begin{equation}\label{eq:nullorVoltageDrain}
    U_{nullor,d} = U_{gs}
\end{equation}
\begin{equation}\label{eq:nullorVoltageSource}
    U_{nullor,s} = U_{gs} + U_{ref,s}
\end{equation}

De schakeling met een stroombron aan de source heeft de mogelijkheid om betere ruiseigenschappen te hebben. De stroombron kan ook een hogere impedantie hebben dan de weerstand, wat goed is volgens mij. 

% b d h i q t z