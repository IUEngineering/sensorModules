\subsection{De drempelspanning van de ISFET uitlezen}

% TODO: Bronnen
Om de pH waarde van de ISFET uit te lezen moet de drempelspanning gemeten worden. Deze is namelijk linear afhankelijk van de pH waarde.
Om de drempelspanning te meten kan een regelsysteem gebruikt worden. Door de gate-source spanning te variëren kan de spanning over en de stroom door de drain en de source van de ISFET gelijk gehouden worden.

Er zijn meerdere mogelijke implementaties van een dergelijk regelsysteem. In \autoref{fig:measureCircuits} staan er drie.
Elk van deze schakelingen gebruikt een nullor om de drain-source spanning van de ISFET gelijk te houden. Ook gebruikt elk van deze schakelingen een referentiespanning. De implementatie van deze referentiespanning wordt verder besproken in \autoref{sec:referenceVoltage}.

De drain-source spanning $U_{ds}$ en drain-source stroom $I_{ds}$ zijn van te voren gedefinieerd. Deze zijn ook te vinden in de datasheet van de ISFET\cite{isfet}. Uit deze twee waardes kunnen de referentiespanningen en weerstandswaardes van de schakelingen gevonden worden.
Voor de schakeling in \autoref{fig:measureCurrent} is de spanningsreferentie te vinden door middel van \autoref{eq:URefSource}.
\begin{equation}\label{eq:URefSource}
    U_{ref,s} = U_{dd} - U_{ds}
\end{equation}
Voor \autoref{fig:measureResistor} is de referentiespanning gelijk aan de drain-source spanning.
\begin{equation}\label{eq:URefDrain}
    U_{ref,d} = U_{ds}
\end{equation}
Voor de waarde van de weerstand in \autoref{fig:measureResistor} kan \autoref{eq:measureResistorVal} gebruikt worden.
\begin{equation}\label{eq:measureResistorVal}
    R = \frac{U_{dd} - U_{ds}}{I_{ds}}
\end{equation}


\begin{figure}[ht]
    \centering
    \begin{subfigure}[b]{0.45\textwidth}
        \centering
        \def\svgwidth{\textwidth}
        \input{img/ISFETCircuitBest.pdf_tex}
        \caption{Met een weerstand aan de drain.}
        \label{fig:measureResistor}
    \end{subfigure}
    \hfill
    \begin{subfigure}[b]{0.45\textwidth}
        \centering
        \def\svgwidth{\textwidth}
        \input{img/ISFETCircuit.pdf_tex}
        \caption{Met een stroombron.}
        \label{fig:measureCurrent}
    \end{subfigure}
    \caption{De uitleesschakelingen voor de ISFET.}
    \label{fig:measureCircuits}
\end{figure}

Beide schakelingen heeft voor- en nadelen.
Bij de schakeling in \autoref{fig:measureResistor} zit de source van de ISFET direct verbonden met de aarde. Dit heeft als voordeel dat de uitgang van de nullor gelijk is aan de gate-source spanning. Hierdoor hoeft de nullor lagere spanningen te genereren om de gate-source spanning van de mosfet naar de goede waarde te krijgen. De spanning die de nullor moet genereren in het geval van een drain weerstand is te vinden door middel van \autoref{eq:nullorVoltageDrain}. In het geval van een stroombron aan de source is dat \autoref{eq:nullorVoltageSource}.

\begin{equation}\label{eq:nullorVoltageDrain}
    U_{nullor,d} = U_{gs}
    \tagaddtext{[\si{\volt}]}
\end{equation}
\begin{equation}\label{eq:nullorVoltageSource}
    U_{nullor,s} = U_{gs} + U_{ref,s}
    \tagaddtext{[\si{\volt}]}
\end{equation}

De schakeling met een stroombron aan de source heeft de mogelijkheid om betere ruiseigenschappen te hebben. De stroombron kan ook een hogere impedantie hebben dan de weerstand, wat goed is volgens mij. 

\newcommand\ph{\mathrm{pH}}

De meetschakeling heeft een aantal ruisbronnen. De nullor heeft een ingangsstroom- en spanninsruisbron. Daarnaast genereert de weerstand ook thermische ruis. Deze ruisbronnen zijn te zien in \autoref{fig:measureNoise}.
\begin{figure}[ht]
    \centering
    \def\svgwidth{0.6\textwidth}
    \input{img/ISFETCircuitBestNoise.pdf_tex}
    \caption{De ruisbronnen van de meetschakeling.}
    \label{fig:measureNoise}
\end{figure}

De overdracht van deze schakeling is gelijk aan de uitgangsspanning gedeeld door de ingangsspanning van de nullor. Door de werking van de schakeling blijft de ingansspanning altijd gelijk en is de uitgangsspanning lineair afhankelijk van de pH waarde. Hierdoor is de overdracht $H(\ph)$ een functie van de gemeten pH waarde.
Omdat $U_{ds}$ en $I_{ds}$ van de ISFET niet veranderen, kan de impedantie ervan gezien worden als weerstand, met een waarde van $\frac{U_{ds}}{I_{ds}}$. Hierdoor wordt er een nieuwe ruisbron $i_{n,ds}$ toegevoegd. Met deze weerstand kunnen de bronnen $i_{n,ref}$, $i_{n,R}$ en $i_{n,ds}$ worden getransformeerd naar een spanningsbron $u{n,in}$ aan de ingang van de nullor.
Vervolgens kan deze, samen met de spanningsruisbronnen $u_{n,ref}$ en $u_{n,n}$ naar de uitgang getransformeerd worden. Dit komt uit op een spanningsruisbron aan de uitgang, zoals te zien in \autoref{fig:measureNoiseMoved}. De spectrale spanningsruisdichtheid hiervan is te berekenen door middel van \autoref{eq:measureNoiseOut}.

\begin{equation}\label{eq:measureNoiseOut}
    Su_{n,out} = \left(Su_{n,ref} + Su_{n,n} + Si_{n,in}\left(Z_{fet} // R\right)^2\right) \cdot H^2(\ph)
    \tagaddtext{[\si{\volt\squared\per\hertz}]}
\end{equation}
\begin{equation}
    Si_{n,in} = Si_{n,n} + Si_{n,R} + Si_{n,ds}
    \tagaddtext{[\si{\ampere\squared\per\hertz}]}
    \label{eq:measureNoiseCurrentIn}
\end{equation}


De waardes van deze ruisbronnen zijn te vinden in \autoref{tab:measureNoiseValues}.

\begin{table}[ht]
    \centering
    \begin{tabular}{c|l}
        Ruisbron & Waarde \\
        \hline 
        $Su_{n,ref}$ & Zie \autoref{sec:referenceVoltage} \\
        $Su_{n,n}$   & Implementatie nullor \\
        $Si_{n,n}$   & Implementatie nullor \\
        $Si_{n,R}$   & $\frac{4kT}{R}$ \\
        $Si_{n,ds}$  & $4kT\frac{I_{ds}}{U_{ds}}$ \\
    \end{tabular}
    \caption{Waar de waardes van de ruisbronnen vandaan gehaald kunnen worden.}
    \label{tab:measureNoiseValues}
\end{table}

\begin{figure}[ht]
    \centering
    \def\svgwidth{0.6\textwidth}
    \input{img/ISFETCircuitBestNoiseMoved.pdf_tex}
    \caption{De meetschakeling met verschoven ruisbronnen.}
    \label{fig:measureNoiseMoved}
\end{figure}