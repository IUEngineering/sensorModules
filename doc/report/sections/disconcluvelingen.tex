\section{Discussie}
De uitgang van de uitleesschakeling is op dit moment een oscillerend signaal, zoals besproken in \cref{sec:stabilityTestResults}. De ADC geeft echter wel een waarde die proportioneel lijkt te zijn met de pH waarde van de oplossing die gemeten wordt.
De maximumwaardes van de pieken lijken te veranderen op basis van pH. Hiermee is het echter niet mogelijk om de pH nauwkeurig genoeg te bepalen om aan de specificaties uit \cref{tab:systemSpecs} te voldoen.

Het energie verzamelende piëzo-element produceert op dit moment aanzienlijk minder energie dan in de datasheet gespecificeerd staat. Dit zou kunnen liggen aan de gebruikte testopstelling.

Het totale gemiddelde vermogen dat het systeem gebruikt voldoet aan de \qty{10}{\milli\watt} eis uit \cref{tab:systemSpecs}.

\newpage
\section{Conclusie}
De sensormodule kan op dit moment nog niet goed de pH waarde uitlezen. Dit zorgt ervoor dat de sensormodule nog niet gebruikt kan worden voor het onderzoek doen naar het produceren van drinkwater. Het vermogensverbruik van de module ligt onder het maximum.


\newpage
\section{Aanbevelingen}
Om het systeem te verbeteren moet uitgezocht worden wat de oscillerende uitgang van de uitleesschakeling veroorzaakt. Dit moet verholpen worden voordat het systeem gebruikt kan worden voor het doen van onderzoek naar het produceren van drinkwater.


% zet misschien nog iets over energiezuiniger maken.
% Noem energie waarde uit illya test.