\subsection{De ISFET uitlezen}

In \cref{sec:ontwerp} is voor elk onderdeel van het systeem besproken wat de eisen zijn van dat onderdeel. Dit hoofdstuk gaat in op de implementatie van elk van deze systeemonderdelen.

\subsection{Spanningsreferentie}
Zoals besproken in \cref{sec:referenceVoltage} kunnen de weerstandswaardes van de spanningsreferentie erg hoog gekozen worden. Met een $R_1$ van \qty{5.6}{\mega\ohm} gebruikt de spanningsdeler \qty{1.65}{\micro\watt}.

Volgens \cref{eq:dividerNoise} heeft de condensatorwaarde wel effect op de ruis. Met een condensator van \qty{1}{\micro\farad} produceert de spanningsreferentie \qty{64.4}{\nano\volt} aan ruis. Dit zorgt voor een signaal-ruis verhouding van \qty{138}{\decibel}, wat meer dan genoeg is.

Deze gekozen waardes en de resulterende eigenschappen zijn te vinden in \cref{tab:divider}.

\begin{table}[!htbp]
    \centering
    \begin{tabular}{l|l|l}
        Symbool & Waarde & Eenheid \\
        \hline
        $R_1$       & 5.6  & $\si{\mega\ohm}$   \\
        $R_2$       & 1.0  & $\si{\mega\ohm}$   \\
        $C$         & 1.0  & $\si{\micro\farad}$\\
        $P$         & 1.65 & $\si{\micro\watt}$ \\
        $u_{n,out}$ & 64.4 & $\si{\nano\volt}$  \\
        SNR         & 138  & $\si{\decibel}$
    \end{tabular}
    \caption{De gekozen waardes van de spanningsdeler, met het resulterende vermogensverbruik en de ruiseigenschappen.}
    \label{tab:divider}
\end{table}

\begin{figure}[!htbp]
    \centering
    \def\svgwidth{7cm}
    \subsection{Spanningsreferentie}\label{sec:referenceVoltage}

De ISFET uitleesschakeling heeft een spanningsreferentie nodig om te werken.
% TODO: Vertel misschien over andere methoden.
Hiervoor is als implementatie een spanningsdeler gekozen. De schakeling van deze spanningsdeler is te zien in \cref{fig:divider}.
De condensator wordt gebruikt om ruis te verminderen op hogere frequenties, en dient ook als filter voor hoogfrequente storing in de voedingsspanning.

\begin{figure}[!htbp]
    \centering
    \def\svgwidth{0.5\textwidth}
    \subsection{Spanningsreferentie}\label{sec:referenceVoltage}

De ISFET uitleesschakeling heeft een spanningsreferentie nodig om te werken.
% TODO: Vertel misschien over andere methoden.
Hiervoor is als implementatie een spanningsdeler gekozen. De schakeling van deze spanningsdeler is te zien in \cref{fig:divider}.
De condensator wordt gebruikt om ruis te verminderen op hogere frequenties, en dient ook als filter voor hoogfrequente storing in de voedingsspanning.

\begin{figure}[!htbp]
    \centering
    \def\svgwidth{0.5\textwidth}
    \subsection{Spanningsreferentie}\label{sec:referenceVoltage}

De ISFET uitleesschakeling heeft een spanningsreferentie nodig om te werken.
% TODO: Vertel misschien over andere methoden.
Hiervoor is als implementatie een spanningsdeler gekozen. De schakeling van deze spanningsdeler is te zien in \cref{fig:divider}.
De condensator wordt gebruikt om ruis te verminderen op hogere frequenties, en dient ook als filter voor hoogfrequente storing in de voedingsspanning.

\begin{figure}[!htbp]
    \centering
    \def\svgwidth{0.5\textwidth}
    \input{img/divider.pdf_tex}
    \caption{De schakeling van de spanningsdeler die dient als spanningsreferentie.}
    \label{fig:divider}
\end{figure}

De overdracht van deze spanningsdeler is te vinden in \cref{eq:dividerTransfer}.
\begin{equation}\label{eq:dividerTransfer}
    H(s) = \frac{U_{ref}(s)}{U_{dd}(s)} = \frac{R_2}{R_1 + R_2 + R_2Cs}
\end{equation}

\subsubsection{Vermogen}
Het vermogen dat de spanningsdeler dissipeert, kan met \cref{eq:dividerPower} berekend worden.
\begin{equation}\label{eq:dividerPower}
    P(s) = U_{dd}^2(s)\frac{1+R_2Cs}{R_1 + R_2 + R_1R_2Cs}
    \tagaddtext{[\si{\watt}]}
\end{equation}
Met een constante DC ingangsspanning kan dit vereenvoudigd worden naar \cref{eq:dividerPowerSimple}.
\begin{equation}\label{eq:dividerPowerSimple}
    P = \frac{U_{dd}^2}{R_1 + R_2}
    \tagaddtext{[\si{\watt}]}
\end{equation}

\subsubsection{Ruis}
Om de ruis van deze schakeling te berekenen moet een aantal stappen genomen worden. Aangezien de ingangsbron $U_{dd}$ een spanningsbron is, kan deze als kortsluiting genomen worden. Op deze manier kunnen de twee weerstanden parallel genomen worden, en verandert de schakeling in een simpel RC filter. In \cref{fig:dividerNoise} is deze omgebouwde schakeling te zien.

\begin{figure}[!htbp]
    \centering
    \def\svgwidth{0.35\textwidth}
    \input{img/dividerNoise.pdf_tex}
    \caption{De omgebouwde schakeling om ruis mee te berekenen.}
    \label{fig:dividerNoise}
\end{figure}

\noindent
Voor de spectrale spanningsruisdichtheid aan de uitgang $U_{ref}$ kan \cref{eq:dividerNoiseLaplace} worden opgesteld.
\begin{equation}\label{eq:dividerNoiseLaplace}
    S_{n,u_{ref}} = 4kTR_e\left(\frac{1}{1 + R_eCs}\right)^2
    \tagaddtext{[\si{\volt\squared\per\hertz}]}
\end{equation}
Wanneer de absolute waarde van de ruis wordt genomen, kan deze over de bandbreedte geïntegreerd worden. Dit resulteert in \cref{eq:dividerNoiseInt}, waar B de bandbreedte is.
\begin{equation}\label{eq:dividerNoiseInt}
    u_{n,ref}^2 = 4kTR_e\int_{B} \frac{1}{1 + (2\pi f R_e C)^2} df
    \tagaddtext{[\si{\volt\squared}]}
\end{equation}
Met een oneindige bandbreedte komt deze integraal uit op \cref{eq:dividerNoiseIntegratedInf}.
\begin{equation}\label{eq:dividerNoiseIntegratedInf}
    u_{n,ref}^2 = \lim_{f\rightarrow\infty}\frac{2kT}{\pi C} \arctan(2\pi f R_eC)
    \tagaddtext{[\si{\volt\squared}]}
\end{equation}
Aangezien de inverse tangens $\frac{\pi}{2}$ nadert, komt dit limiet uit op \cref{eq:dividerNoise}.
\begin{equation}\label{eq:dividerNoise}
    u_{n,ref}^2 = \frac{kT}{C}
    \tagaddtext{[\si{\volt\squared}]}
\end{equation}
Omdat een oneindige bandbreedte gebruikt is om op \cref{eq:dividerNoise} te komen, berekend deze de ruis in het ergste geval. De weerstandswaardes van $R_1$ en $R_2$ zijn hierbij irrelevant. Hierdoor is ruis geen bepalende factor meer tijdens het kiezen van de weerstandswaardes van de spanningsdeler, en kunnen deze volledig gebaseerd worden op vermogensverbruik. Volgens \cref{eq:dividerPowerSimple} is het vermogen omgekeerd evenredig met de som van de weerstandswaardes. Daarbij zit de uitgang van de spanningsdeler direct verbonden met de ingang van een nullor. Er hoeft dus geen rekening gehouden te worden met de uitgangsimpedantie van de spanningsbron. Hierdoor is het voor het vermogensverbruik voordelig om de weerstandswaardes zo hoog mogelijk te kiezen.

\subsubsection{Simulatie}

Om te verifiëren dat de spanningsreferentie goed werkt, is er een aantal simulaties uitgevoerd.

In \cref{fig:referenceSimFreq} is het resultaat van een AC simulatie te zien. Hier is $H(f)$ de overdracht van $U_{dd}$ naar

\begin{figure}[!htbp]
    \centering
    \pgfplotsset{width=0.7\textwidth}
    \input{plots/referenceSimFreq.tex}
    \caption{Het resultaat van een AC simulatie van de spanningsreferentie.}
    \label{fig:referenceSimFreq}
\end{figure}


\begin{figure}[!htbp]
    \centering
    \pgfplotsset{width=0.7\textwidth}
    \input{plots/referenceSimTrans.tex}
    \caption{Het resultaat van een transient simulatie van de spanningsreferentie.}
    \label{fig:referenceSimTrans}
\end{figure}


\begin{figure}[!htbp]
    \centering
    \pgfplotsset{width=0.7\textwidth}
    \input{plots/referenceSimNoise.tex}
    \caption{Het resultaat van een ruissimulatie van de spanningsreferentie.}
    \label{fig:referenceSimNoise}
\end{figure}

% 64nV aan ruis
    \caption{De schakeling van de spanningsdeler die dient als spanningsreferentie.}
    \label{fig:divider}
\end{figure}

De overdracht van deze spanningsdeler is te vinden in \cref{eq:dividerTransfer}.
\begin{equation}\label{eq:dividerTransfer}
    H(s) = \frac{U_{ref}(s)}{U_{dd}(s)} = \frac{R_2}{R_1 + R_2 + R_2Cs}
\end{equation}

\subsubsection{Vermogen}
Het vermogen dat de spanningsdeler dissipeert, kan met \cref{eq:dividerPower} berekend worden.
\begin{equation}\label{eq:dividerPower}
    P(s) = U_{dd}^2(s)\frac{1+R_2Cs}{R_1 + R_2 + R_1R_2Cs}
    \tagaddtext{[\si{\watt}]}
\end{equation}
Met een constante DC ingangsspanning kan dit vereenvoudigd worden naar \cref{eq:dividerPowerSimple}.
\begin{equation}\label{eq:dividerPowerSimple}
    P = \frac{U_{dd}^2}{R_1 + R_2}
    \tagaddtext{[\si{\watt}]}
\end{equation}

\subsubsection{Ruis}
Om de ruis van deze schakeling te berekenen moet een aantal stappen genomen worden. Aangezien de ingangsbron $U_{dd}$ een spanningsbron is, kan deze als kortsluiting genomen worden. Op deze manier kunnen de twee weerstanden parallel genomen worden, en verandert de schakeling in een simpel RC filter. In \cref{fig:dividerNoise} is deze omgebouwde schakeling te zien.

\begin{figure}[!htbp]
    \centering
    \def\svgwidth{0.35\textwidth}
    \begin{tikzpicture}
    \pgfplotsset{width=\textwidth}
    \newcommand\BOLZ{1.380649e-23}
    \newcommand\TEMP{300}
    \newcommand\OMEGAC{15*2*pi}
    \newcommand\RESRAT{(7/11)}
    \newcommand\REQU{(1/(1/x + \RESRAT/x))}
    \newcommand\CAP{0.000001}

    \pgfplotsset{set layers}
    \begin{axis}[
        xmode=log,
        ymode=log,
        xlabel={$R_1 [\si{\ohm}]$},
        ylabel={$u_{n,out} [\si{\volt}]$},
        xmin=1e3, xmax=2e6,
        grid=major
    ]

    \addplot [
        red,
        domain=1e3:2e6,
        samples=201
    ]
    {sqrt((4 * \BOLZ * \TEMP / \CAP) * rad(atan(\REQU * \CAP * \OMEGAC)))};
    \end{axis}
\end{tikzpicture}
    \caption{De omgebouwde schakeling om ruis mee te berekenen.}
    \label{fig:dividerNoise}
\end{figure}

\noindent
Voor de spectrale spanningsruisdichtheid aan de uitgang $U_{ref}$ kan \cref{eq:dividerNoiseLaplace} worden opgesteld.
\begin{equation}\label{eq:dividerNoiseLaplace}
    S_{n,u_{ref}} = 4kTR_e\left(\frac{1}{1 + R_eCs}\right)^2
    \tagaddtext{[\si{\volt\squared\per\hertz}]}
\end{equation}
Wanneer de absolute waarde van de ruis wordt genomen, kan deze over de bandbreedte geïntegreerd worden. Dit resulteert in \cref{eq:dividerNoiseInt}, waar B de bandbreedte is.
\begin{equation}\label{eq:dividerNoiseInt}
    u_{n,ref}^2 = 4kTR_e\int_{B} \frac{1}{1 + (2\pi f R_e C)^2} df
    \tagaddtext{[\si{\volt\squared}]}
\end{equation}
Met een oneindige bandbreedte komt deze integraal uit op \cref{eq:dividerNoiseIntegratedInf}.
\begin{equation}\label{eq:dividerNoiseIntegratedInf}
    u_{n,ref}^2 = \lim_{f\rightarrow\infty}\frac{2kT}{\pi C} \arctan(2\pi f R_eC)
    \tagaddtext{[\si{\volt\squared}]}
\end{equation}
Aangezien de inverse tangens $\frac{\pi}{2}$ nadert, komt dit limiet uit op \cref{eq:dividerNoise}.
\begin{equation}\label{eq:dividerNoise}
    u_{n,ref}^2 = \frac{kT}{C}
    \tagaddtext{[\si{\volt\squared}]}
\end{equation}
Omdat een oneindige bandbreedte gebruikt is om op \cref{eq:dividerNoise} te komen, berekend deze de ruis in het ergste geval. De weerstandswaardes van $R_1$ en $R_2$ zijn hierbij irrelevant. Hierdoor is ruis geen bepalende factor meer tijdens het kiezen van de weerstandswaardes van de spanningsdeler, en kunnen deze volledig gebaseerd worden op vermogensverbruik. Volgens \cref{eq:dividerPowerSimple} is het vermogen omgekeerd evenredig met de som van de weerstandswaardes. Daarbij zit de uitgang van de spanningsdeler direct verbonden met de ingang van een nullor. Er hoeft dus geen rekening gehouden te worden met de uitgangsimpedantie van de spanningsbron. Hierdoor is het voor het vermogensverbruik voordelig om de weerstandswaardes zo hoog mogelijk te kiezen.

\subsubsection{Simulatie}

Om te verifiëren dat de spanningsreferentie goed werkt, is er een aantal simulaties uitgevoerd.

In \cref{fig:referenceSimFreq} is het resultaat van een AC simulatie te zien. Hier is $H(f)$ de overdracht van $U_{dd}$ naar

\begin{figure}[!htbp]
    \centering
    \pgfplotsset{width=0.7\textwidth}
    \begin{tikzpicture}
    \tikzset{
        small dot/.style={fill=black,circle,scale=0.4,thick},
    }

    \begin{axis}[
        xmode=log,
        xlabel={$f$ [\unit{\hertz}]},
        ylabel={$H(f)$ [\unit{\decibel}]},
        grid=major,
        height=6cm
    ]
        \addplot [
            mark=none,
            line width=0.5mm
        ] table[x=freq,y=out] {sim/referenceSimFreq.dat};
        % \addplot [
        %     red,
        %     mark=*
        % ] coordinates {(0.18714337, -19.391)};
        \node [small dot,pin={[pin edge={line width=0.3mm,black}]0:kantelpunt}] at (0.18714337, -19.391) {};
    \end{axis}
\end{tikzpicture}


    \caption{Het resultaat van een AC simulatie van de spanningsreferentie.}
    \label{fig:referenceSimFreq}
\end{figure}


\begin{figure}[!htbp]
    \centering
    \pgfplotsset{width=0.7\textwidth}
    \begin{tikzpicture}
    \tikzset{
        small dot/.style={fill=black,circle,scale=0.4},
    }

    \begin{axis}[
        xlabel={$t$ [\unit{\second}]},
        ylabel={$U_{ref}$ [\unit{\volt}]},
        ytick       ={0,0.05,0.1,0.15},
        yticklabels ={0,0.05,0.1,0.15},
        grid=major,
        height=6cm,
    ]
        \addplot [
            mark=none,
            line width=0.5mm
        ] table[x=time,y=out] {sim/referenceSimTrans.dat};
        \node [small dot,pin={[pin edge={line width=0.3mm,black}]0:Voeding wordt geactiveerd}] at (1,0) {};
    \end{axis}


\end{tikzpicture}


    \caption{Het resultaat van een transient simulatie van de spanningsreferentie.}
    \label{fig:referenceSimTrans}
\end{figure}


\begin{figure}[!htbp]
    \centering
    \pgfplotsset{width=0.7\textwidth}
    \begin{tikzpicture}

    \begin{axis}[
        xmode=log,
        xlabel={$f$ [\unit{\hertz}]},
        ylabel={$\sqrt{S_{u,n}} \,\,\,\, \left[\unit{\nano\volt}/\sqrt{\unit{\hertz}}\right]$},
        grid=major,
        height=6cm
    ]
    \addplot [
        mark=none,
        line width=0.5mm,
        y filter/.code={\pgfmathparse{#1*1e9}\pgfmathresult}
    ] table[x=freq,y=noise] {sim/referenceSimNoise.dat};
    \end{axis}
\end{tikzpicture}


    \caption{Het resultaat van een ruissimulatie van de spanningsreferentie.}
    \label{fig:referenceSimNoise}
\end{figure}

% 64nV aan ruis
    \caption{De schakeling van de spanningsdeler die dient als spanningsreferentie.}
    \label{fig:divider}
\end{figure}

De overdracht van deze spanningsdeler is te vinden in \cref{eq:dividerTransfer}.
\begin{equation}\label{eq:dividerTransfer}
    H(s) = \frac{U_{ref}(s)}{U_{dd}(s)} = \frac{R_2}{R_1 + R_2 + R_2Cs}
\end{equation}

\subsubsection{Vermogen}
Het vermogen dat de spanningsdeler dissipeert, kan met \cref{eq:dividerPower} berekend worden.
\begin{equation}\label{eq:dividerPower}
    P(s) = U_{dd}^2(s)\frac{1+R_2Cs}{R_1 + R_2 + R_1R_2Cs}
    \tagaddtext{[\si{\watt}]}
\end{equation}
Met een constante DC ingangsspanning kan dit vereenvoudigd worden naar \cref{eq:dividerPowerSimple}.
\begin{equation}\label{eq:dividerPowerSimple}
    P = \frac{U_{dd}^2}{R_1 + R_2}
    \tagaddtext{[\si{\watt}]}
\end{equation}

\subsubsection{Ruis}
Om de ruis van deze schakeling te berekenen moet een aantal stappen genomen worden. Aangezien de ingangsbron $U_{dd}$ een spanningsbron is, kan deze als kortsluiting genomen worden. Op deze manier kunnen de twee weerstanden parallel genomen worden, en verandert de schakeling in een simpel RC filter. In \cref{fig:dividerNoise} is deze omgebouwde schakeling te zien.

\begin{figure}[!htbp]
    \centering
    \def\svgwidth{0.35\textwidth}
    \begin{tikzpicture}
    \pgfplotsset{width=\textwidth}
    \newcommand\BOLZ{1.380649e-23}
    \newcommand\TEMP{300}
    \newcommand\OMEGAC{15*2*pi}
    \newcommand\RESRAT{(7/11)}
    \newcommand\REQU{(1/(1/x + \RESRAT/x))}
    \newcommand\CAP{0.000001}

    \pgfplotsset{set layers}
    \begin{axis}[
        xmode=log,
        ymode=log,
        xlabel={$R_1 [\si{\ohm}]$},
        ylabel={$u_{n,out} [\si{\volt}]$},
        xmin=1e3, xmax=2e6,
        grid=major
    ]

    \addplot [
        red,
        domain=1e3:2e6,
        samples=201
    ]
    {sqrt((4 * \BOLZ * \TEMP / \CAP) * rad(atan(\REQU * \CAP * \OMEGAC)))};
    \end{axis}
\end{tikzpicture}
    \caption{De omgebouwde schakeling om ruis mee te berekenen.}
    \label{fig:dividerNoise}
\end{figure}

\noindent
Voor de spectrale spanningsruisdichtheid aan de uitgang $U_{ref}$ kan \cref{eq:dividerNoiseLaplace} worden opgesteld.
\begin{equation}\label{eq:dividerNoiseLaplace}
    S_{n,u_{ref}} = 4kTR_e\left(\frac{1}{1 + R_eCs}\right)^2
    \tagaddtext{[\si{\volt\squared\per\hertz}]}
\end{equation}
Wanneer de absolute waarde van de ruis wordt genomen, kan deze over de bandbreedte geïntegreerd worden. Dit resulteert in \cref{eq:dividerNoiseInt}, waar B de bandbreedte is.
\begin{equation}\label{eq:dividerNoiseInt}
    u_{n,ref}^2 = 4kTR_e\int_{B} \frac{1}{1 + (2\pi f R_e C)^2} df
    \tagaddtext{[\si{\volt\squared}]}
\end{equation}
Met een oneindige bandbreedte komt deze integraal uit op \cref{eq:dividerNoiseIntegratedInf}.
\begin{equation}\label{eq:dividerNoiseIntegratedInf}
    u_{n,ref}^2 = \lim_{f\rightarrow\infty}\frac{2kT}{\pi C} \arctan(2\pi f R_eC)
    \tagaddtext{[\si{\volt\squared}]}
\end{equation}
Aangezien de inverse tangens $\frac{\pi}{2}$ nadert, komt dit limiet uit op \cref{eq:dividerNoise}.
\begin{equation}\label{eq:dividerNoise}
    u_{n,ref}^2 = \frac{kT}{C}
    \tagaddtext{[\si{\volt\squared}]}
\end{equation}
Omdat een oneindige bandbreedte gebruikt is om op \cref{eq:dividerNoise} te komen, berekend deze de ruis in het ergste geval. De weerstandswaardes van $R_1$ en $R_2$ zijn hierbij irrelevant. Hierdoor is ruis geen bepalende factor meer tijdens het kiezen van de weerstandswaardes van de spanningsdeler, en kunnen deze volledig gebaseerd worden op vermogensverbruik. Volgens \cref{eq:dividerPowerSimple} is het vermogen omgekeerd evenredig met de som van de weerstandswaardes. Daarbij zit de uitgang van de spanningsdeler direct verbonden met de ingang van een nullor. Er hoeft dus geen rekening gehouden te worden met de uitgangsimpedantie van de spanningsbron. Hierdoor is het voor het vermogensverbruik voordelig om de weerstandswaardes zo hoog mogelijk te kiezen.

\subsubsection{Simulatie}

Om te verifiëren dat de spanningsreferentie goed werkt, is er een aantal simulaties uitgevoerd.

In \cref{fig:referenceSimFreq} is het resultaat van een AC simulatie te zien. Hier is $H(f)$ de overdracht van $U_{dd}$ naar

\begin{figure}[!htbp]
    \centering
    \pgfplotsset{width=0.7\textwidth}
    \begin{tikzpicture}
    \tikzset{
        small dot/.style={fill=black,circle,scale=0.4,thick},
    }

    \begin{axis}[
        xmode=log,
        xlabel={$f$ [\unit{\hertz}]},
        ylabel={$H(f)$ [\unit{\decibel}]},
        grid=major,
        height=6cm
    ]
        \addplot [
            mark=none,
            line width=0.5mm
        ] table[x=freq,y=out] {sim/referenceSimFreq.dat};
        % \addplot [
        %     red,
        %     mark=*
        % ] coordinates {(0.18714337, -19.391)};
        \node [small dot,pin={[pin edge={line width=0.3mm,black}]0:kantelpunt}] at (0.18714337, -19.391) {};
    \end{axis}
\end{tikzpicture}


    \caption{Het resultaat van een AC simulatie van de spanningsreferentie.}
    \label{fig:referenceSimFreq}
\end{figure}


\begin{figure}[!htbp]
    \centering
    \pgfplotsset{width=0.7\textwidth}
    \begin{tikzpicture}
    \tikzset{
        small dot/.style={fill=black,circle,scale=0.4},
    }

    \begin{axis}[
        xlabel={$t$ [\unit{\second}]},
        ylabel={$U_{ref}$ [\unit{\volt}]},
        ytick       ={0,0.05,0.1,0.15},
        yticklabels ={0,0.05,0.1,0.15},
        grid=major,
        height=6cm,
    ]
        \addplot [
            mark=none,
            line width=0.5mm
        ] table[x=time,y=out] {sim/referenceSimTrans.dat};
        \node [small dot,pin={[pin edge={line width=0.3mm,black}]0:Voeding wordt geactiveerd}] at (1,0) {};
    \end{axis}


\end{tikzpicture}


    \caption{Het resultaat van een transient simulatie van de spanningsreferentie.}
    \label{fig:referenceSimTrans}
\end{figure}


\begin{figure}[!htbp]
    \centering
    \pgfplotsset{width=0.7\textwidth}
    \begin{tikzpicture}

    \begin{axis}[
        xmode=log,
        xlabel={$f$ [\unit{\hertz}]},
        ylabel={$\sqrt{S_{u,n}} \,\,\,\, \left[\unit{\nano\volt}/\sqrt{\unit{\hertz}}\right]$},
        grid=major,
        height=6cm
    ]
    \addplot [
        mark=none,
        line width=0.5mm,
        y filter/.code={\pgfmathparse{#1*1e9}\pgfmathresult}
    ] table[x=freq,y=noise] {sim/referenceSimNoise.dat};
    \end{axis}
\end{tikzpicture}


    \caption{Het resultaat van een ruissimulatie van de spanningsreferentie.}
    \label{fig:referenceSimNoise}
\end{figure}

% 64nV aan ruis
    \caption{De spanningsreferentie schakeling, ook te vinden in \cref{fig:divider}.}
    \label{fig:dividerForContext}
\end{figure}

\subsubsection{Simulatie}
Om te verifiëren dat de spanningsreferentie goed werkt, is er een aantal simulaties uitgevoerd.
In \cref{fig:referenceSimFreq} is het resultaat van een AC simulatie te zien. Hier is $H(f)$ de overdracht van de voeding naar de uitgang van de spanningsdeler. De overdracht begint op \qty{-16.4}{\decibel}, aangezien dat de overdracht van de schakeling is op \qty{0}{\hertz}.

\begin{figure}[!htbp]
    \centering
    \pgfplotsset{width=0.7\textwidth}
    \begin{tikzpicture}
    \tikzset{
        small dot/.style={fill=black,circle,scale=0.4,thick},
    }

    \begin{axis}[
        xmode=log,
        xlabel={$f$ [\unit{\hertz}]},
        ylabel={$H(f)$ [\unit{\decibel}]},
        grid=major,
        height=6cm
    ]
        \addplot [
            mark=none,
            line width=0.5mm
        ] table[x=freq,y=out] {sim/referenceSimFreq.dat};
        % \addplot [
        %     red,
        %     mark=*
        % ] coordinates {(0.18714337, -19.391)};
        \node [small dot,pin={[pin edge={line width=0.3mm,black}]0:kantelpunt}] at (0.18714337, -19.391) {};
    \end{axis}
\end{tikzpicture}


    \caption{Het resultaat van een AC simulatie van de spanningsreferentie.}
    \label{fig:referenceSimFreq}
\end{figure}

In \cref{fig:referenceSimTrans} is het resultaat van een transient simulatie te zien. Deze simulatie laat zien dat het 5 seconden duurt voordat de spanningsreferentie de gewenste spanning van \qty{0.15}{\volt} bereikt.
\begin{figure}[!htbp]
    \centering
    \pgfplotsset{width=0.7\textwidth}
    \begin{tikzpicture}
    \tikzset{
        small dot/.style={fill=black,circle,scale=0.4},
    }

    \begin{axis}[
        xlabel={$t$ [\unit{\second}]},
        ylabel={$U_{ref}$ [\unit{\volt}]},
        ytick       ={0,0.05,0.1,0.15},
        yticklabels ={0,0.05,0.1,0.15},
        grid=major,
        height=6cm,
    ]
        \addplot [
            mark=none,
            line width=0.5mm
        ] table[x=time,y=out] {sim/referenceSimTrans.dat};
        \node [small dot,pin={[pin edge={line width=0.3mm,black}]0:Voeding wordt geactiveerd}] at (1,0) {};
    \end{axis}


\end{tikzpicture}


    \caption{Het resultaat van een transient simulatie van de spanningsreferentie.}
    \label{fig:referenceSimTrans}
    % \label{fig:referenceSimCis}
\end{figure}

In \cref{fig:referenceSimNoise} is het resultaat van een ruissimulatie te zien. Hier mag tycho iets over vertellen veel plezier tycho!%TODO: dat
\begin{figure}[!htbp]
    \centering
    \pgfplotsset{width=0.7\textwidth}
    \begin{tikzpicture}

    \begin{axis}[
        xmode=log,
        xlabel={$f$ [\unit{\hertz}]},
        ylabel={$\sqrt{S_{u,n}} \,\,\,\, \left[\unit{\nano\volt}/\sqrt{\unit{\hertz}}\right]$},
        grid=major,
        height=6cm
    ]
    \addplot [
        mark=none,
        line width=0.5mm,
        y filter/.code={\pgfmathparse{#1*1e9}\pgfmathresult}
    ] table[x=freq,y=noise] {sim/referenceSimNoise.dat};
    \end{axis}
\end{tikzpicture}


    \caption{Het resultaat van een ruissimulatie van de spanningsreferentie.}
    \label{fig:referenceSimNoise}
\end{figure}


\subsection{Nullor implementatie}
Voor de nullor die gebruikt wordt om de ISFET uit te lezen in \cref{sec:ISFETLees} moet een implementatie gekozen worden. De uitleesschakeling mag volgens de specificaties maximaal \qty{200}{\micro\watt}  gebruiken. De constante stroom die door de weerstand en ISFET in \cref{fig:measureResistor} heen loopt, zorgt voor een constant vermogensverbruik van \qty{165}{\micro\watt}. Hierdoor mag de nullor implementatie maximaal \qty{35}{\micro\watt} gebruiken. Het maximale dynamische vermogen dat deze nullor implementatie aan de uitgang zal moeten kunnen leveren, is gelijk aan het maximale vermogen dat het filter kan dissiperen. Er blijft dan afgerond nog \qty{34}{\micro\watt} aan statisch vermogen over. Dit resulteert in een maximale quiescent stroom van \qty{10.3}{\micro\ampere}.

De uitleesschakeling moet een minimale SNR hebben van 40 dB. De maximale ruisspanning en stroom die de nullor mag genereren aan de ingang zijn te berekenen met \cref{eq:nullorImplementNoise}. Deze vergelijking is afgeleid uit \Cref{eq:measureNoiseFull}.
\begin{equation} \label{eq:nullorImplementNoise}
    S_{u_{{n,n}}} + S_{i_{{n,in}}}\left(Z_{fet} // R\right)^2 = \frac{S_{u_{{n,out}}}}{H^2(\ph)} - S_{u_{{n,ref}}}
\end{equation}

De LTC2064 opamp heeft (buiten shutdown) een quiescent stroom van \qty{2.5}{\micro\ampere}, wat op \qty{3.3}{\volt} resulteert in een vermogen van \qty{8.25}{\micro\watt}. Daarbij heeft deze een spectrale ruisdichtheid van \qty{12}{\femto\ampere\hertz^{-0.5}} en $\qty{220}{\nano\volt\hertz^{-0.5}}$\cite{LTC2064}. Dit zit volgens \cref{eq:nullorImplementNoise} ver onder het maximum. Daarnaast is zowel de ingangsafwijking als de 1/f ruis van deze opamp erg laag. Hierdoor is deze opamp gekozen voor het ontwerp.


\begin{figure}[!htbp]
    \centering
    \pgfplotsset{width=0.7\textwidth}
    \begin{tikzpicture}
    \tikzset{
        small dot/.style={fill=black,circle,scale=0.4},
    }

    \begin{axis}[
        ylabel={$U_{out}$ [\unit{\volt}]},
        xlabel={$U_{t}$ [\unit{\milli\volt}] - \qty{1.8}{\volt}},
        grid=major,
        height=6cm,
    ]
        \addplot [
            mark=none,
            line width=0.5mm,
        ] table[x=thresh,y=out] {sim/readoutSimTrans.dat};
        % \node [small dot,pin=0:{Voeding wordt geactiveerd}] at (1,0) {};
    \end{axis}
    readoutSimTrans.dat

\end{tikzpicture}


    \caption{Het resultaat van meerdere transient simulaties op de uitleesschakeling. De uitgangsspanning van de schakeling is geplot op basis van de ingestelde drempelspanning van de FET.}
    \label{fig:readoutSimTrans}
\end{figure}