\subsection{De ISFET uitlezen}
Om een ISFET uit te kunnen lezen moet er een ISFET worden gekozen. Voor dit project moet \si{\pH} tot op \qty{0.05}{\pH} worden gemeten\footnote{Zie de specificaties in \cref{sec:systemSpecifications}.}. Om dit te kunnen doen is er een ISFET met een hoge chemische gevoeligheid nodig. Om hieraan te voldoen is er een ISFET nodig die is gebaseerd op $\mathrm{Ta_2O_5}$. De MSFET 3330-2 ISFET van MICROSENS is gebaseerd op $\mathrm{Ta_2O_5}$ en heeft dus een hoge chemische gevoeligheid \cite{bergveld2003thirtyYearsISFET,bergveld1985impactOfMosfetBasedSensors,isfet}.

De MSFET 3330-2 heeft een gevoeligheid van \qty{-55}{\milli\volt\pH^{-1}} \cite{isfet}. Een verandering van \qty{0.05}{\pH} geeft een spanningsverandering van \qty{2.75}{\milli\volt}. Wanneer de \SNR voor dit kleinste signaal minimale \qty{40}{\decibel} moet zijn, mag de ruisvloer niet hoger zijn dan \qty{27.5}{\micro\volt}. Volgens \cref{eq:measureNoiseFull} is de ruisvloer van de uitleesschakeling opgebouwd uit verschillende ruisbronnen. \Cref{eq:measureNoiseFull2} is een herhaling van \cref{eq:measureNoiseFull}.
\begin{equation}\label{eq:measureNoiseFull2}
    S_{u_{{n,out}}} = \left(S_{u_{{n,ref}}} + S_{u_{{n,n}}} + S_{i_{{n,in}}}\left(Z_{fet} // R\right)^2\right) \cdot \left(\frac{U_{o,max}}{U_{ref}}\right)^2
    \tagaddtext{[\si{\volt\squared\per\hertz}]}
\end{equation}

In de datasheet van de MSFET 3330-2 is aangeraden om de ISFET uit te lezen met een drainstroom van \qty{100}{\micro\ampere} en een drain source spanning van \qty{500}{\milli\volt} \cite{isfet}. Het is dan ook mogelijk om de drain source impedantie te berekenen. $Z_{fet}$ komt dan uit op \qty{5}{\kilo\ohm}.
\begin{figure}[!htbp]
    \centering
    \def\svgwidth{0.4\textwidth}
    \input{img/ISFETCircuitBest.pdf_tex}
    \caption{De schakeling die gebruikt zal worden om de ISFET mee uit te lezen.}
    \label{fig:measureResistorImplementatie}
\end{figure}
De weerstand $R$ in \cref{fig:measureResistorImplementatie} wordt gebruikt om de drainstroom in te stellen. Met een voedingsspanning van \qty{3.3}{\volt} moet $R$ \qty{28}{\kilo\ohm} gekozen worden om een drainstroom van \qty{100}{\micro\ampere} te krijgen. Door \cref{eq:measureNoiseFull2} te gebruiken kan berekend worden dat de spanningsreferentie en de nullor implementatie elk niet meer dan \qty{1.3}{\pico\volt^2\hertz^{-1}} aan ruis mogen genereren.

In \cref{sec:energyBudgets} is gespecificeerd dat de ISFET uitleesschakeling niet meer dan \qty{600}{\micro\watt} mag verbruiken. Omdat de drainstroom is ingesteld op \qty{100}{\micro\ampere} en de voedingsspanning \qty{3.3}{\volt} is zal de uitleesschakeling minimaal \qty{330}{\micro\watt} verbruiken. Hierdoor mogen de spanningsreferentie en de nullor implementatie elk maximaal \qty{135}{\micro\watt} verbruiken.

In de aankomende twee subparagrafen zal ingegaan worden op de implementatie van de spanningsreferentie en de implementatie van de nullor. Hierbij zal rekening worden gehouden met de ruis en vermogens eisen die in deze paragraaf aanbod zijn gekomen.

\subsubsection{Spanningsreferentie}
Zoals besproken in \cref{sec:referenceVoltage} kunnen de weerstandswaardes van de spanningsreferentie erg hoog gekozen worden. Met een $R_1$ van \qty{5.6}{\mega\ohm} gebruikt de spanningsdeler \qty{1.65}{\micro\watt}.

Volgens \cref{eq:dividerNoise} heeft de condensatorwaarde wel effect op de ruis. Met een condensator van \qty{1}{\micro\farad} produceert de spanningsreferentie \qty{64.4}{\nano\volt} aan ruis. Dit zorgt voor een signaal-ruis verhouding van \qty{138}{\decibel}, wat meer dan genoeg is.

Deze gekozen waardes en de resulterende eigenschappen zijn te vinden in \cref{tab:divider}.

\begin{table}[!htbp]
    \centering
    \begin{tabular}{l|l|l}
        Symbool & Waarde & Eenheid \\
        \hline
        $R_1$       & 5.6  & $\si{\mega\ohm}$   \\
        $R_2$       & 1.0  & $\si{\mega\ohm}$   \\
        $C$         & 1.0  & $\si{\micro\farad}$\\
        $P$         & 1.65 & $\si{\micro\watt}$ \\
        $u_{n,out}$ & 64.4 & $\si{\nano\volt}$  \\
        SNR         & 138  & $\si{\decibel}$
    \end{tabular}
    \caption{De gekozen waardes van de spanningsdeler, met het resulterende vermogensverbruik en de ruiseigenschappen.}
    \label{tab:divider}
\end{table}

\begin{figure}[!htbp]
    \centering
    \def\svgwidth{7cm}
    \subsection{Spanningsreferentie}\label{sec:referenceVoltage}

De ISFET uitleesschakeling heeft een spanningsreferentie nodig om te werken. Deze spanningsreferentie kan op meerdere manieren gegenereerd worden.
% TODO: Vertel misschien over andere methoden.
Uiteindelijk is er een spanningsdeler gekozen om de spanningsreferentie mee te implementeren. De schakeling van deze spanningsdeler is te zien in \autoref{fig:divider}.
De condensator wordt gebruikt om ruis te verminderen op hogere frequenties, en dient ook als filter voor hoogfrequente fouten in de voedingsspanning.

\begin{figure}[ht]
    \centering
    \def\svgwidth{0.5\textwidth}
    \subsection{Spanningsreferentie}\label{sec:referenceVoltage}

De ISFET uitleesschakeling heeft een spanningsreferentie nodig om te werken. Deze spanningsreferentie kan op meerdere manieren gegenereerd worden.
% TODO: Vertel misschien over andere methoden.
Uiteindelijk is er een spanningsdeler gekozen om de spanningsreferentie mee te implementeren. De schakeling van deze spanningsdeler is te zien in \autoref{fig:divider}.
De condensator wordt gebruikt om ruis te verminderen op hogere frequenties, en dient ook als filter voor hoogfrequente fouten in de voedingsspanning.

\begin{figure}[ht]
    \centering
    \def\svgwidth{0.5\textwidth}
    \subsection{Spanningsreferentie}\label{sec:referenceVoltage}

De ISFET uitleesschakeling heeft een spanningsreferentie nodig om te werken. Deze spanningsreferentie kan op meerdere manieren gegenereerd worden.
% TODO: Vertel misschien over andere methoden.
Uiteindelijk is er een spanningsdeler gekozen om de spanningsreferentie mee te implementeren. De schakeling van deze spanningsdeler is te zien in \autoref{fig:divider}.
De condensator wordt gebruikt om ruis te verminderen op hogere frequenties, en dient ook als filter voor hoogfrequente fouten in de voedingsspanning.

\begin{figure}[ht]
    \centering
    \def\svgwidth{0.5\textwidth}
    \input{img/divider.pdf_tex}
    \caption{De schakeling van de spanningsdeler die dient als spanningsreferentie.}
    \label{fig:divider}
\end{figure}

\noindent
De overdracht van deze spanningsdeler is te vinden in \autoref{eq:dividerTransfer}.
\begin{equation}\label{eq:dividerTransfer}
    H(s) = \frac{U_{ref}(s)}{U_{dd}(s)} = \frac{R_2}{R_1 + R_2 + R_2Cs}
\end{equation}

\noindent
Het vermogen dat de spanningsdeler dissipeert, kan met \autoref{eq:dividerPower} berekend worden.
\begin{equation}\label{eq:dividerPower}
    P(s) = U_{dd}(s)^2\frac{1+R_2Cs}{R_1 + R_2 + R_1R_2Cs}
\end{equation}
Met een constante DC ingangsspanning kan dit vereenvoudigd worden naar \autoref{eq:dividerPowerSimple}.
\begin{equation}\label{eq:dividerPowerSimple}
    P = \frac{U_{dd}^2}{R_1 + R_2}
\end{equation}

\noindent
Om de ruis van deze schakeling te berekenen moet een aantal stappen genomen worden. Aangezien de ingangsbron $U_{dd}$ een spanningsbron is, kan deze als kortsluiting genomen worden. Op deze manier kunnen de twee weerstanden parallel genomen worden, en verandert de schakeling in een simpel RC filter. In \autoref{fig:dividerNoise} is deze omgebouwde schakeling te zien.

\begin{figure}[ht]
    \centering
    \def\svgwidth{0.35\textwidth}
    \input{img/dividerNoise.pdf_tex}
    \caption{De omgebouwde schakeling om ruis mee te berekenen.}
    \label{fig:dividerNoise}
\end{figure}

\noindent
Voor de spectrale spanningsruisdichtheid aan de uitgang $U_{ref}$ kan \autoref{eq:dividerNoiseLaplace} worden opgesteld.
\begin{equation}\label{eq:dividerNoiseLaplace}
    S_{n,u_{ref}} = 4kTR_e\left(\frac{1}{1 + R_eCs}\right)^2
\end{equation}
Wanneer de absolute waarde van de ruis wordt genomen, kan deze over de bandbreedte geïntegreerd worden. Dit resulteert in \autoref{eq:dividerNoiseInt}.
\begin{equation}\label{eq:dividerNoiseInt}
    u_{n,ref}^2 = \int_{\omega_l}^{\omega_h} 4kTR_e\left(\frac{1}{\sqrt{1 + (R_eC\omega)^2}}\right)^2 d\omega
\end{equation}
Het integraal van deze formule komt uit op \autoref{eq:dividerNoiseIntegrated}.
\begin{equation}\label{eq:dividerNoiseIntegrated}
    u_{n,ref}^2 = \frac{4kT}{C}\left[\arctan(R_eC\omega_h) - \arctan(R_eC\omega_l)\right]
\end{equation}

Zolang de spanning stabiel is hoeft de referentie geen exact gedefinieerde spanning te hebben. Dit is omdat de ADC die de waarde van de sensor gaat uitlezen deze referentiespanning ook als referentie zal gebruiken. De waarde moet echter ergens rond de 1.1V zitten, om de stroombron te laten werken.
In \autoref{sec:currentSource} is hier meer over te lezen.
Aangezien de ingangsspanning 1.8V is, zal de DC overdracht $1.1 / 1.8 = \frac{11}{18}$ zijn. Hieruit komt de weerstandsverhouding in \autoref{eq:dividerResistors}.
\begin{equation}\label{eq:dividerResistors}
    \frac{R_1}{R_2} = \frac{7}{11}
\end{equation}
Een hogere $R_1$ zorgt voor een lager vermogensverbruik, maar ook een hogere ruis. Dit is te zien in \autoref{fig:dividerPlots}.

\begin{figure}
    \centering
    \begin{subfigure}[b]{0.45\textwidth}
        \centering
        \input{plots/dividerNoise}
        \caption{Spanningsruis}
        \label{fig:dividerNoisePlot}
    \end{subfigure}
    \hfill
    \begin{subfigure}[b]{0.45\textwidth}
        \centering
        \input{plots/dividerPower}
        \caption{Vermogensverbruik}
        \label{fig:dividerPower}
    \end{subfigure}
    \caption{De ruis en het vermogensverbruik van de spanningsdeler, ten opzichte van de gekozen weerstandswaarde $R_1$.}
    \label{fig:dividerPlots}
\end{figure}
Op $R_1 = 1\si{\mega\ohm}$ en $C = 10\si{\micro\farad}$ is de spanningsruis $u_{n,ref} = 51\si{\nano\volt}$ en het vermogensverbruik $P = 1.3\si{\micro\watt}$. De andere weerstandswaarde is dan $R_2 \approx 1.6 \si{\mega\ohm}$. Deze waardes vallen binnen de specificaties.

[WELKE SPECS??]
    \caption{De schakeling van de spanningsdeler die dient als spanningsreferentie.}
    \label{fig:divider}
\end{figure}

\noindent
De overdracht van deze spanningsdeler is te vinden in \autoref{eq:dividerTransfer}.
\begin{equation}\label{eq:dividerTransfer}
    H(s) = \frac{U_{ref}(s)}{U_{dd}(s)} = \frac{R_2}{R_1 + R_2 + R_2Cs}
\end{equation}

\noindent
Het vermogen dat de spanningsdeler dissipeert, kan met \autoref{eq:dividerPower} berekend worden.
\begin{equation}\label{eq:dividerPower}
    P(s) = U_{dd}(s)^2\frac{1+R_2Cs}{R_1 + R_2 + R_1R_2Cs}
\end{equation}
Met een constante DC ingangsspanning kan dit vereenvoudigd worden naar \autoref{eq:dividerPowerSimple}.
\begin{equation}\label{eq:dividerPowerSimple}
    P = \frac{U_{dd}^2}{R_1 + R_2}
\end{equation}

\noindent
Om de ruis van deze schakeling te berekenen moet een aantal stappen genomen worden. Aangezien de ingangsbron $U_{dd}$ een spanningsbron is, kan deze als kortsluiting genomen worden. Op deze manier kunnen de twee weerstanden parallel genomen worden, en verandert de schakeling in een simpel RC filter. In \autoref{fig:dividerNoise} is deze omgebouwde schakeling te zien.

\begin{figure}[ht]
    \centering
    \def\svgwidth{0.35\textwidth}
    \begin{tikzpicture}
    \pgfplotsset{width=\textwidth}
    \newcommand\BOLZ{1.380649e-23}
    \newcommand\TEMP{300}
    \newcommand\OMEGAC{15*2*pi}
    \newcommand\RESRAT{(7/11)}
    \newcommand\REQU{(1/(1/x + \RESRAT/x))}
    \newcommand\CAP{0.000001}

    \pgfplotsset{set layers}
    \begin{axis}[
        xmode=log,
        ymode=log,
        xlabel={$R_1 [\si{\ohm}]$},
        ylabel={$u_{n,out} [\si{\volt}]$},
        xmin=1e3, xmax=1e7,
        grid=major
    ]

    \addplot [
        red,
        domain=1e3:1e7,
        samples=201
    ]
    {sqrt((4 * \BOLZ * \TEMP / \CAP) * rad(atan(\REQU * \CAP * \OMEGAC)))};
    \end{axis}
\end{tikzpicture}
    \caption{De omgebouwde schakeling om ruis mee te berekenen.}
    \label{fig:dividerNoise}
\end{figure}

\noindent
Voor de spectrale spanningsruisdichtheid aan de uitgang $U_{ref}$ kan \autoref{eq:dividerNoiseLaplace} worden opgesteld.
\begin{equation}\label{eq:dividerNoiseLaplace}
    S_{n,u_{ref}} = 4kTR_e\left(\frac{1}{1 + R_eCs}\right)^2
\end{equation}
Wanneer de absolute waarde van de ruis wordt genomen, kan deze over de bandbreedte geïntegreerd worden. Dit resulteert in \autoref{eq:dividerNoiseInt}.
\begin{equation}\label{eq:dividerNoiseInt}
    u_{n,ref}^2 = \int_{\omega_l}^{\omega_h} 4kTR_e\left(\frac{1}{\sqrt{1 + (R_eC\omega)^2}}\right)^2 d\omega
\end{equation}
Het integraal van deze formule komt uit op \autoref{eq:dividerNoiseIntegrated}.
\begin{equation}\label{eq:dividerNoiseIntegrated}
    u_{n,ref}^2 = \frac{4kT}{C}\left[\arctan(R_eC\omega_h) - \arctan(R_eC\omega_l)\right]
\end{equation}

Zolang de spanning stabiel is hoeft de referentie geen exact gedefinieerde spanning te hebben. Dit is omdat de ADC die de waarde van de sensor gaat uitlezen deze referentiespanning ook als referentie zal gebruiken. De waarde moet echter ergens rond de 1.1V zitten, om de stroombron te laten werken.
In \autoref{sec:currentSource} is hier meer over te lezen.
Aangezien de ingangsspanning 1.8V is, zal de DC overdracht $1.1 / 1.8 = \frac{11}{18}$ zijn. Hieruit komt de weerstandsverhouding in \autoref{eq:dividerResistors}.
\begin{equation}\label{eq:dividerResistors}
    \frac{R_1}{R_2} = \frac{7}{11}
\end{equation}
Een hogere $R_1$ zorgt voor een lager vermogensverbruik, maar ook een hogere ruis. Dit is te zien in \autoref{fig:dividerPlots}.

\begin{figure}
    \centering
    \begin{subfigure}[b]{0.45\textwidth}
        \centering
        \begin{tikzpicture}
    \pgfplotsset{width=\textwidth}
    \newcommand\BOLZ{1.380649e-23}
    \newcommand\TEMP{300}
    \newcommand\OMEGAC{15*2*pi}
    \newcommand\RESRAT{(7/11)}
    \newcommand\REQU{(1/(1/x + \RESRAT/x))}
    \newcommand\CAP{0.000001}

    \pgfplotsset{set layers}
    \begin{axis}[
        xmode=log,
        ymode=log,
        xlabel={$R_1 [\si{\ohm}]$},
        ylabel={$u_{n,out} [\si{\volt}]$},
        xmin=1e3, xmax=1e7,
        grid=major
    ]

    \addplot [
        red,
        domain=1e3:1e7,
        samples=201
    ]
    {sqrt((4 * \BOLZ * \TEMP / \CAP) * rad(atan(\REQU * \CAP * \OMEGAC)))};
    \end{axis}
\end{tikzpicture}
        \caption{Spanningsruis}
        \label{fig:dividerNoisePlot}
    \end{subfigure}
    \hfill
    \begin{subfigure}[b]{0.45\textwidth}
        \centering
        \begin{tikzpicture}
    \pgfplotsset{width=\textwidth}
    \newcommand\OMEGAC{10*2*pi}
    \newcommand\RESRAT{(7/11)}

    \begin{axis}[
        xmode=log,
        ymode=log,
        xlabel={$R_1 [\si{\ohm}]$},
        ylabel={$P [\si{\watt}]$},
        xmin=1e3, xmax=2e6,
        grid=major
    ]

    \addplot [
        blue,
        domain=1e3:2e6,
        samples=201
    ]
    {1.8 / (x + x/\RESRAT)};
    \end{axis}
\end{tikzpicture}
        \caption{Vermogensverbruik}
        \label{fig:dividerPower}
    \end{subfigure}
    \caption{De ruis en het vermogensverbruik van de spanningsdeler, ten opzichte van de gekozen weerstandswaarde $R_1$.}
    \label{fig:dividerPlots}
\end{figure}
Op $R_1 = 1\si{\mega\ohm}$ en $C = 10\si{\micro\farad}$ is de spanningsruis $u_{n,ref} = 51\si{\nano\volt}$ en het vermogensverbruik $P = 1.3\si{\micro\watt}$. De andere weerstandswaarde is dan $R_2 \approx 1.6 \si{\mega\ohm}$. Deze waardes vallen binnen de specificaties.

[WELKE SPECS??]
    \caption{De schakeling van de spanningsdeler die dient als spanningsreferentie.}
    \label{fig:divider}
\end{figure}

\noindent
De overdracht van deze spanningsdeler is te vinden in \autoref{eq:dividerTransfer}.
\begin{equation}\label{eq:dividerTransfer}
    H(s) = \frac{U_{ref}(s)}{U_{dd}(s)} = \frac{R_2}{R_1 + R_2 + R_2Cs}
\end{equation}

\noindent
Het vermogen dat de spanningsdeler dissipeert, kan met \autoref{eq:dividerPower} berekend worden.
\begin{equation}\label{eq:dividerPower}
    P(s) = U_{dd}(s)^2\frac{1+R_2Cs}{R_1 + R_2 + R_1R_2Cs}
\end{equation}
Met een constante DC ingangsspanning kan dit vereenvoudigd worden naar \autoref{eq:dividerPowerSimple}.
\begin{equation}\label{eq:dividerPowerSimple}
    P = \frac{U_{dd}^2}{R_1 + R_2}
\end{equation}

\noindent
Om de ruis van deze schakeling te berekenen moet een aantal stappen genomen worden. Aangezien de ingangsbron $U_{dd}$ een spanningsbron is, kan deze als kortsluiting genomen worden. Op deze manier kunnen de twee weerstanden parallel genomen worden, en verandert de schakeling in een simpel RC filter. In \autoref{fig:dividerNoise} is deze omgebouwde schakeling te zien.

\begin{figure}[ht]
    \centering
    \def\svgwidth{0.35\textwidth}
    \begin{tikzpicture}
    \pgfplotsset{width=\textwidth}
    \newcommand\BOLZ{1.380649e-23}
    \newcommand\TEMP{300}
    \newcommand\OMEGAC{15*2*pi}
    \newcommand\RESRAT{(7/11)}
    \newcommand\REQU{(1/(1/x + \RESRAT/x))}
    \newcommand\CAP{0.000001}

    \pgfplotsset{set layers}
    \begin{axis}[
        xmode=log,
        ymode=log,
        xlabel={$R_1 [\si{\ohm}]$},
        ylabel={$u_{n,out} [\si{\volt}]$},
        xmin=1e3, xmax=1e7,
        grid=major
    ]

    \addplot [
        red,
        domain=1e3:1e7,
        samples=201
    ]
    {sqrt((4 * \BOLZ * \TEMP / \CAP) * rad(atan(\REQU * \CAP * \OMEGAC)))};
    \end{axis}
\end{tikzpicture}
    \caption{De omgebouwde schakeling om ruis mee te berekenen.}
    \label{fig:dividerNoise}
\end{figure}

\noindent
Voor de spectrale spanningsruisdichtheid aan de uitgang $U_{ref}$ kan \autoref{eq:dividerNoiseLaplace} worden opgesteld.
\begin{equation}\label{eq:dividerNoiseLaplace}
    S_{n,u_{ref}} = 4kTR_e\left(\frac{1}{1 + R_eCs}\right)^2
\end{equation}
Wanneer de absolute waarde van de ruis wordt genomen, kan deze over de bandbreedte geïntegreerd worden. Dit resulteert in \autoref{eq:dividerNoiseInt}.
\begin{equation}\label{eq:dividerNoiseInt}
    u_{n,ref}^2 = \int_{\omega_l}^{\omega_h} 4kTR_e\left(\frac{1}{\sqrt{1 + (R_eC\omega)^2}}\right)^2 d\omega
\end{equation}
Het integraal van deze formule komt uit op \autoref{eq:dividerNoiseIntegrated}.
\begin{equation}\label{eq:dividerNoiseIntegrated}
    u_{n,ref}^2 = \frac{4kT}{C}\left[\arctan(R_eC\omega_h) - \arctan(R_eC\omega_l)\right]
\end{equation}

Zolang de spanning stabiel is hoeft de referentie geen exact gedefinieerde spanning te hebben. Dit is omdat de ADC die de waarde van de sensor gaat uitlezen deze referentiespanning ook als referentie zal gebruiken. De waarde moet echter ergens rond de 1.1V zitten, om de stroombron te laten werken.
In \autoref{sec:currentSource} is hier meer over te lezen.
Aangezien de ingangsspanning 1.8V is, zal de DC overdracht $1.1 / 1.8 = \frac{11}{18}$ zijn. Hieruit komt de weerstandsverhouding in \autoref{eq:dividerResistors}.
\begin{equation}\label{eq:dividerResistors}
    \frac{R_1}{R_2} = \frac{7}{11}
\end{equation}
Een hogere $R_1$ zorgt voor een lager vermogensverbruik, maar ook een hogere ruis. Dit is te zien in \autoref{fig:dividerPlots}.

\begin{figure}
    \centering
    \begin{subfigure}[b]{0.45\textwidth}
        \centering
        \begin{tikzpicture}
    \pgfplotsset{width=\textwidth}
    \newcommand\BOLZ{1.380649e-23}
    \newcommand\TEMP{300}
    \newcommand\OMEGAC{15*2*pi}
    \newcommand\RESRAT{(7/11)}
    \newcommand\REQU{(1/(1/x + \RESRAT/x))}
    \newcommand\CAP{0.000001}

    \pgfplotsset{set layers}
    \begin{axis}[
        xmode=log,
        ymode=log,
        xlabel={$R_1 [\si{\ohm}]$},
        ylabel={$u_{n,out} [\si{\volt}]$},
        xmin=1e3, xmax=1e7,
        grid=major
    ]

    \addplot [
        red,
        domain=1e3:1e7,
        samples=201
    ]
    {sqrt((4 * \BOLZ * \TEMP / \CAP) * rad(atan(\REQU * \CAP * \OMEGAC)))};
    \end{axis}
\end{tikzpicture}
        \caption{Spanningsruis}
        \label{fig:dividerNoisePlot}
    \end{subfigure}
    \hfill
    \begin{subfigure}[b]{0.45\textwidth}
        \centering
        \begin{tikzpicture}
    \pgfplotsset{width=\textwidth}
    \newcommand\OMEGAC{10*2*pi}
    \newcommand\RESRAT{(7/11)}

    \begin{axis}[
        xmode=log,
        ymode=log,
        xlabel={$R_1 [\si{\ohm}]$},
        ylabel={$P [\si{\watt}]$},
        xmin=1e3, xmax=2e6,
        grid=major
    ]

    \addplot [
        blue,
        domain=1e3:2e6,
        samples=201
    ]
    {1.8 / (x + x/\RESRAT)};
    \end{axis}
\end{tikzpicture}
        \caption{Vermogensverbruik}
        \label{fig:dividerPower}
    \end{subfigure}
    \caption{De ruis en het vermogensverbruik van de spanningsdeler, ten opzichte van de gekozen weerstandswaarde $R_1$.}
    \label{fig:dividerPlots}
\end{figure}
Op $R_1 = 1\si{\mega\ohm}$ en $C = 10\si{\micro\farad}$ is de spanningsruis $u_{n,ref} = 51\si{\nano\volt}$ en het vermogensverbruik $P = 1.3\si{\micro\watt}$. De andere weerstandswaarde is dan $R_2 \approx 1.6 \si{\mega\ohm}$. Deze waardes vallen binnen de specificaties.

[WELKE SPECS??]
    \caption{De spanningsreferentie schakeling, ook te vinden in \cref{fig:divider}.}
    \label{fig:dividerForContext}
\end{figure}

% \paragraph{Simulatie}
Om te verifiëren dat de spanningsreferentie goed werkt, is er een aantal simulaties uitgevoerd.
In \cref{fig:referenceSimFreq} is het resultaat van een AC simulatie te zien. Hier is $H(f)$ de overdracht van de voeding naar de uitgang van de spanningsdeler. De overdracht begint op \qty{-16.4}{\decibel}, aangezien dat de overdracht van de schakeling is op \qty{0}{\hertz}.

\begin{figure}[!htbp]
    \centering
    \pgfplotsset{width=0.7\textwidth}
    \begin{tikzpicture}
    \tikzset{
        small dot/.style={fill=black,circle,scale=0.3},
    }

    \begin{axis}[
        xmode=log,
        xlabel={$f$ [\unit{\hertz}]},
        ylabel={$H(f)$ [\unit{\decibel}]},
        grid=major
    ]
        \addplot [
            mark=none,
            line width=0.5mm
        ] table[x=freq,y=out] {sim/referenceSimFreq.dat};
        % \addplot [
        %     red,
        %     mark=*
        % ] coordinates {(0.18714337, -19.391)};
        \node [small dot,pin=0:{kantelpunt}] at (0.18714337, -19.391) {};
    \end{axis}
\end{tikzpicture}
    \caption{Het resultaat van een AC simulatie van de spanningsreferentie.}
    \label{fig:referenceSimFreq}
\end{figure}

In \cref{fig:referenceSimTrans} is het resultaat van een transient simulatie te zien. Deze simulatie laat zien dat het 5 seconden duurt voordat de spanningsreferentie de gewenste spanning van \qty{0.15}{\volt} bereikt.
\begin{figure}[!htbp]
    \centering
    \pgfplotsset{width=0.7\textwidth}
    \begin{tikzpicture}

    \begin{axis}[
        xlabel={$t$ [\unit{\second}]},
        ylabel={$H(f)$ [\unit{\decibel}]},
        grid=major
    ]
    \addplot [
        mark=none,
        line width=0.5mm
    ] table[x=time,y=out] {sim/referenceSimTrans.dat};
    \end{axis}
\end{tikzpicture}


    \caption{Het resultaat van een transient simulatie van de spanningsreferentie.}
    \label{fig:referenceSimTrans}
    % \label{fig:referenceSimCis}
\end{figure}

In \cref{fig:referenceSimNoise} is het resultaat van een ruissimulatie te zien. Hier mag tycho iets over vertellen veel plezier tycho!%TODO: dat
\begin{figure}[!htbp]
    \centering
    \pgfplotsset{width=0.7\textwidth}
    \begin{tikzpicture}

    \begin{axis}[
        xmode=log,
        xlabel={$f$ [\unit{\hertz}]},
        ylabel={$\sqrt{S_{u,n}} \,\,\,\, \left[\unit{\nano\volt}/\sqrt{\unit{\hertz}}\right]$},
        grid=major,
        height=6cm
    ]
    \addplot [
        mark=none,
        line width=0.5mm,
        y filter/.code={\pgfmathparse{#1*1e9}\pgfmathresult}
    ] table[x=freq,y=noise] {sim/referenceSimNoise.dat};
    \end{axis}
\end{tikzpicture}


    \caption{Het resultaat van een ruissimulatie van de spanningsreferentie.}
    \label{fig:referenceSimNoise}
\end{figure}


\subsubsection{Nullor implementatie}
Voor de nullor die gebruikt wordt om de ISFET uit te lezen in \cref{sec:ISFETLees} moet een implementatie gekozen worden. De uitleesschakeling mag volgens de specificaties maximaal \qty{200}{\micro\watt}  gebruiken. De constante stroom die door de weerstand en ISFET in \cref{fig:measureResistor} heen loopt, zorgt voor een constant vermogensverbruik van \qty{165}{\micro\watt}. Hierdoor mag de nullor implementatie maximaal \qty{35}{\micro\watt} gebruiken. Het maximale dynamische vermogen dat deze nullor implementatie aan de uitgang zal moeten kunnen leveren, is gelijk aan het maximale vermogen dat het filter kan dissiperen. Er blijft dan afgerond nog \qty{34}{\micro\watt} aan statisch vermogen over. Dit resulteert in een maximale quiescent stroom van \qty{10.3}{\micro\ampere}.

De uitleesschakeling moet een minimale SNR hebben van 40 dB. De maximale ruisspanning en stroom die de nullor mag genereren aan de ingang zijn te berekenen met \cref{eq:nullorImplementNoise}. Deze vergelijking is afgeleid uit \Cref{eq:measureNoiseFull}.
\begin{equation} \label{eq:nullorImplementNoise}
    S_{u_{{n,n}}} + S_{i_{{n,in}}}\left(Z_{fet} // R\right)^2 = \frac{S_{u_{{n,out}}}}{H^2(\ph)} - S_{u_{{n,ref}}}
\end{equation}

De LTC2064 opamp heeft (buiten shutdown) een quiescent stroom van \qty{2.5}{\micro\ampere}, wat op \qty{3.3}{\volt} resulteert in een vermogen van \qty{8.25}{\micro\watt}. Daarbij heeft deze een spectrale ruisdichtheid van \qty{12}{\femto\ampere\hertz^{-0.5}} en $\qty{220}{\nano\volt\hertz^{-0.5}}$\cite{LTC2064}. Dit zit volgens \cref{eq:nullorImplementNoise} ver onder het maximum. Daarnaast is zowel de ingangsafwijking als de 1/f ruis van deze opamp erg laag. Hierdoor is deze opamp gekozen voor het ontwerp.


\begin{figure}[!htbp]
    \centering
    \pgfplotsset{width=0.7\textwidth}
    \begin{tikzpicture}
    \tikzset{
        small dot/.style={fill=black,circle,scale=0.4},
    }

    \begin{axis}[
        ylabel={$U_{out}$ [\unit{\volt}]},
        xlabel={$U_{t}$ [\unit{\milli\volt}] - \qty{1.8}{\volt}},
        grid=major,
        height=6cm,
    ]
        \addplot [
            mark=none,
            line width=0.5mm,
        ] table[x=thresh,y=out] {sim/readoutSimTrans.dat};
        % \node [small dot,pin=0:{Voeding wordt geactiveerd}] at (1,0) {};
    \end{axis}
\end{tikzpicture}


    \caption{Het resultaat van meerdere transient simulaties op de uitleesschakeling. De uitgangsspanning van de schakeling is geplot op basis van de ingestelde drempelspanning van de FET.}
    \label{fig:readoutSimTrans}
\end{figure}